%!TEX root = lec10.tex
% ================================================================================
% Lecture 10 — Slide 18
% ================================================================================
\begin{frame}[t,fragile]

\mytitle{LangChain: Motivation and Architecture}

\begin{columns}[T,totalwidth=\textwidth]

% --- Left column ---------------------------------------------------------------
\begin{column}[T]{0.48\textwidth}

\vspace{2mm}
\textbf{What LangChain targets}

\begin{itemize}
  \item Reusable prompt templates
  \item Standardized tool access
  \item Simple memory abstractions
\end{itemize}

\vspace{0mm}
LangChain focuses on:
\begin{itemize}
  \item \y{composition}
  \item \y{integration}
  \item rapid prototyping
\end{itemize}

\end{column}

% --- Right column --------------------------------------------------------------
\begin{column}[T]{0.55\textwidth}

\vspace{-3mm}
\textbf{Core building blocks}

\begin{itemize}
  \item LLM interface
  \item PromptTemplate
  \item Chain
  \item Tool
  \item Memory
\end{itemize}

\vspace{2mm}
\textbf{Limitation}

\begin{itemize}
  \item Linear execution model
  \item Limited explicit control flow
\end{itemize}

\vspace{1mm}
\textbf{Key point:}  
LangChain \y{glues components}, it does not control logic.

\end{column}

\end{columns}

\end{frame}
