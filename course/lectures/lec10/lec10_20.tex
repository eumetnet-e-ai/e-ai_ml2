%!TEX root = lec10.tex
% ================================================================================
% Lecture 10 — Slide 20
% ================================================================================
\begin{frame}[t,fragile]

\mytitle{Autonomous Coding Agent — Self-Correcting Loop}

\begin{columns}[T,totalwidth=\textwidth]

% --- Left column ---------------------------------------------------------------
\begin{column}[T]{0.38\textwidth}

\vspace{-2mm}
\textbf{Agent task}

\begin{itemize}
  \item Natural-language goal
  \item LLM generates full script
  \item Script is executed
  \item Errors are fed back automatically
\end{itemize}

\vspace{1mm}
This is already a \y{true agent}:
\begin{itemize}
  \item autonomous retries
  \item internal state (prompt history)
  \item stop criterion
\end{itemize}

\end{column}

% --- Right column --------------------------------------------------------------
\begin{column}[T]{0.6\textwidth}

\vspace{-3mm}
\begin{codeonly}{Autonomous code agent (excerpt)}
def autonomous_code_agent(task):
  for attempt in range(5):
    code = llm.invoke(task).content
    try:
      exec(code)
      return True
    except Exception as e:
      task += traceback.format_exc()
  return False
\end{codeonly}

\vspace{1mm}
\textbf{Key point:}  
No framework required —  
this is \y{pure control logic} around an LLM.

\end{column}

\end{columns}

\end{frame}
