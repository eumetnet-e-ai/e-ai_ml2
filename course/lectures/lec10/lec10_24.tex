%!TEX root = lec10.tex
% ================================================================================
% Lecture 10 — Slide 24
% ================================================================================
\begin{frame}[t,fragile]
\begin{tightmath}

\mytitle{Control Flow in LangGraph}

\begin{columns}[T,totalwidth=\textwidth]

% ------------------------------------------------------------
\begin{column}[T]{0.4\textwidth}

\vspace{-2mm}
\begin{itemize}
  \item No hidden agent loops
  \item No implicit retries
  \item No LLM-driven control decisions
\end{itemize}

\vspace{0mm}
Instead:
\begin{itemize}
  \item Execution follows a \y{directed graph}
  \item Nodes are pure Python functions
  \item Edges define allowed transitions
\end{itemize}

\end{column}

% ------------------------------------------------------------
\begin{column}[T]{0.6\textwidth}

\textbf{Example: Execution graph}

\vspace{-14mm}
\begin{codeonly}{LangGraph control flow}
builder.set_entry_point("extract_forecast_datetime")
builder.add_edge(
  "extract_forecast_datetime",
  "get_latest_forecast_reference_time")
builder.add_edge(
  "get_latest_forecast_reference_time",
  "calculate_lead_time" )
[...]
builder.set_finish_point("plot_temperature")
\end{codeonly}

\vspace{0mm}
The graph itself defines \y{what happens next}.  
There is no hidden agent controller.

\end{column}

\end{columns}

\end{tightmath}
\end{frame}
