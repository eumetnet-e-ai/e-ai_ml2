%!TEX root = lec10.tex
% ================================================================================
% Lecture 10 — Slide 27
% ================================================================================
\begin{frame}[t,fragile]
\begin{tightmath}

\mytitle{Failure Handling and Robustness}

\begin{columns}[T,totalwidth=\textwidth]

% ------------------------------------------------------------
\begin{column}[T]{0.36\textwidth}

\vspace{-2mm}
\begin{itemize}
  \item LLM extraction errors
  \item Missing or delayed data
  \item Tool execution failures
  \item Invalid intermediate state
\end{itemize}

\vspace{0mm}
\footnotesize
{\bf LangGraph strategy} \\
Each node handles \y{local failure} \\
State records partial results \\
Graph execution remains controlled

\normalsize
\vspace{2mm}
\textbf{Key principle}
\begin{itemize}
  \item \rtext{Failures are \y{data}, not crashes}
\end{itemize}

\end{column}

% ------------------------------------------------------------
\begin{column}[T]{0.68\textwidth}

\textbf{Example: guarded node}

\vspace{-6mm}
\begin{codeonly}{Failure-aware node}
def plot_temperature_node(state: MyState) -> MyState:
  if state.get("temperature_data") is None:
    state["output"] = "No data available"
    return state

  plot_t2m_EU(state["temperature_data"], save_plot=True)
  state["output"] = "Plot created"
  return state
\end{codeonly}

\vspace{2mm}
Failures do not break the agent.  
They update the state and allow the graph to terminate safely.

\end{column}

\end{columns}

\end{tightmath}
\end{frame}
