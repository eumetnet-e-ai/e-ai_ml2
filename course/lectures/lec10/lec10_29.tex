%!TEX root = lec10.tex
% ================================================================================
% Lecture 10 — Slide 29
% ================================================================================
\begin{frame}[t,fragile]
\begin{tightmath}

\mytitle{When Multi-Agent Systems Make Sense}

\begin{columns}[T,totalwidth=\textwidth]

% ------------------------------------------------------------
\begin{column}[T]{0.54\textwidth}

\vspace{-2mm}
\textbf{One agent is sufficient when}
\begin{itemize}
  \item tasks are sequential
  \item state is compact
  \item logic is well-defined
  \item tools dominate execution
\end{itemize}

\vspace{1mm}
\textbf{Multiple agents are useful when}
\begin{itemize}
  \item \y{responsibilities} are clearly separable
  \item different reasoning styles are needed
  \item tasks can proceed independently
  \item \y{software components} can be done independently
\end{itemize}


\end{column}

% ------------------------------------------------------------
\begin{column}[T]{0.5\textwidth}

\vspace{-5mm}
\textbf{Key warning}
\begin{itemize}
  \item multi-agent systems are \y{not} better by default
\end{itemize}

\textbf{Typical multi-agent roles}

\vspace{0mm}
\begin{itemize}
  \item planner / coordinator
  \item \rtext{\bf domain expert}
  \item \rtext{\y{\bf tool executor}}
  \item verifier or critic
\end{itemize}

\vspace{1mm}
\textbf{Design principle}

\vspace{1mm}
\begin{quote}
Add agents only when  
you can explain their responsibility.
\end{quote}

\end{column}

\end{columns}

\end{tightmath}
\end{frame}
