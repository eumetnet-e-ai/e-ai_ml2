%!TEX root = lec10.tex
% ================================================================================
% Lecture 10 — Slide 13
% ================================================================================
\begin{frame}[t,fragile]

\mytitle{Error Handling and Feedback Loops}

\begin{columns}[T,totalwidth=\textwidth]

% --- Left column ---------------------------------------------------------------
\begin{column}[T]{0.48\textwidth}

\vspace{-2mm}
\textbf{Why errors are central}

\begin{itemize}
  \item LLM-generated code is often incomplete
  \item Small syntax or logic errors are common
  \item First attempt rarely works
\end{itemize}

\vspace{0mm}
Errors provide \y{structured feedback}:
\begin{itemize}
  \item missing imports
  \item wrong assumptions
  \item invalid API usage
\end{itemize}

\end{column}

% --- Right column --------------------------------------------------------------
\begin{column}[T]{0.55\textwidth}

\vspace{-1cm}
\textbf{Typical feedback loop}

\vspace{1mm}
\begin{codeonly}{Error capture and retry}
try:
    exec(code)
    success = True
except Exception as e:
    error = traceback.format_exc()
    success = False
\end{codeonly}

\vspace{2mm}
\textbf{Key design choices}

\begin{itemize}
  \item feed back full traceback
  \item limit number of retries
  \item detect repeating failures
\end{itemize}

\vspace{1mm}
\textbf{Key point:}  
Errors drive \y{self-correction}.

\end{column}

\end{columns}

\end{frame}
