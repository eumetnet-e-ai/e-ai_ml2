%!TEX root = lec10.tex
% ================================================================================
% Lecture 10 — Slide 14
% ================================================================================
\begin{frame}[t,fragile]

\mytitle{The First Coding Agent: \\ \hspace*{1cm}\y{Self-Correcting Loops}}

\begin{columns}[T,totalwidth=\textwidth]

% --- Left column ---------------------------------------------------------------
\begin{column}[T]{0.4\textwidth}

\vspace{-2mm}
\textbf{Core idea}

\begin{itemize}
  \item LLM generates code
  \item Code is executed
  \item Errors are captured
  \item LLM is prompted to fix them
\end{itemize}

\vspace{0mm}
This creates an \y{autonomous correction loop}.

\vspace{1mm}
\textbf{Minimal success criteria}

\begin{itemize}
  \item Code executes without error
  \item Output matches basic expectations
\end{itemize}

\end{column}

% --- Right column --------------------------------------------------------------
\begin{column}[T]{0.59\textwidth}

\vspace{-15mm}
\textbf{Minimal agent loop}

\vspace{1mm}
\begin{codeonly}{Self-correcting loop}
for attempt in range(max_tries):
  code = llm(prompt)
  try:
    exec(code)
    break
  except Exception as e:
    prompt += traceback.format_exc()
\end{codeonly}

\vspace{1mm}
\rtext{\bf Why this is already an agent}

\begin{itemize}
  \item autonomous retries
  \item internal state (prompt history)
  \item decision: retry vs stop
\end{itemize}

\vspace{1mm}
No framework required.

\end{column}

\end{columns}

\end{frame}
