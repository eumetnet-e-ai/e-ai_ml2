%!TEX root = lec10.tex
% ================================================================================
% Lecture 10 — Slide 30
% ================================================================================
\begin{frame}[t,fragile]
\begin{tightmath}

\mytitle{CrewAI: Role-Based Agent Collaboration}

\begin{columns}[T,totalwidth=\textwidth]

% ------------------------------------------------------------
\begin{column}[T]{0.54\textwidth}

\vspace{-2mm}
\textbf{What CrewAI provides}
\begin{itemize}
  \item explicit agent roles
  \item task delegation
  \item simple coordination logic
  \item readable high-level structure
\end{itemize}

\vspace{0mm}
\textbf{Typical use cases}
\begin{itemize}
  \item document analysis
  \item research workflows
  \item report generation
  \item exploratory automation
\end{itemize}



\end{column}

% ------------------------------------------------------------
\begin{column}[T]{0.5\textwidth}

\vspace{0mm}
\textbf{Strength}
\begin{itemize}
  \item fast prototyping of multi-agent ideas
\end{itemize}

\vspace{1mm}
\textbf{Limitations}
\begin{itemize}
  \item implicit memory handling
  \item limited state visibility
  \item weak failure control
  \item hard to test systematically
\end{itemize}

\vspace{1mm}
\textbf{Key takeaway}

\vspace{1mm}
\begin{quote}
\raggedright
CrewAI is useful for \y{coordination demos},  
not for operating critical systems.
\end{quote}

\end{column}

\end{columns}

\end{tightmath}
\end{frame}
