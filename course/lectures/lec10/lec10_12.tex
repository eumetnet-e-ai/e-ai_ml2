%!TEX root = lec10.tex
% ================================================================================
% Lecture 10 — Slide 12
% ================================================================================
\begin{frame}[t,fragile]

\mytitle{Executing Generated Code Safely}

\begin{columns}[T,totalwidth=\textwidth]

% --- Left column ---------------------------------------------------------------
\begin{column}[T]{0.48\textwidth}

\vspace{-2mm}
\textbf{Execution options}

\begin{itemize}
  \item \texttt{exec} inside current process
  \item Separate process via \texttt{subprocess}
  \item File-based execution
\end{itemize}

\vspace{0mm}
\textbf{Trade-offs}

\begin{itemize}
  \item \texttt{exec}: fast, \y{unsafe}
  \item \texttt{subprocess}: isolated, slower
  \item Files: traceable, debuggable
\end{itemize}

\vspace{0mm}
\textbf{Rule of thumb}

Never execute LLM code  
\y{without isolation or inspection}.

\end{column}

% --- Right column --------------------------------------------------------------
\begin{column}[T]{0.55\textwidth}

\vspace{-1cm}
\textbf{Minimal example}

\vspace{1mm}
\begin{codeonly}{File-based execution}
with open("gen.py","w") as f:
    f.write(code)

import subprocess
subprocess.run(
  ["python", "gen.py"],
  check=True)
\end{codeonly}

\vspace{2mm}
\textbf{Security risks}

\begin{itemize}
  \item File system access
  \item Network calls
  \item Infinite loops
\end{itemize}

\vspace{1mm}
Mitigation requires \y{process isolation}.

\end{column}

\end{columns}

\end{frame}
