%!TEX root = lec10.tex
% ================================================================================
% Lecture 10 — Slide 19
% ================================================================================
\begin{frame}[t,fragile]

\mytitle{Prompt Templates and Tool Integration (LangChain)}

\begin{columns}[T,totalwidth=\textwidth]

% --- Left column ---------------------------------------------------------------
\begin{column}[T]{0.47\textwidth}

\vspace{-2mm}
\begin{itemize}
  \item Separates instructions and variables
  \item Enforces a fixed structure
  \item Improves reuse and testing
\end{itemize}

\vspace{1mm}
\begin{codeonly}{PromptTemplate example}
from langchain.prompts import PromptTemplate

prompt = PromptTemplate(
  input_variables=["x"],
  template=
  "Write Python code computing f(x) "
  "such that |f(x)| < 10." )
\end{codeonly}

\end{column}

% --- Right column --------------------------------------------------------------
\begin{column}[T]{0.52\textwidth}

\vspace{-1mm}
\textbf{Tool integration}

\vspace{1mm}
\begin{codeonly}{Tool definition}
from langchain.tools import tool

@tool
def square(x: float) -> float:
    return x * x
\end{codeonly}

\vspace{1mm}
\textbf{Critical limitations}

\begin{itemize}
  \item LLM may select wrong tool
  \item Arguments may be malformed
\end{itemize}

\vspace{1mm}
\textbf{Rule:}  
Tool calls must be \y{validated outside} the LLM.

\end{column}

\end{columns}

\end{frame}
