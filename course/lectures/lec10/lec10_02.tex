%!TEX root = lec10.tex
% ================================================================================
% Lecture 10 — Slide 02
% ================================================================================
\begin{frame}[t,fragile]
\begin{tightmath}

\mytitle{Tool Contracts and Schemas}

\begin{columns}[T,totalwidth=\textwidth]

% ------------------------------------------------------------
\begin{column}[T]{0.52\textwidth}

\vspace{-2mm}
A tool is defined by:
\begin{itemize}
  \item name
  \item description
  \item input schema
\end{itemize}

\vspace{1mm}
Schemas specify:
\begin{itemize}
  \item required arguments
  \item data types
  \item allowed structure
\end{itemize}

\vspace{1mm}
\textbf{Important:}
\begin{itemize}
  \item tools are defined by the \y{system}
  \item never invented by the model
\end{itemize}

\end{column}

% ------------------------------------------------------------
\begin{column}[T]{0.48\textwidth}

\textbf{Example (conceptual)}

\vspace{-10mm}
\begin{codeonly}{Tool schema}
name: get_temperature
arguments:
  location: string
  leadtime: integer
\end{codeonly}

\vspace{1mm}
The model learns:
\begin{itemize}
  \item when this tool applies
  \item how to fill arguments
\end{itemize}

\footnotesize\color{red}
\begin{lstlisting}
{
  "tool_call": {
    "name": "<function name>",
    "arguments": { ... }
  }
}
\end{lstlisting}

\end{column}

\end{columns}

\end{tightmath}
\end{frame}
