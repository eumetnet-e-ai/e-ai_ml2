%!TEX root = lec10.tex
% ================================================================================
% Lecture 10 — Slide 25
% ================================================================================
\begin{frame}[t,fragile]
\begin{tightmath}

\mytitle{LLMs as Nodes in LangGraph}

\begin{columns}[T,totalwidth=\textwidth]

% ------------------------------------------------------------
\begin{column}[T]{0.42\textwidth}

\vspace{-2mm}
\begin{itemize}
  \item LLMs do \emph{not} control execution
  \item LLMs do \emph{not} manage memory
  \item LLMs do \emph{not} decide termination
\end{itemize}

\vspace{0mm}
Instead:
\begin{itemize}
  \item LLMs are used for \y{local reasoning tasks}
  \item Each call has a well-defined input
  \item Each call produces a bounded output
\end{itemize}


\end{column}

% ------------------------------------------------------------
\begin{column}[T]{0.59\textwidth}

\vspace{-8mm}
\hspace{1cm}\begin{minipage}{5cm}
\textbf{Typical LLM roles:}
\begin{itemize}
  \item information extraction
  \item classification
  \item summarization
\end{itemize}
\end{minipage}

\vspace{2mm}
\begin{codeonly}{LangGraph node}
def extr_loc_node(state: MyState) -> MyState:
  location = extr_loc(state["query"])
  state["fc_location"] = location
  return state
\end{codeonly}

\vspace{0mm}
The LLM acts as a \y{pure transformation}  
from input fields to output fields.

\end{column}

\end{columns}

\end{tightmath}
\end{frame}
