%!TEX root = lec03.tex
% ================================================================================
% Lecture 3 — Slide 04
% ================================================================================
\begin{frame}[t,fragile]

\mytitle{IFS Open Data — Access via Client Library}

\begin{columns}[T,totalwidth=\textwidth]

% --- Left column ---------------------------------------------------------------
\begin{column}[T]{0.42\textwidth}

\vspace{-2mm}
\begin{itemize}
  \item IFS data is not exposed as simple directories
  \item Access is service-based and load-controlled
  \item \y{Small subsets} can be retrieved efficiently
\end{itemize}

\vspace{1mm}
\textbf{Key idea}

\begin{itemize}
  \item Use ECMWF’s official open-data client
  \item Request fields by \y{parameter, step, level}
  \item Result is a standard GRIB2 file
\end{itemize}
\end{column}

% --- Right column --------------------------------------------------------------
\begin{column}[T]{0.56\textwidth}

\vspace{-0.3cm}
\begin{codeonly}{Downloading IFS fields with ecmwf-opendata}
from ecmwf.opendata import Client

client = Client(
    source="ecmwf",
    model="ifs",
)

client.retrieve(
    time=0,
    type="fc",
    step=24,
    param=["2t", "msl"],
    target="ifs_2t.grib2")
\end{codeonly}

\end{column}

\end{columns}

\end{frame}
