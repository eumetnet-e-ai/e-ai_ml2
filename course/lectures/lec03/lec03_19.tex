%!TEX root = lec03.tex
% ================================================================================
% Lecture 3 — Slide 19
% ================================================================================
\begin{frame}[t]

\mytitle{GPU Access in Practice — Why It Depends on the Platform}

\begin{columns}[T,totalwidth=\textwidth]

% --- Left column ---------------------------------------------------------------
\begin{column}{0.53\textwidth}

\vspace{-2mm}
\begin{itemize}
  \item Most laptops \emph{do have a GPU}
  \item But GPU \y{type and backend} differ strongly
  \item Python code must match the backend
\end{itemize}

\vspace{1mm}
\textbf{Common GPU types}

\begin{itemize}
  \item \y{NVIDIA} — typical for HPC clusters
  \item \y{Apple GPU} — Apple Silicon 
  \item \y{AMD / Radeon} — some laptops, workstations
\end{itemize}

\vspace{1mm}
\textbf{Key message}

\begin{itemize}
  \item The \y{software interface} matters
\end{itemize}

\end{column}

% --- Right column --------------------------------------------------------------
\begin{column}{0.48\textwidth}

\vspace{-1mm}
\textbf{GPU backends used in Python}

\vspace{1mm}
\begin{itemize}
  \item \texttt{CUDA} — NVIDIA GPUs (Linux, HPC)
  \item \texttt{MPS / Metal} — Apple GPUs (macOS)
  \item \texttt{CPU fallback} — always available
\end{itemize}

\vspace{2mm}
\textbf{Implication for ML workflows}

\begin{itemize}
  \item Development often on laptops
  \item Training often on HPC systems
  \item Code must be \y{portable across backends}
\end{itemize}

\end{column}

\end{columns}

\end{frame}
