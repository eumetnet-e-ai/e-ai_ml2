%!TEX root = lec03.tex
% ================================================================================
% Lecture 3 — Slide 14
% ================================================================================
\begin{frame}[t,fragile]

\mytitle{Feedback Files — Structure and Indexing}

\begin{columns}[T,totalwidth=\textwidth]

% --- Left column ---------------------------------------------------------------
\begin{column}{0.35\textwidth}

\vspace{-2mm}
\begin{itemize}
  \item Stored as structured \y{NetCDF} files
  \item Separate layers for:
    \begin{itemize}
      \item report metadata
      \item individual observations
    \end{itemize}
\end{itemize}

\vspace{1mm}
\textbf{Key idea}

\begin{itemize}
  \item One report may contain many observations
  \item Explicit indexing links both layers
\end{itemize}

\end{column}

% --- Right column --------------------------------------------------------------
\begin{column}{0.63\textwidth}

\vspace{-1mm}
\textbf{Core variables (excerpt)}

\begin{verbatim}
i_body    (nreport)     start index of report
l_body    (nreport)     number in report

obs       (nobs)        observation
veri_data (nmodel,nobs) model equivalent H(x)
\end{verbatim}

\vspace{1mm}
\textbf{Interpretation}

\begin{itemize}
  \item \texttt{i\_body:l\_body} maps reports → obs
  \item \texttt{obs} and \texttt{veri\_data} are aligned
  \item Enables direct O–B, O–A diagnostics
\end{itemize}

\end{column}

\end{columns}

\end{frame}
