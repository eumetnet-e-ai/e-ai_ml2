%!TEX root = lec12.tex
% ================================================================================
% Chapter 12 — Slide 08
% ================================================================================
\begin{frame}[t,fragile]
\begin{tightmath}

\mytitle{Tracking URI}

\begin{columns}[T,totalwidth=\textwidth]

% ------------------------------------------------------------
\begin{column}[T]{0.48\textwidth}

\vspace{-2mm}
\footnotesize
\textbf{Definition}

\begin{itemize}
  \item URI = \texttt{Uniform Resource Identifier}
  \item \y{address} of the MLflow tracking backend
  \item Decides \y{where} runs are written
\end{itemize}

\begin{itemize}
  \item Destination of
    \begin{itemize}
      \item experiments
      \item runs
      \item metrics
      \item artifacts
    \end{itemize}
\end{itemize}

\vspace{2mm}
\textbf{Keep untouched:}

\begin{itemize}
  \item Training code
  \item Logging calls
\end{itemize}

\end{column}

% ------------------------------------------------------------
\begin{column}[T]{0.52\textwidth}

\vspace{-6mm}
\footnotesize
\textbf{How it is set in code}

\begin{codeonly}{}
import mlflow

mlflow.set_tracking_uri(
    "http://localhost:5000")
\end{codeonly}

\vspace{2mm}
\textbf{Typical values}

\begin{itemize}
  \item \texttt{file:./mlruns} for local storage only
  \item \texttt{http://localhost:5000} for using the mlflow server
\end{itemize}

\vspace{2mm}
\tiny
The MLflow tracking behavior is fully determined by the tracking URI.
If the URI is set to \texttt{sqlite:///mlflow.db}, the Python process writes
experiment metadata directly into a local SQLite database, without using
any HTTP communication or server process.
In contrast, if the URI is set to \texttt{http://localhost:5000}, all tracking
data is sent via HTTP to a running MLflow server, which then stores the
results using its configured backend store (e.g.\ SQLite).


\end{column}

\end{columns}

\end{tightmath}
\end{frame}
