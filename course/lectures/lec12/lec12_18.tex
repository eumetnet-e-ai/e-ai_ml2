%!TEX root = lec12.tex
% ================================================================================
% Lecture 12 — Slide 18
% ================================================================================
\begin{frame}[t,fragile]

\mytitle{Background and \y{Persistent Execution}}

\begin{columns}[T,totalwidth=\textwidth]

% --- Left column ---------------------------------------------------------------
\begin{column}[T]{0.48\textwidth}

\textbf{Why this matters}

\begin{itemize}
  \item Long-running experiments
  \item Remote machines
  \item Logout-safe operation
\end{itemize}

\vspace{2mm}
\rtext{\bf Servers must survive terminals.}

\vspace{4mm}
\footnotesize
These mechanisms are \y{generic system tools} for running long-lived services;
MLflow behaves like any other server process.

\end{column}

% --- Right column --------------------------------------------------------------
\begin{column}[T]{0.48\textwidth}

\vspace{-8mm}
\begin{codeonly}{Common patterns}
mlflow server ... &

nohup mlflow server ... &

screen -S mlflow
\end{codeonly}

\vspace{2mm}
\footnotesize
Use \texttt{screen} or \texttt{tmux} for
interactive re-attachment and monitoring.

\vspace{4mm}
\tiny\color{darkgreen}
\begin{lstlisting}
tmux new -s mlflow          # start a persistent session
mlflow server --host 0.0.0.0 --port 5000
Ctrl+B  D                  # detach from session
tmux attach -t mlflow      # reconnect later
\end{lstlisting}

\end{column}

\end{columns}

\end{frame}
