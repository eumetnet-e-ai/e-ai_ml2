%!TEX root = lec12.tex
% ================================================================================
% Lecture 12 — Slide 17
% ================================================================================
\begin{frame}[t,fragile]

\mytitle{\y{Server Mode} for Collaboration}

\begin{columns}[T,totalwidth=\textwidth]

% --- Left column ---------------------------------------------------------------
\begin{column}[T]{0.48\textwidth}

\textbf{Why server mode?}

\begin{itemize}
  \item \y{Multiple users}
  \item Multiple machines
  \item Shared experiment history
\end{itemize}

\vspace{2mm}
\textbf{Command}

\begin{itemize}
  \item \texttt{mlflow server}
\end{itemize}

\vspace{2mm}
\textbf{Important point}

\begin{itemize}
  \item Training code stays \y{unchanged}
  \item Only the tracking URI changes
\end{itemize}

\end{column}

% --- Right column --------------------------------------------------------------
\begin{column}[T]{0.48\textwidth}

\vspace{-2mm}
\begin{codeonly}{Server startup}
mlflow server \\
  --host 0.0.0.0 \\
  --port 5000
\end{codeonly}

\vspace{2mm}
\footnotesize
Clients connect via \texttt{http://<server>:5000}
and log runs remotely.

\vspace{4mm}
\rtext{
{\bf Server Setup}

An MLflow server can run on any host reachable by domain name or IP address; clients only need the tracking URI to connect.
}

\end{column}

\end{columns}

\end{frame}
