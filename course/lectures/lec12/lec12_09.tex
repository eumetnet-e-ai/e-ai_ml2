%!TEX root = lec12.tex
% ================================================================================
% Chapter 12 — Slide 09
% ================================================================================
\begin{frame}[t,fragile]
\begin{tightmath}

\mytitle{Minimal Logging Example}

\begin{columns}[T,totalwidth=\textwidth]

% ------------------------------------------------------------
\begin{column}[T]{0.46\textwidth}

\vspace{-2mm}
\footnotesize
\textbf{Essential steps}

\begin{itemize}
  \item Select an experiment
  \item Start a run
  \item Log parameters
  \item Log metrics
\end{itemize}

\vspace{2mm}
\textbf{Key idea}

\begin{itemize}
  \item One run = one execution
  \item Everything else is optional
\end{itemize}

\vspace{2mm}
\begin{codeonly}{Upload Artefact}
mlflow.log_artifact("a.png")
\end{codeonly}

\end{column}

% ------------------------------------------------------------
\begin{column}[T]{0.56\textwidth}

\vspace{-8mm}
\footnotesize
\begin{codeonly}{Minimal MLflow code}
import mlflow

mlflow.set_experiment("Demo")

with mlflow.start_run():
    mlflow.log_param("lr", 1e-3)
    mlflow.log_metric("loss", 0.42)
\end{codeonly}

\vspace{1mm}
\hspace{5mm}
\begin{minipage}{6cm}
\footnotesize
\textbf{What this code does}

\begin{itemize}
  \item Creates or selects the experiment \texttt{Demo}
  \item Opens a new run with a unique ID
  \item Stores one parameter and one metric
  \item Makes the run visible in the MLflow UI
\end{itemize}
\end{minipage}

\end{column}

\end{columns}

\end{tightmath}
\end{frame}
