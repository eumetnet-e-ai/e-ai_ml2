%!TEX root = lec05.tex
% ================================================================================
% Lecture 5 — Slide 12
% ================================================================================
\begin{frame}[t,fragile]

\mytitle{GNN Application Example — Advection on a Periodic Grid}

\begin{columns}[T,totalwidth=\textwidth]

% --- Left column ---------------------------------------------------------------
\begin{column}[T]{0.48\textwidth}

\vspace{-2mm}
\textbf{Physical setup}

\begin{itemize}
  \item Nodes arranged on a periodic ring
  \item Each node holds a scalar field value
  \item Goal: learn one-step time evolution
\end{itemize}

\vspace{1mm}
\textbf{Learning task}

\begin{itemize}
  \item Input: field at time $t$
  \item Output: field at time $t+\Delta t$
  \item Advection-like transport process
\end{itemize}

\vspace{1mm}
Mathematically:
\[
z^{t+1} \;\approx\; f_\theta(G, z^t)
\]

\end{column}

% --- Right column --------------------------------------------------------------
\begin{column}[T]{0.5\textwidth}

\vspace{-4mm}
\includegraphics[width=0.9\textwidth]{../../images/img05/gnn_test_tds_1.png}

\vspace{1mm}
\centering
Graph structure with periodic connectivity

\vspace{1mm}
\raggedright
\textbf{Why a GNN?}

\begin{itemize}
  \item Local interactions dominate dynamics
  \item Translation invariance on the ring
  \item Same update rule for all nodes
\end{itemize}

\end{column}

\end{columns}

\end{frame}
