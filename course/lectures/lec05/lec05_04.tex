%!TEX root = lec05.tex
% ================================================================================
% Lecture 5 — Slide 04
% ================================================================================
\begin{frame}[t,fragile]

\mytitle{Model Capacity and Trainable Parameters}

\begin{columns}[T,totalwidth=\textwidth]

% --- Left column ---------------------------------------------------------------
\begin{column}[T]{0.4\textwidth}

\textbf{What defines model capacity?}

\begin{itemize}
  \item Number of parameters
  \item Network depth
  \item Width of layers
\end{itemize}

\vspace{1mm}
Higher capacity:
\begin{itemize}
  \item Fits more complex functions
  \item Risk of overfitting
\end{itemize}

\end{column}

% --- Right column --------------------------------------------------------------
\begin{column}[T]{0.56\textwidth}

\textbf{Parameter counting}

\vspace{-2mm}
\begin{eqnarray*}
\text{\y{Each Layer}}(n_{\text{in}}\!\rightarrow\!n_{\text{out}}): &&
n_{\text{in}}\cdot n_{\text{out}} + n_{\text{out}} \\[1mm]
\text{Total parameters} &=& 
\sum_{\ell} \left(
n_{\ell-1} n_{\ell} + n_{\ell}
\right)
\end{eqnarray*}

\vspace{1mm}
Example (1–16–16–1):
\[
(1\!\cdot\!16+16) + (16\!\cdot\!16+16) + (16\!\cdot\!1+1) = 337
\]

\end{column}

\end{columns}

\end{frame}
