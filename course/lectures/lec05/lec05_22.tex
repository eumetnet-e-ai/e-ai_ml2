%!TEX root = lec05.tex
% ================================================================================
% Lecture 5 — Slide 21b
% ================================================================================
\begin{frame}[t,fragile]

\mytitle{From LSTM States to Reconstruction Error}

\begin{columns}[T,totalwidth=\textwidth]

% --- Left column ---------------------------------------------------------------
\begin{column}[T]{0.42\textwidth}

\textbf{What goes in}

\begin{itemize}
  \item Input sequence $x_{1:T}$
  \item One value per time step
\end{itemize}

\vspace{1mm}
\textbf{What the LSTM does}

\begin{itemize}
  \item Updates $(h_t, c_t)$ sequentially
  \item Encodes temporal structure
\end{itemize}

\vspace{1mm}
\textbf{What comes out}

\begin{itemize}
  \item Hidden states $h_t$
  \item Latent temporal representation
\end{itemize}

\end{column}

% --- Right column --------------------------------------------------------------
\begin{column}[T]{0.56\textwidth}

\vspace{-2mm}
\textbf{Reconstruction step}

\[
\hat{x}_t = W_y h_t + b_y
\]

\vspace{1mm}
\textbf{Training objective}

\[
\mathcal{L}
= \frac{1}{T} \sum_{t=1}^T
\left\| x_t - \hat{x}_t \right\|^2
\]

\vspace{1mm}
\textbf{Anomaly detection}

\begin{itemize}
  \item Low error: normal sequence
  \item High error: anomalous sequence
\end{itemize}

\vspace{1mm}
\raggedright
$W_y, b_y$ learned jointly with LSTM weights

\end{column}

\end{columns}

\end{frame}
