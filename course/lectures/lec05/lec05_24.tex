%!TEX root = lec05.tex
% ================================================================================
% Lecture 5 — Slide 23
% ================================================================================
\begin{frame}[t,fragile]

\mytitle{Coding the LSTM Autoencoder}

\begin{columns}[T,totalwidth=\textwidth]

% --- Left column ---------------------------------------------------------------
\begin{column}[T]{0.25\textwidth}

\textbf{Model structure}

\begin{itemize}
  \item Encoder LSTM
  \item Decoder LSTM
  \item Linear reconstruction
\end{itemize}

\vspace{0mm}
\textbf{Data flow:}
\begin{itemize}
  \item Sequence $\rightarrow$ hidden state
  \item Hidden state $\rightarrow$ sequence
\end{itemize}

\vspace{-1mm}
{\tiny\begin{lstlisting}
Input:  (B, T, 1)
Hidden: (L, B, H)
Output: (B, T, 1)
\end{lstlisting}}

\end{column}

% --- Right column --------------------------------------------------------------
\begin{column}[T]{0.75\textwidth}

\vspace{-4mm}
\begin{codeonly}{LSTM autoencoder (PyTorch)}
class LSTMAutoencoder(nn.Module):
  def __init__(self, hidden_dim=32, layers=2):
    super().__init__()
    self.encoder = nn.LSTM(
      1, hidden_dim, layers, batch_first=True)
    self.decoder = nn.LSTM(
      1, hidden_dim, layers, batch_first=True)
    self.out = nn.Linear(hidden_dim, 1)

  def forward(self, x):
    _, (h, c) = self.encoder(x)
    z = torch.zeros_like(x)
    y, _ = self.decoder(z, (h, c))
    return self.out(y)
\end{codeonly}

\end{column}

\end{columns}

\end{frame}
