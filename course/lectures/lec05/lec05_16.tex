%!TEX root = lec05.tex
% ================================================================================
% Lecture 5 — Slide 17
% ================================================================================
\begin{frame}[t,fragile]

\mytitle{CNN Structure for Function Classification}

\begin{columns}[T,totalwidth=\textwidth]

% --- Left column ---------------------------------------------------------------
\begin{column}[T]{0.3\textwidth}

\textbf{Network structure}

\begin{itemize}
  \item 1D convolutions for local patterns
  \item Nonlinear feature extraction
  \item Fully connected classifier
\end{itemize}

{\footnotesize\begin{lstlisting}
Input:   (B, 1, 50)
Conv1:   (B, 16, 50)
Conv2:   (B, 32, 50)
Flatten: (B, 1600)
FC:      (B, 4)
\end{lstlisting}}

\end{column}

% --- Right column --------------------------------------------------------------
\begin{column}[T]{0.75\textwidth}

\vspace{-4mm}
\begin{codeonly}{CNN model definition}
class FunctionClassifierCNN(nn.Module):
  def __init__(self):
     super().__init__()
     self.conv1 = nn.Conv1d(1, 16,5, padding=2)
     self.conv2 = nn.Conv1d(16,32,5, padding=2)
     self.fc1   = nn.Linear(32*50, 128)
     self.fc2   = nn.Linear(128, 4)

  def forward(self, x):
     x = torch.relu(self.conv1(x))
     x = torch.relu(self.conv2(x))
     x = x.view(x.size(0), -1)
     x = torch.relu(self.fc1(x))
     return self.fc2(x)
\end{codeonly}

\end{column}

\end{columns}

\end{frame}
