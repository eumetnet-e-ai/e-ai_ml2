%!TEX root = lec05.tex
% ================================================================================
% Lecture 5 — Slide 6
% ================================================================================
\begin{frame}[t,fragile]

\mytitle{FFNN as a \y{Function Approximator}}

\begin{columns}[T,totalwidth=\textwidth]

% --- Left column ---------------------------------------------------------------
\begin{column}[T]{0.42\textwidth}

\vspace{-2mm}
\textbf{Problem setting}

\begin{itemize}
  \item Unknown target function $f(x)$
  \item Discrete training data available
  \item Goal: approximate $f$ using a neural network
\end{itemize}

\vspace{-1mm}
\textbf{Neural-network view}

\begin{itemize}
  \item The network defines a family of functions
  \item Parameters determine the concrete shape
  \item Training = selecting a suitable function
\end{itemize}

\end{column}

% --- Right column --------------------------------------------------------------
\begin{column}[T]{0.56\textwidth}

\vspace{-4mm}
\[
f(x) \;\approx\; f_\theta(x)
\]

\vspace{1mm}
\includegraphics[width=0.48\textwidth]{../../images/img05/deep_nn_function_approximation.png}
\includegraphics[width=0.48\textwidth]{../../images/img05/deep_nn_loss_curve.png}

\vspace{1mm}
\centering
Ground truth (left) vs.\ NN approximation \\
and training history (right)

\vspace{1mm}
\raggedright
\textbf{Intuition}

\begin{itemize}
  \item Each layer further shapes the function
  \item Nonlinearities enable complex forms
  \item A smooth approximation emerges step by step
\end{itemize}

\end{column}

\end{columns}

\end{frame}
