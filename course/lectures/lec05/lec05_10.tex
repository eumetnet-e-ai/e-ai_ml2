%!TEX root = lec05.tex
% ================================================================================
% Lecture 5 — Slide 10
% ================================================================================
\begin{frame}[t,fragile]

\mytitle{Adjacency and \texttt{edge\_index}}

\begin{columns}[T,totalwidth=\textwidth]

% --- Left column ---------------------------------------------------------------
\begin{column}[T]{0.4\textwidth}

\textbf{Adjacency matrix}

\begin{itemize}
  \item Matrix $A \in \{0,1\}^{N\times N}$
  \item $A_{ij}=1$ if nodes $i,j$ are connected
  \item Simple, but memory expensive
\end{itemize}

\vspace{1mm}
\begin{minipage}[b]{2.2cm}
$
\begin{pmatrix}
0 & 1 & 0 & 1\\
1 & 0 & 1 & 0\\
0 & 1 & 0 & 1\\
1 & 0 & 1 & 0
\end{pmatrix}
$
\end{minipage}
\begin{minipage}{3.2cm}
\includegraphics[width=3.2cm]{../../images/img05/diff_matrix_smaller_const.png}
\end{minipage}

\end{column}

% --- Right column --------------------------------------------------------------
\begin{column}[T]{0.58\textwidth}

\vspace{-5mm}
\textbf{\texttt{edge\_index} representation}

\begin{itemize}
  \item Sparse edge list format
  \item Two rows: source and target nodes
  \item Standard in PyTorch Geometric
\end{itemize}

\vspace{1mm}
\[
\texttt{edge\_index} =
\begin{pmatrix}
0 & 1 & 1 & 2 \\
1 & 0 & 2 & 1
\end{pmatrix}
\]

\vspace{1mm}
\raggedright
\textbf{Why this matters}

\begin{itemize}
  \item Scales to large graphs
  \item Efficient message passing
  \item Natural for irregular structures
\end{itemize}

\end{column}

\end{columns}

\end{frame}
