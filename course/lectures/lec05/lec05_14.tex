%!TEX root = lec05.tex
% ================================================================================
% Lecture 5 — Slide 14
% ================================================================================
\begin{frame}[t,fragile]

\mytitle{Why Convolutional Neural Networks?}

\begin{columns}[T,totalwidth=\textwidth]

% --- Left column ---------------------------------------------------------------
\begin{column}[T]{0.4\textwidth}

\textbf{Key idea}

\begin{itemize}
  \item Exploit \y{local structure}
  \item Same operation everywhere
  \item Strong inductive bias
\end{itemize}

\vspace{1mm}
\textbf{Typical data}

\begin{itemize}
  \item Images (2D grids)
  \item Time series (1D signals)
  \item Physical fields on grids
\end{itemize}

\vspace{1mm}
CNNs assume:
\[
\text{locality} + \text{translation invariance}
\]

\end{column}

% --- Right column --------------------------------------------------------------
\begin{column}[T]{0.58\textwidth}

\vspace{-2mm}
\textbf{Convolution instead of full connectivity}

\begin{itemize}
  \item Small kernel slides over input
  \item Shared weights across positions
  \item Far fewer parameters than FFNNs
\end{itemize}

\vspace{-1mm}
Mathematically (1D):
\begin{eqnarray*}
y_i &=& \sum_{k=-K}^{K} w_k \, x_{i+k}
\end{eqnarray*}

\vspace{-2mm}
Interpretation:
\begin{itemize}
  \item Learn local patterns
  \item \y{Detect edges, waves, peaks}
  \item Build hierarchy via depth
\end{itemize}

\end{column}

\end{columns}

\end{frame}
