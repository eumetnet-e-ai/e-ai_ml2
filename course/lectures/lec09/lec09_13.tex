%!TEX root = lec09.tex
% ================================================================================
% Lecture 9 — Slide 13
% ================================================================================
\begin{frame}[t,fragile]
\begin{tightmath}

\mytitle{Why Diffusion Sampling Is Correct}

\begin{columns}[T,totalwidth=\textwidth]

% ------------------------------------------------------------
\begin{column}[T]{0.5\textwidth}

Assume a scalar Gaussian data distribution:
\[
x_0 \sim \mathcal{N}(\mu_0,\sigma_0^2),
\quad x_1 = x_0 + \epsilon,
\]
with $\epsilon \sim \mathcal{N}(0,\sigma^2)$. 

\vspace{1mm}
The optimal denoiser (MSE sense) is:
\[
f^*(x_1) = \mathbb{E}[x_0 \mid x_1]
= \frac{\sigma^2 \mu_0 + \sigma_0^2 x_1}{\sigma_0^2 + \sigma^2}.
\]

This pulls the noisy sample toward $\mu_0$,
but \emph{reduces variance}.

\end{column}

% ------------------------------------------------------------
\begin{column}[T]{0.48\textwidth}

\vspace{-1mm}
\textbf{Key observation}

\begin{itemize}
  \item Denoising alone shrinks variance
  \item Mean is biased toward $\mu_0$
\end{itemize}

\vspace{1mm}
\textbf{Therefore}

To recover the full distribution,
we must sample:
\[
x_0 = f^*(x_1) + \sqrt{\sigma_{\text{post}}^2}\,\xi,
\quad \xi \sim \mathcal{N}(0,1).
\]

\vspace{1mm}
This is exactly what diffusion does
at every step.

\end{column}

\end{columns}

\end{tightmath}
\end{frame}
