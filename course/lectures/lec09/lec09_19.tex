%!TEX root = lec09.tex
% ================================================================================
% Lecture 9 — Slide 19
% ================================================================================
\begin{frame}[t,fragile]

\mytitle{Why PyTorch Lightning?}

\begin{columns}[T,totalwidth=\textwidth]

\begin{column}[T]{0.55\textwidth}

\textbf{Problem with raw PyTorch}

\begin{itemize}
  \item Training loops get long and repetitive
  \item Device handling (CPU/GPU) is manual
  \item Logging and checkpoints are ad hoc
\end{itemize}

\vspace{2mm}
Code quickly becomes hard to read and maintain.

\tiny
\begin{lstlisting}
from torch_geometric.loader import DataLoader

# Create dataset
dataset = [generate_sample_graph(n_nodes=64, dn=8, k=3) for _ in range(200)]
loader = DataLoader(dataset, batch_size=16, shuffle=True)

# Train
model = GNNInterpolator()
trainer = pl.Trainer(max_epochs=200, logger=False, enable_checkpointing=False)
trainer.fit(model, loader)
\end{lstlisting}

\end{column}

\begin{column}[T]{0.4\textwidth}

\textbf{Lightning idea}

\begin{itemize}
  \item Separate \emph{what} from \emph{how}
  \item Model logic vs. training mechanics
  \item Convention over configuration
\end{itemize}

\vspace{2mm}
Lightning is \emph{PyTorch}, not a replacement.

\end{column}

\end{columns}

\end{frame}
