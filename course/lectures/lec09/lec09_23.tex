%!TEX root = lec09.tex
% ================================================================================
% Lecture 9 — Slide 23
% ================================================================================
\begin{frame}[t,fragile]

\mytitle{Graphs in PyTorch Geometric}

\begin{columns}[T,totalwidth=\textwidth]

\begin{column}[T]{0.6\textwidth}

\begin{codeonly}{Basic PyG data object}
from torch_geometric.data import Data

data = Data(
    x = node_features,
    edge_index = edge_index,
    y = targets
)
\end{codeonly}

\end{column}

\begin{column}[T]{0.35\textwidth}

\textbf{Key concept}

\begin{itemize}
  \item Nodes = rows of \texttt{x}
  \item Edges define message flow
  \item Graph size is flexible
\end{itemize}

\vspace{2mm}
No grid assumptions required.

\end{column}

\end{columns}

\end{frame}
