%!TEX root = lec09.tex
% ================================================================================
% Lecture 9 — Slide 11
% ================================================================================
\begin{frame}[t,fragile]
\begin{tightmath}

\mytitle{Closed-Form Forward Diffusion}

\begin{columns}[T,totalwidth=\textwidth]

% ------------------------------------------------------------
\begin{column}[T]{0.62\textwidth}

Because all forward steps are Gaussian, the marginal
distribution of $x_t$ given $x_0$ has a closed form:
\[
q(x_t \mid x_0)
= \mathcal{N}\!\left(
x_t;\, \sqrt{\bar{\alpha}_t}\,x_0,\,
(1-\bar{\alpha}_t) I
\right)
\]

with cumulative noise factor
\[
\bar{\alpha}_t
= \prod_{s=1}^t (1-\beta_s).
\]

\vspace{1mm}
Equivalently, sampling can be written as
\[
x_t
= \sqrt{\bar{\alpha}_t}\,x_0
+ \sqrt{1-\bar{\alpha}_t}\,\epsilon,
\quad \epsilon \sim \mathcal{N}(0,I).
\]

\end{column}

% ------------------------------------------------------------
\begin{column}[T]{0.36\textwidth}

\vspace{-1mm}
\textbf{Interpretation}

\begin{itemize}
  \item Signal amplitude decays with $\sqrt{\bar{\alpha}_t}$
  \item Noise amplitude grows with $\sqrt{1-\bar{\alpha}_t}$
  \item $t$ directly controls noise level
\end{itemize}

\vspace{1mm}
As $t \to T$:
\[
\bar{\alpha}_t \to 0
\quad \Rightarrow \quad
x_t \sim \mathcal{N}(0,I).
\]

\vspace{1mm}
This matches the example exactly.

\end{column}

\end{columns}

\end{tightmath}
\end{frame}
