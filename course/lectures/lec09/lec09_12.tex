%!TEX root = lec09.tex
% ================================================================================
% Lecture 9 — Slide 12
% ================================================================================
\begin{frame}[t,fragile]
\begin{tightmath}

\mytitle{Reverse Diffusion Process}

\begin{columns}[T,totalwidth=\textwidth]

% ------------------------------------------------------------
\begin{column}[T]{0.62\textwidth}

The reverse process aims to invert the forward diffusion:
\[
p_\theta(x_{0:T})
= p(x_T)\prod_{t=1}^T p_\theta(x_{t-1} \mid x_t),
\]
with $p(x_T)=\mathcal{N}(0,I)$. 

\vspace{1mm}
Each reverse step is modeled as a Gaussian:
\[
p_\theta(x_{t-1} \mid x_t)
= \mathcal{N}\!\left(
x_{t-1};\, \mu_\theta(x_t,t),\, \Sigma_t
\right),
\]
where $\mu_\theta$ is predicted by a neural network.

\vspace{1mm}
The variance $\Sigma_t$ is usually fixed or predefined.

\end{column}

% ------------------------------------------------------------
\begin{column}[T]{0.36\textwidth}

\vspace{-1mm}
\textbf{What is learned?}

\begin{itemize}
  \item Only the conditional mean
  \item One network for all $t$
  \item Local denoising steps
\end{itemize}

\vspace{1mm}
\textbf{Key idea}

Each reverse step removes
\y{a small amount of noise}.

\vspace{1mm}
This turns sampling into a sequence
of simple corrections.

\end{column}

\end{columns}

\end{tightmath}
\end{frame}
