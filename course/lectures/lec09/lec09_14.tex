%!TEX root = lec09.tex
% ================================================================================
% Lecture 9 — Slide 14
% ================================================================================
\begin{frame}[t,fragile]

\mytitle{Why Graph Networks?}

\begin{columns}[T,totalwidth=\textwidth]

% ------------------------------------------------------------
\begin{column}[T]{0.55\textwidth}

\textbf{The problem}

\begin{itemize}
  \item \y{Data is sparse, irregular, incomplete}
  \item Classical grids assume full coverage
  \item Interpolation rules are often fixed
\end{itemize}

\vspace{2mm}
In many applications, we only know values at
\begin{itemize}
  \item a few locations,
  \item irregular positions,
  \item changing resolutions.
\end{itemize}

\end{column}

% ------------------------------------------------------------
\begin{column}[T]{0.4\textwidth}

\vspace{-6mm}
\textbf{The graph perspective}

\begin{itemize}
  \item Nodes represent \y{locations}
  \item Edges represent \y{influence} / neighborhood
  \item Features store values and metadata
\end{itemize}

\vspace{2mm}
\textbf{Key idea}

\begin{itemize}
  \item \y{Geometry} is encoded in the graph
  \item \rtext{Learning happens locally}
  \item Same model works on different domains
\end{itemize}

\end{column}

\end{columns}

\end{frame}
