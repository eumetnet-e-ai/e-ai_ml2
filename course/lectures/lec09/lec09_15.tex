%!TEX root = lec09.tex
% ================================================================================
% Lecture 9 — Slide 15
% ================================================================================
\begin{frame}[t,fragile]

\mytitle{Graph Example: Reconstructing a Function from Sparse Data}

\begin{columns}[T,totalwidth=\textwidth]

% ------------------------------------------------------------
\begin{column}[T]{0.55\textwidth}

\textbf{Setup}

\begin{itemize}
  \item 1D spatial domain discretized into nodes
  \item True function $f(x)$ defined on all nodes
  \item Observations available only at few locations
\end{itemize}

\vspace{2mm}
Each node carries:
\begin{itemize}
  \item observed value (or zero if missing)
  \item a binary observation mask
\end{itemize}

\vspace{2mm}
The task is to reconstruct $f(x)$ at \emph{all} nodes.

\end{column}

% ------------------------------------------------------------
\begin{column}[T]{0.4\textwidth}

\includegraphics[width=\textwidth]{../../images/img09/gnn_obs_naive_j1_jj1_0.png}

\vspace{1mm}
\centering
Sparse observations and GNN reconstruction

\end{column}

\end{columns}

\end{frame}
