%!TEX root = lec09.tex
% ================================================================================
% Lecture 9 — Slide 10
% ================================================================================
\begin{frame}[t,fragile]
\begin{tightmath}

\mytitle{Forward Diffusion Process}

\begin{columns}[T,totalwidth=\textwidth]

% ------------------------------------------------------------
\begin{column}[T]{0.62\textwidth}

Let $x_0 \sim p_{\text{data}}(x)$ be a data sample.

\vspace{1mm}
The \y{forward diffusion} process is a Markov chain:
\[
q(x_{1:T} \mid x_0)
= \prod_{t=1}^T q(x_t \mid x_{t-1})
\]

with Gaussian transitions
\[
q(x_t \mid x_{t-1})
= \mathcal{N}\!\left(
x_t;\, \sqrt{1-\beta_t}\,x_{t-1},\, \beta_t I
\right),
\]
with $\beta_t \in (0,1)$. 

\vspace{1mm}
Equivalently, each step can be written as
{\color{red}
\[
x_t = \sqrt{1-\beta_t}\,x_{t-1} + \sqrt{\beta_t}\,\epsilon_t,
\quad \epsilon_t \sim \mathcal{N}(0,I).
\]
}
\end{column}

% ------------------------------------------------------------
\begin{column}[T]{0.36\textwidth}

\vspace{-1mm}
\textbf{Consequences}

\begin{itemize}
  \item Mean is damped at each step
  \item Variance increases monotonically
  \item Signal-to-noise ratio decreases
\end{itemize}

\vspace{1mm}
After many steps:
\[
x_T \;\approx\; \mathcal{N}(0,I)
\]

\vspace{1mm}
This \y{connects data} to \y{pure noise}.

\end{column}

\end{columns}

\end{tightmath}
\end{frame}
