%!TEX root = lec09.tex
% ================================================================================
% Lecture 9 — Slide 17
% ================================================================================
\begin{frame}[t,fragile]

\mytitle{Message Passing: Minimal Code Example}

\begin{columns}[T,totalwidth=\textwidth]

% ------------------------------------------------------------
\begin{column}[T]{0.64\textwidth}

\vspace{-3mm}
\begin{codeonly}{One GNN message-passing layer (conceptual)}
class GNNLayer(nn.Module):
  def __init__(self, dim):
    super().__init__()
    self.self_lin  = nn.Linear(dim, dim)
    self.neigh_lin = nn.Linear(dim, dim)

  def forward(self, x, edge_index):
    row, col = edge_index
    agg = torch.zeros_like(x)
    agg.index_add_(0, row, x[col])
    return F.relu(
        self.self_lin(x) +
        self.neigh_lin(agg) )
\end{codeonly}

\end{column}

% ------------------------------------------------------------
\begin{column}[T]{0.34\textwidth}

\vspace{-1cm}
\textbf{What this shows}

\begin{itemize}
  \item Neighbor features are summed
  \item Same weights used everywhere
  \item Graph defines information flow
\end{itemize}

\vspace{2mm}
\textbf{Key insight}

\begin{itemize}
  \item No coordinates needed
  \item No stencil prescribed
  \item Structure comes from edges
\end{itemize}

\end{column}

\end{columns}

\end{frame}
