%!TEX root = lec06.tex
% ================================================================================
% Lecture 6 — Slide 3
% ================================================================================
\begin{frame}[t,fragile]

\mytitle{From Token IDs to Embedding Vectors}

\begin{columns}[T,totalwidth=\textwidth]

% --- Left column ---------------------------------------------------------------
\begin{column}[T]{0.48\textwidth}

\vspace{-1mm}
\textbf{What the model actually sees}

\begin{itemize}
  \item Tokens are integers
  \item No words, no strings, no grammar
\end{itemize}

\vspace{1mm}
Example:
\[
\{104,\,2871,\,9123,\,214,\,201,\,4421,\,13\}
\]

\vspace{1mm}
\textbf{Problem}

\begin{itemize}
  \item Integers have no semantic meaning
  \item Distance between IDs is arbitrary
\end{itemize}

\end{column}

% --- Right column --------------------------------------------------------------
\begin{column}[T]{0.5\textwidth}

\vspace{-1mm}
\textbf{Solution: embeddings}

Each token ID $i$ is mapped to a vector:
\[
i \;\longrightarrow\; e_i \in \mathbb{R}^{d_{\text{model}}}
\]

\vspace{-2mm}
\textbf{Embedding matrix}

\vspace{-2mm}
\[
E \in \mathbb{R}^{V \times d_{\text{model}}}
\]

\begin{itemize}
  \item $V$: \y{vocabulary size} (number of distinct tokens)
  \item $d_{\text{model}}$: \y{embedding dimension}
  \item $E_i = e_i$: embedding of token $i$
\end{itemize}

\end{column}

\end{columns}

\vspace{3mm}
\centering
{\footnotesize
Meaning is not stored in tokens, but in their \y{learned vector representations.}
}

\end{frame}
