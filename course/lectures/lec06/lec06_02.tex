%!TEX root = lec06.tex
% ================================================================================
% Lecture 6 — Slide 2
% ================================================================================
\begin{frame}[t,fragile]

\mytitle{LLMs as Sequence-to-Sequence Machines}

\begin{columns}[T,totalwidth=\textwidth]

% --- Left column ---------------------------------------------------------------
\begin{column}[T]{0.48\textwidth}

\vspace{-1mm}
\textbf{Core abstraction}

\begin{itemize}
  \item Input: sequence of tokens
  \item Output: sequence of tokens
  \item \y{Tokens} represent words or subwords
\end{itemize}

\vspace{1mm}
\textbf{Mathematical view}

\[
(x_1, \dots, x_n)
\;\longrightarrow\;
(y_1, \dots, y_m)
\]

\vspace{1mm}
\begin{itemize}
  \item Variable-length input
  \item Variable-length output
\end{itemize}

\end{column}

% --- Right column --------------------------------------------------------------
\begin{column}[T]{0.5\textwidth}

\vspace{-1mm}
\textbf{Next-token prediction}

\begin{itemize}
  \item Output generated one token at a time
  \item Each step predicts a probability distribution
\end{itemize}

\vspace{-1mm}
\[
p(y_t \mid x_{1:n}, y_{1:t-1})
\]

\vspace{1mm}
\textbf{Consequences}

\begin{itemize}
  \item Same model for all tasks
  \item Tasks differ only by input prompt
  \item No task-specific architecture needed
\end{itemize}

\end{column}

\end{columns}

\vspace{3mm}
\centering
{\footnotesize
LLMs reduce language understanding and generation to \y{probabilistic sequence modeling.}
}

\end{frame}
