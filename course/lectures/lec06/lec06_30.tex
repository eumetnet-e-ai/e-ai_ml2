%!TEX root = lec06.tex
% ================================================================================
% Lecture 6 — Slide 30
% ================================================================================
\begin{frame}[t,fragile]

\mytitle{Streaming from Ollama using Python}

\begin{columns}[T,totalwidth=\textwidth]

\begin{column}[T]{0.33\textwidth}

\textbf{Why Python streaming?}

\begin{itemize}
  \item Interactive applications
  \item Live UIs 
  \item Custom text processing
\end{itemize}

\vspace{1mm}
\textbf{Mechanism}

\begin{itemize}
  \item HTTP response stream
  \item Incremental JSON decoding
  \item Append text chunks
\end{itemize}

\end{column}

\begin{column}[T]{0.72\textwidth}

\vspace{-3mm}
\begin{codeonly}{Streaming response (Python)}
import requests, json
url = "http://localhost:11434/api/generate"
data = {
  "model": "mistral",
  "prompt": "Explain self-attention.",
  "stream": True }
r = requests.post(url, json=data, stream=True)
for line in r.iter_lines():
    if line:
        msg = json.loads(line)
        print(msg["response"], end="")
\end{codeonly}

\vspace{0mm}
{\footnotesize
Text is printed as soon as tokens arrive.
}

\end{column}

\end{columns}

\vspace{0mm}
\centering
\y{Streaming enables responsive, real-time LLM applications.}

\end{frame}
