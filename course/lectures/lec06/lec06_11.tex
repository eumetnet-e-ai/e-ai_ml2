%!TEX root = lec06.tex
% ================================================================================
% Lecture 6 — Slide 11
% ================================================================================
\begin{frame}[t,fragile]

\mytitle{Attention Matrix Interpretation}

\begin{columns}[T,totalwidth=\textwidth]

% --- Left column ---------------------------------------------------------------
\begin{column}[T]{0.48\textwidth}

\vspace{-1mm}
\textbf{Attention matrix}

\vspace{-4mm}
\[
A = \operatorname{softmax}
\!\left(
\frac{QK^T}{\sqrt{d_k}}
\right)
\quad\in\quad \mathbb{R}^{n \times n}
\]

\vspace{1mm}
\begin{itemize}
  \item Row $i$: attention of token $i$
  \item Columns: which tokens are attended to
\end{itemize}

\vspace{1mm}
\y{\bf Softmax:} Each row sums to 1: 
\[
\sum_j A_{ij} = 1
\]

\end{column}

% --- Right column --------------------------------------------------------------
\begin{column}[T]{0.5\textwidth}

\vspace{-6mm}
\textbf{Interpretation}

\begin{itemize}
  \item Soft, learned dependency graph
  \item Fully connected
  \item Directional (row-wise)
\end{itemize}

\vspace{1mm}
\textbf{Key insight}

\begin{itemize}
  \item No fixed neighborhood
  \item No predefined structure
  \item Recomputed at every layer
\end{itemize}

\vspace{1mm}
\centering
\y{Attention learns which tokens influence}
\y{each other.}

\end{column}

\end{columns}

\vspace{2mm}
\centering
{\footnotesize
\rtext{Self-attention dynamically constructs a weighted interaction graph.}
}

\end{frame}
