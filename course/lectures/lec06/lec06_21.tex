%!TEX root = lec06.tex
% ================================================================================
% Lecture 6 — Slide 19b
% ================================================================================
\begin{frame}[t,fragile]

\mytitle{Self-Attention Layer: Class Setup}

\begin{columns}[T,totalwidth=\textwidth]

% --- Left column ---------------------------------------------------------------
\begin{column}[T]{0.33\textwidth}

\vspace{-1mm}
\textbf{What is defined here}

\begin{itemize}
  \item Learnable weight matrices
  \item Head configuration
  \item No computation yet
\end{itemize}

\vspace{1mm}
\textbf{Key parameters}

\begin{itemize}
  \item $d_{\text{model}}$: embedding dimension
  \item $H$: number of heads
  \item $d_k = d_{\text{model}} / H$
\end{itemize}

\end{column}

% --- Right column --------------------------------------------------------------
\begin{column}[T]{0.66\textwidth}

\vspace{-4mm}
\begin{codeonly}{Self-attention layer (initialization)}
class SelfAttention(nn.Module):
   def __init__(self, d_m, num_heads):
     super().__init__()
     assert d_model % num_heads == 0
     self.num_heads = num_heads
     self.d_k = d_m // num_heads
     
     self.q_linear = nn.Linear(d_m, d_m)
     self.k_linear = nn.Linear(d_m, d_m)
     self.v_linear = nn.Linear(d_m, d_m)
     self.out_linear = nn.Linear(d_m, d_m)
\end{codeonly}
\centering
{\footnotesize
This sets up the attention mechanism structurally.
}

\end{column}

\end{columns}

\vspace{2mm}
\centering
\y{The class defines \emph{what can be learned}, not how it is used.}

\end{frame}
