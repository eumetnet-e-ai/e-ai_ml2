%!TEX root = lec06.tex
% ================================================================================
% Lecture 6 — Slide 4
% ================================================================================
\begin{frame}[t,fragile]

\mytitle{Why Sequence Order Matters}

\begin{columns}[T,totalwidth=\textwidth]

% --- Left column ---------------------------------------------------------------
\begin{column}[T]{0.48\textwidth}

\vspace{-1mm}
\textbf{Problem with embeddings alone}

\begin{itemize}
  \item Embeddings encode token meaning
  \item But \y{ignore position in the sequence}
\end{itemize}

\vspace{1mm}
Example:
\[
\text{dog bites man}
\quad\neq\quad
\text{man bites dog}
\]

\vspace{1mm}
Yet both contain the same tokens.

\end{column}

% --- Right column --------------------------------------------------------------
\begin{column}[T]{0.5\textwidth}

\vspace{-1mm}
\textbf{Key observation}

\begin{itemize}
  \item Transformers process tokens \y{in parallel}
  \item No inherent notion of order
\end{itemize}

\vspace{1mm}
\textbf{Consequence}

\begin{itemize}
  \item Sequence order must be added explicitly
  \item Otherwise:
\end{itemize}
\[
(x_1, x_2, x_3)
\;\equiv\;
(x_3, x_1, x_2)
\]

\vspace{1mm}
\centering
\y{Order information is not optional.}

\end{column}

\end{columns}

\vspace{2mm}
\centering
{\footnotesize
Transformers require an \rtext{explicit mechanism to encode token positions}.
}

\end{frame}
