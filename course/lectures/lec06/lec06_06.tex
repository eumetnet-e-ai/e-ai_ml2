%!TEX root = lec06.tex
% ================================================================================
% Lecture 6 — Slide 6
% ================================================================================
\begin{frame}[t,fragile]

\mytitle{Sinusoidal Positional Encoding}

\begin{columns}[T,totalwidth=\textwidth]

% --- Left column ---------------------------------------------------------------
\begin{column}[T]{0.48\textwidth}

\vspace{-1mm}
\textbf{Sinusoidal construction}

\begin{itemize}
  \item Fixed, deterministic encoding
  \item Different frequencies per dimension
\end{itemize}

For position $i$ and dimension $j$:

\vspace{-4mm}
\[
p_{i,2j}   = \sin\!\left(\frac{i}{10000^{2j/d_{\text{model}}}}\right)
\]
\[
p_{i,2j+1} = \cos\!\left(\frac{i}{10000^{2j/d_{\text{model}}}}\right)
\]

\vspace{1mm}
\begin{itemize}
  \item Smooth variation with position
  \item Relative distances are encoded
\end{itemize}

\end{column}

% --- Right column --------------------------------------------------------------
\begin{column}[T]{0.5\textwidth}

\vspace{-1mm}
\textbf{Why sinusoids?}

\begin{itemize}
  \item Works for arbitrary sequence lengths
  \item No additional parameters
  \item Relative positions can be inferred
\end{itemize}

\vspace{1mm}
\textbf{Why \emph{add}, not concatenate?}

{\scriptsize
\begin{itemize}
  \item Addition keeps dimension at $d_{\text{model}}$
  \item Attention projections expect fixed size
  \item Concatenation doubles dimension
  \item Would require retraining all layers
\end{itemize}
}

\vspace{1mm}
\y{Addition preserves efficiency} 
\y{and model structure.}

\end{column}

\end{columns}

\vspace{1mm}
\centering
{\footnotesize
Positional information is injected without changing model dimensionality.
}

\end{frame}
