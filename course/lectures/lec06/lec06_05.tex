%!TEX root = lec06.tex
% ================================================================================
% Lecture 6 — Slide 5
% ================================================================================
\begin{frame}[t,fragile]

\mytitle{Positional Encoding}

\begin{columns}[T,totalwidth=\textwidth]

% --- Left column ---------------------------------------------------------------
\begin{column}[T]{0.48\textwidth}

\vspace{-1mm}
\textbf{Goal}

\begin{itemize}
  \item Inject \y{order information} into embeddings
  \item Keep full parallelism
\end{itemize}

\vspace{1mm}
\textbf{Idea}

\begin{itemize}
  \item Assign a position-dependent vector
  \item Add it to the token embedding
\end{itemize}

\vspace{1mm}
\[
x_i \;=\; e_i + p_i
\]

\end{column}

% --- Right column --------------------------------------------------------------
\begin{column}[T]{0.5\textwidth}

\vspace{-1mm}
\textbf{Positional encoding matrix}

\[
P \in \mathbb{R}^{n \times d_{\text{model}}}
\]

\begin{itemize}
  \item $n$: sequence length
  \item $p_i$: encoding of position $i$
\end{itemize}

\vspace{1mm}
\textbf{Key properties}

\begin{itemize}
  \item Same dimension as embeddings
  \item Fixed (sinusoidal) or learned
  \item Enables attention to use order
\end{itemize}

\end{column}

\end{columns}

\vspace{2mm}
\centering
{\footnotesize
Positional encodings turn a set of tokens into an ordered sequence.
}

\end{frame}
