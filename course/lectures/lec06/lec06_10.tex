%!TEX root = lec06.tex
% ================================================================================
% Lecture 6 — Slide 10
% ================================================================================
\begin{frame}[t,fragile]

\mytitle{Why Scale by $\sqrt{d_k}$?}

\begin{columns}[T,totalwidth=\textwidth]

% --- Left column ---------------------------------------------------------------
\begin{column}[T]{0.4\textwidth}

\vspace{-1mm}
\textbf{Problem without scaling}

\begin{itemize}
  \item Dot products grow with dimension
  \item Large values enter softmax
  \item Gradients become unstable
\end{itemize}

\vspace{1mm}
For random vectors:
\[
Q_i \cdot K_j \sim \mathcal{O}(d_k)
\]

\end{column}

% --- Right column --------------------------------------------------------------
\begin{column}[T]{0.56\textwidth}

\vspace{-9mm}
\textbf{Effect of scaling}

\begin{itemize}
  \item Divide by $\sqrt{d_k}$
  \item Keeps variance roughly constant
\end{itemize}

\vspace{1mm}
\[
\frac{QK^T}{\sqrt{d_k}}
\]

\vspace{1mm}
\textbf{Result}

\begin{itemize}
  \item Softmax stays sensitive
  \item Stable gradients during training
  \item Faster convergence
\end{itemize}

\vspace{1mm}
\centering
\y{Scaling is a numerical necessity, not a heuristic.}

\end{column}

\end{columns}

\vspace{2mm}
\centering
{\footnotesize
The scaling factor prevents softmax saturation in high dimensions.
}

\end{frame}
