%!TEX root = lec06.tex
% ================================================================================
% Lecture 6 — Slide 7
% ================================================================================
\begin{frame}[t,fragile]

\mytitle{Self-Attention: Core Idea}

\begin{columns}[T,totalwidth=\textwidth]

% --- Left column ---------------------------------------------------------------
\begin{column}[T]{0.48\textwidth}

\vspace{-1mm}
\textbf{Key mechanism}

\begin{itemize}
  \item Each token can \y{look at all other tokens}
  \item Importance is computed dynamically
\end{itemize}

\vspace{1mm}
\textbf{Contextual representation}

\begin{itemize}
  \item Token meaning depends on context
  \item Same word, different role
\end{itemize}

\vspace{1mm}
Example:
\[
\text{``The dog chased the cat.''}
\]

\end{column}

% --- Right column --------------------------------------------------------------
\begin{column}[T]{0.5\textwidth}

\vspace{-10mm}
\textbf{What attention does}

\begin{itemize}
  \item Builds a \y{weighted combination} of tokens
  \item Different focus for each position
\end{itemize}

\vspace{1mm}
For token $i$:
\[
x_i \;\longrightarrow\; 
\sum_j w_{ij}\, x_j
\]

\vspace{-2mm}
\textbf{Why this matters}

\begin{itemize}
  \item Long-range dependencies
  \item No fixed context window
  \item Fully parallel computation
\end{itemize}

\end{column}

\end{columns}

\vspace{4mm}
\centering
{\footnotesize
\rtext{Self-attention allows each token to decide which other tokens matter.}
}

\end{frame}
