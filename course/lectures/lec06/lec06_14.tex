%!TEX root = lec06.tex
% ================================================================================
% Lecture 6 — Slide 14
% ================================================================================
\begin{frame}[t,fragile]

\mytitle{Position-Wise Feedforward Network (FFN)}

\begin{columns}[T,totalwidth=\textwidth]

% --- Left column ---------------------------------------------------------------
\begin{column}[T]{0.54\textwidth}

\begin{itemize}
  \item Inside each Transformer block
  \item After self-attention
\end{itemize}

\vspace{1mm}
\textbf{What attention does not do}

\begin{itemize}
  \item Computes weighted averages
  \item Linear combination of values
  \item No feature-wise nonlinearity
\end{itemize}

\vspace{1mm}
\textbf{Why the FFN is needed}

\begin{itemize}
  \item Adds nonlinearity, Recombines features
  \item Increases model expressiveness
\end{itemize}

\end{column}

% --- Right column --------------------------------------------------------------
\begin{column}[T]{0.43\textwidth}

\vspace{-8mm}
\textbf{Feedforward mapping}

For one token vector $x \in \mathbb{R}^{d_{\text{model}}}$:
\[
\operatorname{FFN}(x)
=
\sigma(x W_1 + b_1)\, W_2 + b_2
\]

\vspace{-2mm}
\begin{itemize}
  \item $\sigma(\cdot)$: ReLU or GELU
  \item $W_1 \in \mathbb{R}^{d_{\text{model}} \times d_{\text{ff}}}$
  \item $W_2 \in \mathbb{R}^{d_{\text{ff}} \times d_{\text{model}}}$
\end{itemize}

\vspace{1mm}
Applied to all tokens:

\vspace{-3mm}
\[
Z_{\text{ff}} \in \mathbb{R}^{n \times d_{\text{model}}}
\]

\vspace{-1mm}
{\footnotesize
Same FFN parameters are shared across all positions.
}

\end{column}

\end{columns}

\vspace{2mm}
\centering
\y{Attention mixes tokens; the FFN transforms token features.}

\end{frame}
