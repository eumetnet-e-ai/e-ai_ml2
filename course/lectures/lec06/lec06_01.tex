%!TEX root = lec06.tex
% ================================================================================
% Lecture 6 — Slide 1
% ================================================================================
\begin{frame}[t,fragile]

\mytitle{Why Large Language Models?}

\begin{columns}[T,totalwidth=\textwidth]

% --- Left column ---------------------------------------------------------------
\begin{column}[T]{0.48\textwidth}

\vspace{-1mm}
\textbf{Traditional NLP}

\begin{itemize}
  \item Separate models for each task
  \item Feature engineering required
  \item Limited transfer between tasks
\end{itemize}

\vspace{1mm}
\textbf{Examples}

\begin{itemize}
  \item Translation
  \item Classification
  \item Question answering
\end{itemize}

\vspace{0mm}
\centering
{\footnotesize
LLMs \y{unify} understanding, generation, and reasoning in a single sequence model.
}

\end{column}

% --- Right column --------------------------------------------------------------
\begin{column}[T]{0.5\textwidth}

\vspace{-4mm}
\textbf{Large Language Models}

\begin{itemize}
  \item One model, many tasks
  \item \y{Learns representations from data}
  \item Includes Sub-Word \rtext{Tokenization}
  \item General-purpose language intelligence
\end{itemize}

\vspace{1mm}
\textbf{Key idea}

\begin{itemize}
  \item Language as a \emph{sequence prediction problem}
  \item Everything reduced to:
\end{itemize}

\vspace{-3mm}
\[
\text{tokens}_{\text{in}} \;\longrightarrow\; \text{tokens}_{\text{out}}
\]

\end{column}

\end{columns}

\vspace{0mm}
\[
\{\rtext{The},\rtext{dog},\rtext{chas},\rtext{ed},\rtext{the},\rtext{cat},\rtext{.}\}
\;\;\longrightarrow\;\;
\{104,\,2871,\,9123,\,214,\,201,\,4421,\,13\}
\]

\end{frame}
