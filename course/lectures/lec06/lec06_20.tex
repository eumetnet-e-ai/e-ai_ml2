%!TEX root = lec06.tex
% ================================================================================
% Lecture 6 — Slide 20
% ================================================================================
\begin{frame}[t]

\mytitle{A Tiny Language Dataset}

\begin{columns}[T,totalwidth=\textwidth]

% --- Left -------------------------------------------------
\begin{column}[T]{0.45\textwidth}

\textbf{Vocabulary}

\begin{itemize}
  \item Small, fixed word set
  \item Each word mapped to an ID
\end{itemize}

\vspace{1mm}
Example:
\[
\text{``I am hungry''}
\;\longrightarrow\;
[1,\,2,\,43]
\]

\vspace{1mm}
\textbf{Padding}

\begin{itemize}
  \item Fixed sequence length
  \item Padding token = 0
\end{itemize}

\end{column}

% --- Right ------------------------------------------------
\begin{column}[T]{0.54\textwidth}

\vspace{-9mm}
\textbf{Training sentences}

{\footnotesize
\begin{itemize}
  \item I am hungry
  \item you are tired
  \item we are happy
  \item they are sad
  \item the weather is nice
\end{itemize}
}

\vspace{1mm}
\textbf{Training pairs}

\vspace{-3mm}
\[
\text{Input: } (x_1,\dots,x_{n-1})
\quad\rightarrow\quad
\text{Target: } (x_1,\dots,x_n)
\]

\vspace{1mm}
\centering
{\footnotesize
Teacher forcing: true previous tokens are known.
}

\end{column}

\end{columns}

\vspace{2mm}
\centering
\y{The model learns language from very simple sequences.}

\end{frame}
