%!TEX root = lec19.tex
% ================================================================================
% Lecture 19 — Slide 13
% ================================================================================
\begin{frame}[t]

\mytitle{Neural SINDy: Results on Noisy Lorenz--63 Data}

\begin{columns}[T,totalwidth=\textwidth]

% ------------------------------------------------------------
\begin{column}[T]{0.35\textwidth}
\footnotesize
\vspace{-2mm}

\textbf{Experimental setup}

\begin{itemize}
  \item Lorenz--63 system
  \item Strong additive noise on observations
  \item Identical SINDy library and sparsity settings
\end{itemize}

\vspace{2mm}
\textbf{Comparison}

\begin{itemize}
  \item Classical SINDy: finite-difference derivatives
  \item Neural SINDy: NN-smoothed trajectory + autograd
\end{itemize}

\end{column}

% ------------------------------------------------------------
\begin{column}[T]{0.63\textwidth}
\centering
\vspace{-2mm}

\includegraphics[width=\textwidth]{../../images/img19/neural_sindy_comparison.png}

\vspace{-2mm}
\raggedright
\footnotesize
Comparison of trajectories, derivatives, and predictions for
classical vs.\ Neural SINDy on noisy data.

\end{column}

\end{columns}

\vspace{0mm}
\textbf{Outcome}

\begin{itemize}
  \item Neural SINDy recovers an improved equation structure
\end{itemize}


\end{frame}
