%!TEX root = lec19.tex
% ================================================================================
% Lecture 19 — Slide 03
% ================================================================================
\begin{frame}[t,fragile]

\mytitle{A Minimal Physics-Informed Neural Network (PINN)}

\begin{columns}[T,totalwidth=\textwidth]

% ------------------------------------------------------------
\begin{column}[T]{0.50\textwidth}
\footnotesize

\textbf{Problem setup}

We consider a simple second-order ODE:
\[
y''(x) + y(x) = 0
\]

with boundary conditions
\[
y(0) = 0, \qquad y'(0) = 1.
\]

\vspace{2mm}
The unique solution is
\[
y(x) = \sin(x).
\]

\vspace{2mm}
This example is deliberately \y{simple}, but already
captures all essential PINN ingredients.

\end{column}

% ------------------------------------------------------------
\begin{column}[T]{0.48\textwidth}
\footnotesize

\textbf{PINN idea}

\begin{itemize}
  \item Approximate $y(x)$ by a neural network $y_\theta(x)$
  \item \y{No training data} $y(x)$ are used
  \item Training is driven by \y{physics constraints}
\end{itemize}

\vspace{2mm}
\textbf{What is enforced}

\begin{itemize}
  \item Differential equation via automatic differentiation
  \item Boundary conditions via penalty terms
\end{itemize}

\vspace{2mm}
\rtext{\bf The network learns the solution by minimizing violations of physics.}

\end{column}

\end{columns}

\end{frame}
