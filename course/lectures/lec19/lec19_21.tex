%!TEX root = lec19.tex
% ================================================================================
% Lecture 19 — Slide 21
% ================================================================================
\begin{frame}[t,fragile]
\begin{tightmath}

\mytitle{Generalization: Advection of Unseen Shapes}

\begin{columns}[T,totalwidth=\textwidth]

% ------------------------------------------------------------
\begin{column}[T]{0.52\textwidth}
\footnotesize
\vspace{-2mm}

\textbf{What was trained}

\begin{itemize}
  \item CNN emulator trained only on \y{Gaussian initial conditions}
  \item Local architecture with \y{periodic padding}
  \item Conservative variant enforces exact mass preservation
\end{itemize}

\vspace{1mm}
\textbf{We test:} Sine waves, Top-hat functions (discontinuous), Multiple separated bumps.

\vspace{1mm}
\textbf{What we observe}

\begin{itemize}
  \item Correct \y{translation on the periodic domain}
  \item Shapes are transported without spurious creation or loss of mass
  \item Superpositions (multiple bumps) are handled consistently
\end{itemize}

\end{column}

% ------------------------------------------------------------
\begin{column}[T]{0.5\textwidth}
\centering
\vspace{-8mm}

\includegraphics[width=0.7\textwidth]{../../images/img19/Advection_Periodic_CNN_B_3_60.png}

\vspace{-1mm}

\includegraphics[width=0.7\textwidth]{../../images/img19/Advection_Periodic_CNN_B_2_60.png}

\vspace{-1mm}

\includegraphics[width=0.7\textwidth]{../../images/img19/Advection_Periodic_CNN_B_4_60.png}

\vspace{1mm}
\raggedright
\footnotesize
Examples of \y{\rtext{\bf unseen initial conditions}} advected by the CNN emulator.


\end{column}

\end{columns}


\end{tightmath}
\end{frame}
