%!TEX root = lec19.tex
% ================================================================================
% Lecture 19 — Slide 10
% ================================================================================
\begin{frame}[t, fragile]

\mytitle{SINDy Sparse Regression of Nonlinear Dynamics: Lorenz--63 System}

\begin{columns}[T,totalwidth=\textwidth]

% ------------------------------------------------------------
\begin{column}[T]{0.50\textwidth}
\footnotesize

\vspace{-2mm}
\textbf{True governing equations}

The Lorenz--63 system is defined by
\[
\begin{aligned}
\dot{x} &= \sigma (y - x), \\
\dot{y} &= x(\rho - z) - y, \\
\dot{z} &= x y - \beta z,
\end{aligned}
\qquad
\begin{minipage}{3cm}$
\sigma=10, \\ 
\rho=28, \\
\beta=\tfrac{8}{3}.
$
\end{minipage}
\]

\vspace{0mm}
\textbf{SINDy function library}

SINDy assumes the dynamics can be written as a sparse linear combination of
candidate functions:

\begin{lstlisting}[basicstyle=\ttfamily\scriptsize]
Theta(x,y,z) = [  1,  x, y, z,
                      x*y, x*z, y*z ]
\end{lstlisting}

Only a few of these terms are retained in each equation after sparse
regression.

\vspace{2mm}
\rtext{\bf Goal:} \y{recover the correct active terms} and coefficients
directly from time series data.

\end{column}

% ------------------------------------------------------------
\begin{column}[T]{0.48\textwidth}
\centering
\vspace{-2mm}

\includegraphics[width=6cm]{../../images/img19/sindy_timeseries.png}

\vspace{2mm}
\raggedright
\footnotesize
Time series of the Lorenz--63 state variables used as input for SINDy
{\color{blue}(blue)}.
Numerical derivatives are estimated and matched against the candidate
library during sparse regression. \color{darkgreen}
Discovered dynamics evolution in Green.

\end{column}

\end{columns}

\end{frame}
