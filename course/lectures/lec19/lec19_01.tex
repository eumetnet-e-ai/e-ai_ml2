%!TEX root = lec19.tex
% ================================================================================
% Lecture 19 — Slide 01
% ================================================================================

\begin{frame}[t,fragile]

\mytitle{Lecture 19: AI and Physics and Data}

\begin{columns}[T,totalwidth=\textwidth]

% ------------------------------------------------------------
\begin{column}[T]{0.54\textwidth}
\footnotesize
\vspace{-2mm}

\textbf{Core question}

Given a dynamical system, what can machine learning do?

\vspace{1mm}
\begin{itemize}
  \item Solve known equations (ODE/PDE)
  \item Discover unknown governing laws from observations
  \item Emulate complex dynamics as a surrogate model
\end{itemize}

\vspace{3mm}
\textbf{Unifying viewpoint}

All methods impose (explicitly or implicitly) a constraint:

\rtext{\bf state evolution must be consistent with } $\dot{\mathbf{x}} = \mathbf{f}(\mathbf{x})$.


\vspace{3mm}
How can different ML approaches enforce this consistency?

\end{column}

% ------------------------------------------------------------
\begin{column}[T]{0.42\textwidth}
\footnotesize
\vspace{-2mm}

\textbf{Three routes in this lecture}

\begin{enumerate}
  \item \textbf{PINNs:} \y{physics drives training}
  \item \textbf{SINDy:} \y{sparse laws from data}
  \item \textbf{Neural RHS learning:} \y{black-box emulation}
\end{enumerate}

\vspace{2mm}
\textbf{Key trade-offs}

\begin{itemize}
  \item Accuracy vs.\ interpretability
  \item Data-efficiency vs.\ flexibility
  \item Stability / extrapolation vs.\ expressiveness
\end{itemize}

\vspace{2mm}
\rtext{\bf Message:} \\
same goal (dynamics), different framework.

\end{column}

\end{columns}

\end{frame}
