%!TEX root = lec19.tex
% ================================================================================
% Lecture 19 — Slide 15
% ================================================================================
\begin{frame}[t]

\mytitle{Neural RHS Learning in State Space}

\begin{columns}[T,totalwidth=\textwidth]

% ------------------------------------------------------------
\begin{column}[T]{0.51\textwidth}

\footnotesize
\vspace{-2mm}

\textbf{Experimental setup}

\begin{itemize}
  \item Lorenz--63 vector field \y{sampled in state space}
  \item Random training points covering a 3D domain
  \item Neural network trained on $(\mathbf{x},\dot{\mathbf{x}})$ pairs
\end{itemize}

\vspace{0mm}
\textbf{Key difference to trajectory-based learning}

\begin{itemize}
  \item Training data no longer restricted to the attractor
  \item Vector field is constrained in a full region of state space
  \item Dynamics are learned as a global mapping
\end{itemize}

\vspace{0mm}
\rtext{\bf Outcome:}

The learned neural vector field reproduces the Lorenz dynamics
consistently from unseen initial conditions, not only along the
original trajectory.

\end{column}

% ------------------------------------------------------------
\begin{column}[T]{0.48\textwidth}

\centering
\vspace{-8mm}

\includegraphics[width=8cm]{../../images/img19/rhs_learning_space_sampling_crop.png}

\end{column}

\end{columns}

\end{frame}
