%!TEX root = lec19.tex
% ================================================================================
% Lecture 19 — Slide 09
% ================================================================================
\begin{frame}[t]

\mytitle{SINDy: Sparse Identification of Nonlinear Dynamics}

\begin{columns}[T,totalwidth=\textwidth]

% ------------------------------------------------------------
\begin{column}[T]{0.46\textwidth}
\footnotesize

\vspace{-2mm}
\textbf{Problem setting}

We observe a dynamical system
\[
\dot{\mathbf{x}}(t) = \mathbf{f}(\mathbf{x}(t)),
\qquad
\mathbf{x}(t)\in\mathbb{R}^n,
\]
from time series data $\mathbf{x}(t_i)$.

\vspace{2mm}
\textbf{Key assumption (sparsity)}

The vector field can be written as a sparse combination of candidate functions:
\[
\dot{\mathbf{x}}(t)
\;\approx\;
\Theta(\mathbf{x}(t))\,\Xi,
\]
where
\begin{itemize}
  \item $\Theta(\mathbf{x})$ is a library of functions
        (e.g.\ $1, x, y, z, xy, xz, yz,\dots$),
  \item $\Xi$ is a \y{sparse coefficient matrix}.
\end{itemize}

\end{column}

% ------------------------------------------------------------
\begin{column}[T]{0.55\textwidth}

\vspace{-2mm}
\textbf{Identification}

SINDy solves a sequence of
\[
\min_{\Xi}\;\|\Theta(\mathbf{x})\Xi-\dot{\mathbf{x}}\|_2^2
\]
with thresholding to eliminate small coefficients.


\includegraphics[width=\textwidth]{../../images/img19/SINDy_01.png}

\vspace{-3mm}
\raggedright
\footnotesize
\textbf{Lorenz--63 example (noise-free).}

Left: true trajectory.  
Right: trajectory from \emph{SINDy}.

\end{column}

\end{columns}

\end{frame}
