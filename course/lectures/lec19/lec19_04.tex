%!TEX root = lec19.tex
% ================================================================================
% Lecture 19 — Slide 04
% ================================================================================
\begin{frame}[t,fragile]

\mytitle{PINN Loss: Physics Instead of Data}

\begin{columns}[T,totalwidth=\textwidth]

% ------------------------------------------------------------
\begin{column}[T]{0.52\textwidth}
\footnotesize

\textbf{ODE residual}

Using automatic differentiation, we compute
\[
y_\theta'(x), \qquad y_\theta''(x).
\]

The differential equation is enforced by minimizing
\[
r(x) = y_\theta''(x) + y_\theta(x).
\]

The corresponding loss term is
\[
\mathcal{L}_{\text{ODE}}
=
\frac{1}{N}
\sum_{i=1}^N
\bigl( y_\theta''(x_i) + y_\theta(x_i) \bigr)^2.
\]

\vspace{2mm}
Collocation points $x_i$ are sampled in the domain.

\end{column}

% ------------------------------------------------------------
\begin{column}[T]{0.46\textwidth}
\footnotesize

\vspace{-4mm}
\textbf{Boundary conditions}

Boundary (anchor) constraints enforce uniqueness:
\[
y_\theta(0) = 0,
\qquad
y_\theta'(0) = 1.
\]

This yields the boundary loss
\[
\mathcal{L}_{\text{BC}}
=
\bigl( y_\theta(0) \bigr)^2
+
\bigl( y_\theta'(0) - 1 \bigr)^2.
\]

\vspace{2mm}
\textbf{Total loss}

The parameter $\lambda$ controls the \y{strength of the boundary conditions.}

\[
\mathcal{L}
=
\mathcal{L}_{\text{ODE}}
+
\lambda\,\mathcal{L}_{\text{BC}}.
\]


\vspace{2mm}
\rtext{\bf No data term appears anywhere in the loss.}

\end{column}

\end{columns}

\end{frame}
