%!TEX root = lec19.tex
% ================================================================================
% Lecture 19 — Slide 26
% ================================================================================
\begin{frame}[t,fragile]
\begin{tightmath}

\mytitle{Neural Causal Discovery: Scaling Ideas to Complex Systems}

\footnotesize

\begin{columns}[T,totalwidth=\textwidth]

% ------------------------------------------------------------
\begin{column}[T]{0.5\textwidth}
\vspace{-2mm}

\textbf{Motivation}

\begin{itemize}
  \item Earth-system variables are high-dimensional
  \item dependencies are nonlinear and state-dependent
  \item autocorrelation dominates raw signals
\end{itemize}

\vspace{0mm}
\textbf{Neural causal discovery idea}

\begin{itemize}
  \item first remove \y{self-dependence} (autocorrelation)
  \item learn cross-effects from \y{innovations}
  \item enforce sparsity $\Rightarrow$ interpretable directed links
\end{itemize}

\vspace{0mm}
\textbf{Innovation form:} $x(t) - a\,x(t-1)$

\vspace{0mm}
\rtext{\bf Key message:}
AI helps to \y{extend classical causal ideas}
to nonlinear, high-dimensional dynamics.

\end{column}

% ------------------------------------------------------------
\begin{column}[T]{0.50\textwidth}
\vspace{-2mm}
\centering

\includegraphics[width=0.75\textwidth]{../../images/img19/neural_causal_discovery.png}

\vspace{1mm}
\raggedright
\footnotesize
\textbf{Example result.}
After removing autocorrelation, sparse learning recovers directed links:
Scenario 1: \y{$F\rightarrow P\rightarrow T$}, Scenario 2: \y{$F\rightarrow P$, $F\rightarrow T$}.

\end{column}

\end{columns}

\end{tightmath}
\end{frame}
