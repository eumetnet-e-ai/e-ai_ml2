%!TEX root = lec19.tex
% ================================================================================
% Lecture 19 — Slide 06
% ================================================================================
\begin{frame}[t]

\mytitle{Improving Extrapolation by Wider Residual Sampling}

\begin{columns}[T,totalwidth=\textwidth]

% ------------------------------------------------------------
\begin{column}[T]{0.50\textwidth}
\footnotesize

\textbf{Key modification}

The PINN formulation is unchanged, but the
\y{ODE residual is enforced on a wider domain}.

\vspace{2mm}
\textbf{Training setup}

\begin{itemize}
  \item ODE residual sampled beyond $[0,\,2\pi]$
  \item Boundary conditions still imposed at $x=0$
  \item Same network architecture and loss terms
\end{itemize}

\vspace{2mm}
\textbf{Effect}

\begin{itemize}
  \item Physics is enforced more \y{globally}
  \item Extrapolation becomes significantly more stable
  \item No change in representation or constraints
\end{itemize}

\end{column}

% ------------------------------------------------------------
\begin{column}[T]{0.48\textwidth}
\centering
\vspace{-2mm}

\includegraphics[width=\textwidth]{../../images/img19/pinn_sine_3.png}


\vspace{2mm}
\rtext{ This already fixes many extrapolation problems for simple ODEs.}


\end{column}

\end{columns}

\end{frame}
