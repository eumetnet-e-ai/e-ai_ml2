%!TEX root = lec19.tex
% ================================================================================
% Lecture 19 — Slide 22
% ================================================================================
\begin{frame}[t,fragile]
\begin{tightmath}

\mytitle{Correlation vs Causality in Dynamical Systems}

\footnotesize

\begin{columns}[T,totalwidth=\textwidth]

% ------------------------------------------------------------
\begin{column}[T]{0.50\textwidth}
\vspace{-2mm}

\textbf{The trap}

\begin{itemize}
  \item Many variables in geosciences are \y{strongly correlated}
  \item But correlation alone cannot identify \y{directionality}
  \item Hidden drivers (\y{confounders}) can create spurious links
\end{itemize}

\vspace{0mm}
\textbf{Causal question}

\vspace{-2mm}
\[
P(T\,|\,P) \quad \neq \quad P\bigl(T\,|\,do(P)\bigr)
\]

\vspace{-1mm}
\textbf{Why time series are special}

\begin{itemize}
  \item Dynamics introduce \y{time ordering}
  \item Causal effects typically appear with \y{lags}
  \item Strong autocorrelation can mask cross-effects
\end{itemize}

\end{column}

% ------------------------------------------------------------
\begin{column}[T]{0.48\textwidth}
\vspace{-2mm}

\textbf{Key idea}

\begin{itemize}
  \item Use \y{lagged dependencies} to separate:
  \begin{itemize}
    \item self-dynamics (memory)
    \item cross-variable influences
  \end{itemize}
\end{itemize}

\vspace{2mm}
\textbf{In Earth-system applications this matters because}

\begin{itemize}
  \item interventions (what-if) require causal structure
  \item attribution needs confounding control
  \item robust extrapolation benefits from causal mechanisms
\end{itemize}

\vspace{0mm}
\rtext{\bf Take-home message:}

\vspace{-4mm}
\[
\text{\rtext{similar correlations} do not imply \rtext{same physics}}
\]

\end{column}

\end{columns}

\end{tightmath}
\end{frame}
