%!TEX root = lec19.tex
% ================================================================================
% Lecture 19 — Slide 13
% ================================================================================
\begin{frame}[t]

\begin{tightmath}

\mytitle{Alternative: Learning the Full RHS with Neural Networks}

\begin{columns}[T,totalwidth=\textwidth]

% ------------------------------------------------------------
\begin{column}[T]{0.48\textwidth}
\footnotesize

\vspace{-2mm}
\textbf{Black-box RHS learning}

Instead of discovering equations, one may directly learn the vector field
\[
\dot{\mathbf{x}} = \mathbf{f}_\theta(\mathbf{x}),
\]
where $\mathbf{f}_\theta$ is a neural network.


\vspace{4mm}
The network is trained to match observed time derivatives or trajectories.

\begin{itemize}
  \item Very flexible function class
  \item \y{Can approximate complex, unknown} \y{dynamics}
  \item Naturally handles noise with regularization
\end{itemize}

\end{column}

% ------------------------------------------------------------
\begin{column}[T]{0.48\textwidth}
\footnotesize
\vspace{0mm}

\textbf{Neural SINDy: a different goal}

Neural SINDy uses neural networks only to stabilize intermediate steps
(denoising and differentiation), but still identifies
\[
\dot{\mathbf{x}} \approx \Theta(\mathbf{x})\,\Xi,
\]
with a sparse, explicit structure.

\vspace{4mm}
\rtext{\bf Key distinction:}

\begin{itemize}
  \item Neural RHS learning $\rightarrow$ \y{predictive black box}
  \item Neural SINDy $\rightarrow$ \emph{interpretable equations}
\end{itemize}

\vspace{0mm}
This distinction is critical in scientific modeling, where the goal
is understanding, not only prediction.

\end{column}

\end{columns}

\end{tightmath}

\end{frame}
