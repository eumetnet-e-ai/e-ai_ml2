%!TEX root = lec19.tex
% ================================================================================
% Lecture 19 — Slide 08
% ================================================================================
\begin{frame}[t]

\mytitle{Result: Fourier-Feature PINN for a Modulated Oscillation}

\begin{columns}[T,totalwidth=\textwidth]

% ------------------------------------------------------------
\begin{column}[T]{0.48\textwidth}
\footnotesize

\textbf{What changed compared to previous examples}

\begin{itemize}
  \item Governing ODE and anchor conditions unchanged
  \item Same PINN loss formulation
  \item \y{Only the input representation is modified}
\end{itemize}

\vspace{2mm}
The network uses a Fourier feature embedding.

\vspace{2mm}
\textbf{Observed behavior}

\begin{itemize}
  \item Accurate solution on the training domain $[0,\,5\pi]$
  \item \y{Stable extrapolation} far beyond the training region
  \item Correct phase and amplitude over many oscillations
\end{itemize}

\end{column}

% ------------------------------------------------------------
\begin{column}[T]{0.50\textwidth}
\centering
\vspace{-2mm}

\includegraphics[width=\textwidth]{../../images/img19/pinn_sine_cosine.png}

\vspace{2mm}
\raggedright
\footnotesize
Gray shading indicates the training domain.
The solution is evaluated well beyond the region where the ODE residual
was enforced.


\vspace{2mm}
\rtext{\bf Representation choice alone can control extrapolation quality.}

\end{column}

\end{columns}

\end{frame}
