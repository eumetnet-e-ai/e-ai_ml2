%!TEX root = lec19.tex
% ================================================================================
% Lecture 19 — Slide 02
% ================================================================================

\begin{frame}[t,fragile]

\mytitle{Lecture Roadmap}

\begin{columns}[T,totalwidth=\textwidth]

% ------------------------------------------------------------
\begin{column}[T]{0.48\textwidth}
\footnotesize
\vspace{-2mm}

\textbf{Part I — Physics-Informed Neural Networks}

\vspace{1mm}
\begin{itemize}
  \item learn \y{solutions} of known equations
  \item training uses \y{ODE/PDE residuals} + anchor conditions
  \item representation matters for extrapolation
\end{itemize}

\vspace{3mm}
\textbf{Part II — Discovering equations from data}

\vspace{1mm}
\begin{itemize}
  \item \textbf{SINDy:} sparse regression on a function library
  \item \textbf{Neural SINDy:} smooth NN trajectory $\Rightarrow$ stable derivatives
\end{itemize}

\end{column}

% ------------------------------------------------------------
\begin{column}[T]{0.52\textwidth}
\footnotesize
\vspace{-9mm}

\textbf{Part III — Learning the Force Term}

\vspace{1mm}
\begin{itemize}
  \item learn the unknown RHS / forcing from data:
  \[
  \dot{\mathbf{x}} = \mathbf{f}(\mathbf{x}) + \mathbf{g}_\theta(\mathbf{x})
  \]
  \item \y{hybrid modeling:} known physics + learned closure
  \item stable rollout by integrating the learned system
\end{itemize}

\vspace{2mm}
\textbf{Part IV — Causal Modeling with Neural Networks}

\vspace{1mm}
\begin{itemize}
  \item distinguish \y{correlation} vs.\ \y{cause}
  \item learn structural relations (SCMs) as NN modules
  \item intervene and test counterfactual predictions:
  \[
  do(X=x)
  \quad\Rightarrow\quad
  p(Y\,|\,do(X=x))
  \]
\end{itemize}

\end{column}

\end{columns}

\end{frame}
