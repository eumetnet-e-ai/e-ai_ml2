%!TEX root = lec19.tex
% ================================================================================
% Lecture 19 — Slide 18
% ================================================================================
\begin{frame}[t,fragile]
\begin{tightmath}

\mytitle{CNN Emulator without Mass Conservation}

\begin{columns}[T,totalwidth=\textwidth]

% ------------------------------------------------------------
\begin{column}[T]{0.48\textwidth}
\footnotesize
\vspace{-2mm}

\textbf{Neural model}

\begin{itemize}
  \item Local 1D CNN with \y{circular padding}
  \item Learns one-step map $u^n \mapsto u^{n+1}$
  \item Trained by minimizing one-step MSE
\end{itemize}

\vspace{0mm}
\textbf{What is \rtext{not} enforced}

\begin{itemize}
  \item No mass conservation constraint
  \item No global invariant control
\end{itemize}

\vspace{0mm}
\textbf{Observed behavior}

\begin{itemize}
  \item \y{Low training loss}
  \item Smooth short-term evolution
  \item \rtext{Gradual drift in total mass}
\end{itemize}

\vspace{0mm}
\rtext{\bf Locality alone is not enough to guarantee physical correctness.}

\end{column}

% ------------------------------------------------------------
\begin{column}[T]{0.50\textwidth}
\centering
\vspace{-4mm}

\includegraphics[width=0.9\textwidth]{../../images/img19/Advection_Periodic_CNN_3_0.png}

\vspace{-2mm}

\includegraphics[width=0.9\textwidth]{../../images/img19/Advection_Periodic_CNN_3_60.png}

\vspace{0mm}
\raggedright
\footnotesize
\textbf{CNN without conservation.}
Small amplitude and mass errors accumulate despite visually plausible transport.

\end{column}

\end{columns}

\end{tightmath}
\end{frame}
