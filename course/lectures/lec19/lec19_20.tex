%!TEX root = lec19.tex
% ================================================================================
% Lecture 19 — Slide 20
% ================================================================================
\begin{frame}[t,fragile]
\begin{tightmath}

\mytitle{Long-Time Rollout: Accuracy vs Physical Validity}

\begin{columns}[T,totalwidth=\textwidth]

% ------------------------------------------------------------
\begin{column}[T]{0.46\textwidth}
\footnotesize
\vspace{-2mm}

\textbf{Long-time test}

\begin{itemize}
  \item Rollout over hundreds of time steps
  \item Same initial condition
  \item Compare truth, CNN-free, CNN-conservative
\end{itemize}

\vspace{1mm}
\textbf{Key observation}

\begin{itemize}
  \item CNN-free: errors accumulate steadily
  \item CNN-conservative: structure remains coherent
  \item Physics constraints matter most \y{far beyond training horizon}
\end{itemize}

\vspace{2mm}
{\bf Short-term accuracy is not a proxy for long-term validity.}

\end{column}

% ------------------------------------------------------------
\begin{column}[T]{0.52\textwidth}
\centering
\vspace{-3mm}

\includegraphics[width=0.9\textwidth]{../../images/img19/Advection_Periodic_CNN_Long_0.png}

\vspace{-1mm}

\includegraphics[width=0.9\textwidth]{../../images/img19/Advection_Periodic_CNN_Long_600.png}

\vspace{1mm}
\raggedright
\footnotesize
\rtext{\bf Only the conservative model maintains physically consistent transport.}

\end{column}

\end{columns}


\end{tightmath}
\end{frame}
