%!TEX root = lec19.tex
% ================================================================================
% Lecture 19 — Slide 17
% ================================================================================
\begin{frame}[t,fragile]
\begin{tightmath}

\mytitle{Physics-Constrained Neural Emulators: 1D Periodic Advection}

\begin{columns}[T,totalwidth=\textwidth]

% ------------------------------------------------------------
\begin{column}[T]{0.48\textwidth}
\footnotesize
\vspace{-2mm}

We consider \y{1D linear advection} on a periodic domain:

\vspace{-4mm}
\[
\partial_t u + c\,\partial_x u = 0,
\qquad x \in [0,1], \;\; u(0)=u(1).
\]

\vspace{0mm}
Key physical properties:
\begin{itemize}
  \item Pure \y{translation on a ring}
  \item Exact \y{mass conservation}
  \item Smooth initial conditions remain smooth
\end{itemize}

\vspace{1mm}
\textbf{Learning task}

\begin{itemize}
  \item Learn a one-step map \y{$u^n \mapsto u^{n+1}$}
  \item Training data from a conservative upwind solver
  \item Compare different neural inductive biases
\end{itemize}


\end{column}

% ------------------------------------------------------------
\begin{column}[T]{0.50\textwidth}
\centering
\vspace{-3mm}

\includegraphics[width=\textwidth]{../../images/img19/Advection_Periodic_Truth.png}

\vspace{1mm}
\raggedright
\footnotesize
\textbf{Reference solution (truth).}

Gaussian bump advected periodically.
Snapshots at $t=0,30,60$ clearly show wrap-around and mass preservation.

\vspace{4mm}
\rtext{\bf Physics is simple — but violations are immediately visible.}


\end{column}

\end{columns}

\end{tightmath}
\end{frame}
