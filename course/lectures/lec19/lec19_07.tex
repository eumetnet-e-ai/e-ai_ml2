%!TEX root = lec19.tex
% ================================================================================
% Lecture 19 — Slide 07
% ================================================================================
\begin{frame}[t]

\mytitle{Improving Representation with Fourier Features}

\begin{columns}[T,totalwidth=\textwidth]

% ------------------------------------------------------------
\begin{column}[T]{0.49\textwidth}
\footnotesize

\vspace{-2mm}
\textbf{Motivation}

Standard MLPs learn functions of $x$ that are biased toward
\y{smooth, slowly varying behavior}.

\vspace{2mm}
Oscillatory solutions, such as
\[
y(x) = \sin(x),
\]
are therefore harder to represent and extrapolate.

\vspace{2mm}
\textbf{Fourier feature idea}

Instead of learning directly from $x$, we apply a fixed
\y{feature map}:
\[
x \;\mapsto\;
\bigl(
\sin(\omega_k x),\;
\cos(\omega_k x)
\bigr)_{k=1}^m.
\]

\vspace{2mm}
The neural network then learns a function
of these periodic features.

\end{column}

% ------------------------------------------------------------
\begin{column}[T]{0.52\textwidth}
\footnotesize
\centering
\vspace{-3mm}

\includegraphics[width=6cm]{../../images/img19/pinn_sine_4.png}

\vspace{0mm}
\raggedright
\textbf{Effect on the PINN}

\begin{itemize}
  \item Periodicity is \y{easy to represent}
  \item Long-range extrapolation improves
  \item Fewer parameters are needed
\end{itemize}

\vspace{0mm}
\textbf{Interpretation}

\begin{itemize}
  \item Linear models become \y{Fourier series fits}
  \item Nonlinear MLPs allow mode interactions
\end{itemize}

\vspace{0mm}
\rtext{\bf This changes the representation, not the physics.}

\end{column}

\end{columns}

\end{frame}
