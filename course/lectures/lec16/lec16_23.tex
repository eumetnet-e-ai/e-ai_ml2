%!TEX root = lec16.tex
% ================================================================================
% Lecture 16, Slide 16
% ================================================================================
\begin{frame}[t]
  \mytitle{Nowcasting and NWP: One Continuum (Observations $\rightarrow$ DA $\rightarrow$ Forecast)}

\begin{columns}[T,totalwidth=\textwidth]

% ------------------------------------------------------------
\begin{column}[T]{0.52\textwidth}
\footnotesize
\vspace{2mm}

\textbf{\textcolor{blue}{NWP with Data Assimilation (DA) is the full framework}}

\vspace{0mm}
\begin{itemize}
  \item \textbf{observations} enter continuously:
    {\tiny \newline radar, satellite, aircraft, surface stations, etc.}
  \item \textbf{DA} combines obs + model:
    {\tiny \newline best estimate of the 3D atmosphere at ``now''}
  \item then \textbf{forecast propagation} produces future weather
\end{itemize}

\vspace{0mm}
\textbf{\textcolor{violet}{Nowcasting is the short-range regime}}

\vspace{1mm}
\begin{itemize}
  \item lead times: \textbf{minutes to a few hours}
  \item very high weight of recent observations
  \item focus on rapidly evolving phenomena:
    {\tiny \newline convective storms, precipitation cells}
\end{itemize}

\end{column}

% ------------------------------------------------------------
\begin{column}[T]{0.48\textwidth}
\centering
\vspace{2mm}

\includegraphics[width=\textwidth]{../../images/img16/nwc-to-fcst_crop.png}

\vspace{3mm}
{\tiny \textcolor{gray}{
AI can support different parts: observations, DA, model emulation, and products.
}}

\hspace*{-8mm}\y{There is no strict boundary: it is one continuum.}

\end{column}

\end{columns}

\end{frame}
