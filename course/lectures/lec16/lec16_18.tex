%!TEX root = lec16.tex
% ================================================================================
% Lecture 16 — Slide 18
% ================================================================================
\begin{frame}[t]
\mytitle{LLM Orchestration: Decisions \& Guardrails}

\begin{columns}[T,totalwidth=\textwidth]

% ------------------------------------------------------------
\begin{column}[T]{0.52\textwidth}
\footnotesize
\vspace{2mm}

\textbf{LLM = Decision Engine}

\vspace{2mm}
For each user request, the model decides \y{what happens next}:

\vspace{1mm}
\begin{itemize}
  \item answer directly (when trivial)
  \item call a function tool (fast action)
  \item delegate to an agent (multi-step work)
  \item ask a clarifying question (missing info)
\end{itemize}

\vspace{2mm}
\textbf{Key idea:}
The model routes the task to the best capability.

\vspace{1mm}
{\scriptsize \textit{(Like a dispatcher: choose the right action at the right time.)}}

\end{column}

% ------------------------------------------------------------
\begin{column}[T]{0.48\textwidth}
\footnotesize
\vspace{2mm}

\textbf{Guardrails = System Control}

\vspace{2mm}
The system stays in charge of \y{execution}:

\vspace{1mm}
\begin{itemize}
  \item only approved tools exist (tool registry)
  \item arguments are validated (schemas)
  \item execution runs outside the LLM
  \item logging \& reproducibility by design
\end{itemize}

\vspace{2mm}
\textbf{Result:}
Reliable actions, not hallucinated actions.

\vspace{1mm}
{\scriptsize \textit{The LLM proposes --- the backend disposes.}}

\end{column}

\end{columns}

\end{frame}
