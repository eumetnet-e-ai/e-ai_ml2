%!TEX root = lec16.tex
% ================================================================================
% Lecture 16, Slide 20
% ================================================================================
\begin{frame}[t]
  \mytitle{One Step, Many Steps: Templates + Transport + Accumulations}

\begin{columns}[T,totalwidth=\textwidth]

% ------------------------------------------------------------
\begin{column}[T]{0.48\textwidth}
\footnotesize
\vspace{0mm}

\textbf{\textcolor{blue}{Neural forecasting learns reusable building blocks}}

\vspace{1mm}
\begin{itemize}
  \item learn \textbf{templates} for structures:
    {\tiny \newline fronts, rain bands, vortices, jets}
  \item learn \textbf{how they move and change} from context
    {\tiny \newline (e.g. wind, humidity, stability)}
\end{itemize}

\vspace{0mm}
\textbf{\textcolor{blue}{Then: one step $\rightarrow$ many steps}}

\vspace{-2mm}
\[
x(t) \rightarrow x(t+\Delta t) \rightarrow x(t+2\Delta t) \rightarrow \dots
\]

\vspace{0mm}
\textbf{\textcolor{red}{Important for users: accumulated products}}

\vspace{1mm}
\begin{itemize}
  \item many outputs are stored as \textbf{accumulations}
  \item 1-hour precipitation at lead $L$ is:
\end{itemize}

\end{column}

% ------------------------------------------------------------
\begin{column}[T]{0.48\textwidth}

\vspace{1mm}
\[
RAIN(L) - RAIN(L-1)
\]

\vspace{0mm}
\textbf{\textcolor{violet}{Take-away:}}
\quad \y{Forecasting = detect patterns +} \\
\hfill \y{transport them forward in time.}


\centering
\vspace{4mm}

\includegraphics[width=\linewidth]{../../images/img16/nn_front_template_motion.png}

\vspace{1mm}
{\tiny \textcolor{gray}{
Templates $\rightarrow$ heatmaps $\rightarrow$ shifted patterns $\rightarrow$ next map
(repeated for longer lead times).
}}
\end{column}

\end{columns}

\end{frame}
