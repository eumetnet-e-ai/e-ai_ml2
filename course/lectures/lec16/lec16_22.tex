%!TEX root = lec16.tex
% ================================================================================
% Lecture 16, Slide 15
% ================================================================================
\begin{frame}[t]
  \mytitle{Neural Network Forecasting: One Step $\rightarrow$ Many Steps}

\begin{columns}[T,totalwidth=\textwidth]

% ------------------------------------------------------------
\begin{column}[T]{0.48\textwidth}
\footnotesize
\vspace{1mm}

\textbf{\textcolor{blue}{How an AI forecast is produced}}

\vspace{2mm}
\begin{itemize}
  \item The model learns: \y{state now} $\rightarrow$ \y{state later}
  \item Longer lead times come from \textbf{repeating the prediction step}
\end{itemize}

\vspace{-2mm}
\[
x(t) \rightarrow x(t+\Delta t) \rightarrow x(t+2\Delta t) \rightarrow \dots
\]

\vspace{3mm}
\textbf{\textcolor{red}{Physics detail: accumulated variables}}

\vspace{1mm}
\begin{itemize}
  \item precipitation is often stored as an \textbf{accumulated sum}
  \item to get \textbf{1-hour rain} at lead $L$:
\end{itemize}

\vspace{-3mm}
\[
RAIN(L) - RAIN(L-1)
\]

\end{column}

% ------------------------------------------------------------
\begin{column}[T]{0.52\textwidth}
\centering
\vspace{1mm}

\includegraphics[width=\linewidth]{../../images/img16/rain_gsp_1h_2x4_panels_crop.png}

\vspace{1mm}
\textcolor{gray}{
Example: ICON-EU 1-hour precipitation for multiple lead times.
}

\vspace{3mm}
\rtext{\bf Here: Nowcasting or NWP - depending on input, time scales and variables!}
\end{column}

\end{columns}

\end{frame}
