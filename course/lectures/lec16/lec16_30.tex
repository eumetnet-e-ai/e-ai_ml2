%!TEX root = lec16.tex
% ================================================================================
% Lecture 16 — Slide 30
% ================================================================================
\begin{frame}[t]
\mytitle{Inside the CNN: Many Feature Variables}

\begin{columns}[T,totalwidth=\textwidth]

% ------------------------------------------------------------
\begin{column}[T]{0.44\textwidth}
\footnotesize
\vspace{-2mm}

The CNN transforms the input into many internal variables:

\vspace{2mm}
\begin{center}
{\Large $1 \rightarrow 32 \rightarrow 32 \rightarrow 32 \rightarrow 1$}
\end{center}

\vspace{2mm}
\begin{itemize}
  \item input: one function $f(t,x)$
  \item hidden layers: 32 feature channels each
  \item output: predicted function $\hat f(t+\Delta,x)$
\end{itemize}

\vspace{2mm}
\textbf{Interpretation:}
each channel is a learned detector for local shapes
(edges, slopes, curvature, \dots).

\end{column}

% ------------------------------------------------------------
\begin{column}[T]{0.48\textwidth}
\vspace{-8mm}
\begin{center}
\includegraphics[width=\linewidth]{../../images/img16/cnn_translation_features_h1.png}

\vspace{1mm}
{\scriptsize \textit{Example: 6 of 32 feature channels in layer $h1$.}}
\end{center}

\hspace{-12mm}
\begin{minipage}{7.5cm}
\footnotesize
The variables $h_1,h_2,h_3$ are \y{hidden feature maps} inside the CNN.
They form a \y{latent feature space}: not observed, not physical, but learned.
Each layer contains many channels (e.g.\ 32), extracting different local aspects of the signal.
The forecast emerges by transforming $f(t,\cdot)\rightarrow h_1\rightarrow h_2\rightarrow h_3\rightarrow \hat f(t+\Delta,\cdot)$.
\end{minipage}

\end{column}

\end{columns}
\end{frame}
