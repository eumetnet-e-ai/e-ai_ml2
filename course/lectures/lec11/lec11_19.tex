%!TEX root = lec11.tex
% ================================================================================
% Lecture 11 — Slide 19
% ================================================================================
\begin{frame}[t,fragile]
\begin{tightmath}

\mytitle{Native Function Calling in DAWID}

\begin{columns}[T,totalwidth=\textwidth]

% ------------------------------------------------------------
\begin{column}[T]{0.48\textwidth}

\textbf{What “Native” Means}

\begin{itemize}
  \item Functions declared via explicit schemas
  \item Typed arguments and clear signatures
  \item Direct support by modern LLM APIs
\end{itemize}

\vspace{2mm}
\textbf{Advantages in DAWID}

\begin{itemize}
  \item \y{No JSON parsing} from free text
  \item Fewer hallucinated calls
  \item Safer execution path
\end{itemize}

\end{column}

% ------------------------------------------------------------
\begin{column}[T]{0.48\textwidth}

\vspace{-5mm}
\textbf{DAWID Integration}

\begin{itemize}
  \item Functions registered centrally
  \item Only allowed tools are exposed
  \item Backend controls execution order
\end{itemize}

\vspace{2mm}
\textbf{Typical Use Cases}

\begin{itemize}
  \item Plot generation
  \item Data download and extraction
  \item Model-based diagnostics
\end{itemize}

\vspace{2mm}
\textbf{Design Principle}

\begin{itemize}
  \item Native calls are \y{preferred}
  \item Classical calls remain a fallback
\end{itemize}

\end{column}

\end{columns}

\end{tightmath}
\end{frame}
