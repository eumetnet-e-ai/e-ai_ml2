%!TEX root = lec11.tex
% ================================================================================
% Lecture 11 — Slide 15
% ================================================================================
\begin{frame}[t,fragile]
\begin{tightmath}

\mytitle{Available LLM Models and Capability Tiers}

\begin{columns}[T,totalwidth=\textwidth]

% ------------------------------------------------------------
\begin{column}[T]{0.52\textwidth}

\textbf{Capability Tiers}

\begin{itemize}
  \item \textbf{FAST}  
        Lightweight, fast, cost-efficient
  \item \textbf{CORE}  
        Strong general-purpose models
  \item \textbf{PRO}  
        Heavy reasoning, coding, long context
  \item \textbf{ULTRA}  
        Highest available model capability
\end{itemize}

\vspace{2mm}
\textbf{Design Choice}

\begin{itemize}
  \item One \y{best model} per tier and supplier
  \item No artificial or redundant model options
\end{itemize}

\end{column}

% ------------------------------------------------------------
\begin{column}[T]{0.44\textwidth}

\vspace{-5mm}
\textbf{Supported Model Families}

\begin{itemize}
  \item OpenAI {\tiny (Gpt5.2, GPT5.1, GPT-4o, GPT-5-mini)}
  \item Claude (Anthropic)
  \item Gemini (Google)
  \item LLaMA (local and remote)
  \item Mistral / Mixtral
  \item GPT-OSS
\end{itemize}

\vspace{2mm}
\textbf{Key Principle}

\begin{itemize}
  \item Users select \y{capability}, not internals
  \item Backend resolves tier to concrete model
\end{itemize}

\end{column}

\end{columns}

\end{tightmath}
\end{frame}
