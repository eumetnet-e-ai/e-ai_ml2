%!TEX root = lec11.tex
% ================================================================================
% Lecture 11 — Slide 23
% ================================================================================
\begin{frame}[t,fragile]
\begin{tightmath}

\mytitle{Input Data and Preprocessing}

\begin{columns}[T,totalwidth=\textwidth]

% ------------------------------------------------------------
\begin{column}[T]{0.48\textwidth}

\textbf{Meteorological Input Fields}

\begin{itemize}
  \item Mean sea-level pressure (\textbf{PMSL})
  \item 2\,m temperature (\textbf{T2M})
  \item 2\,m relative humidity (\textbf{RH2M})
  \item 10\,m wind components (\textbf{U10M}, \textbf{V10M})
  \item Land--sea mask (\textbf{FRLAND})
\end{itemize}

\vspace{2mm}
\textbf{Data Source}

\begin{itemize}
  \item ICON analysis and forecast fields
  \item Regular latitude--longitude grid
\end{itemize}

\end{column}

% ------------------------------------------------------------
\begin{column}[T]{0.5\textwidth}

\vspace{-8mm}
\textbf{Preprocessing Steps}

\begin{itemize}
  \item Channel-wise normalization
  \item Scaling to comparable numerical ranges
  \item Binary encoding of land--sea mask
\end{itemize}

\vspace{2mm}
\textbf{Resulting Tensor}

\begin{itemize}
  \item Shape: \y{[6, lat, lon]}
  \item Stored in NetCDF format
  \item One file per analysis or forecast time
\end{itemize}

\vspace{2mm}
\textbf{Design Principle}

\begin{itemize}
  \item Preserve physical meaning
  \item Avoid unnecessary feature engineering
\end{itemize}

\end{column}

\end{columns}

\end{tightmath}
\end{frame}
