%!TEX root = lec11.tex
% ================================================================================
% Lecture 11 — Slide 18
% ================================================================================
\begin{frame}[t,fragile]
\begin{tightmath}

\mytitle{Classical Function Calling in DAWID (JSON-Based)}

\begin{columns}[T,totalwidth=\textwidth]

% ------------------------------------------------------------
\begin{column}[T]{0.48\textwidth}

\textbf{What “Classical” Means in DAWID}

\begin{itemize}
  \item Function calls encoded explicitly in text
  \item Usually formatted as \y{JSON blocks}
  \item Independent of specific LLM providers
\end{itemize}

\vspace{2mm}
\textbf{Why DAWID Still Supports This}

\begin{itemize}
  \item Works with \y{all models} \\ (local and remote)
  \item Robust fallback mechanism
  \item Easy to debug and inspect
\end{itemize}

\end{column}

% ------------------------------------------------------------
\begin{column}[T]{0.52\textwidth}

\textbf{Example: Function Proposal by the LLM}

\begin{codeonly}{JSON-style function call}
{"function_calls": [{
      "name": "get_icon_forecast",
      "arguments": {
        "variable": "t2m",
        "region": "Germany" }}]}
\end{codeonly}

\vspace{2mm}
\textbf{DAWID Interpretation}

\begin{itemize}
  \item JSON is \y{parsed and validated}
  \item Function is checked against allow-list
  \item Execution only happens in backend
\end{itemize}

\end{column}

\end{columns}

\end{tightmath}
\end{frame}
