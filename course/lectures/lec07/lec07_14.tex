%!TEX root = lec07.tex
% ================================================================================
% Lecture 7 — Slide 14
% ================================================================================
\begin{frame}[t,fragile]

\mytitle{FAISS in Practice}

\begin{columns}[T,totalwidth=\textwidth]

% --- Left column ---------------------------------------------------------------
\begin{column}[T]{0.43\textwidth}

\textbf{Index construction}

\begin{itemize}
  \item Choose distance metric
  \item Add all embeddings
  \item Index lives in memory or on disk
\end{itemize} 
Common choice: IndexFlatL2

\vspace{4mm}
\textbf{Step 2: Query routing}

For a query vector \(z_q\), FAISS finds the closest centroids:
\[
\mathcal{C}_q
=
\operatorname*{arg\,top}_{L}
\; \| z_q - c_j \|_2 ,
\quad L \ll M .
\]

\end{column}

% --- Right column --------------------------------------------------------------
\begin{column}[T]{0.55\textwidth}

\vspace{-9mm}
\begin{codeonly}{Build and query FAISS index}
import faiss
d = page_embeddings.shape[1]
index = faiss.IndexFlatL2(d)

index.add(page_embeddings)

distances, indices = index.search(
    query_embedding, k)
\end{codeonly}


\vspace{3mm}
\textbf{Step 3: Local search}

Exact distances are computed only for vectors
stored in the selected regions:
\[
z_i \;\text{with}\; c(z_i) \in \mathcal{C}_q .
\]

\end{column}

\end{columns}

\end{frame}
