%!TEX root = lec07.tex
% ================================================================================
% Lecture 7 — Slide 20
% ================================================================================
\begin{frame}[t]
\begin{tightmath}

\mytitle{Top-\(k\) Retrieval: What Do We Actually Get?}

\begin{columns}[T,totalwidth=\textwidth]

% --- Left column ---------------------------------------------------------------
\begin{column}[T]{0.48\textwidth}

\textbf{FAISS output}

For a query embedding \(z_q\), FAISS returns:
\[
\{(i_1, d_1), (i_2, d_2), \dots, (i_k, d_k)\}
\]

\begin{itemize}
  \item \(i_j\): index of a stored chunk
  \item \(d_j\): distance or similarity score
\end{itemize}

\vspace{2mm}
These indices refer to:
\[
z_{i_j} \;\leftrightarrow\; \text{chunk}_{i_j}
\]

\end{column}

% --- Right column --------------------------------------------------------------
\begin{column}[T]{0.48\textwidth}

\textbf{Important clarification}

\begin{itemize}
  \item FAISS returns \emph{vectors}, not text
  \item Text is recovered via metadata lookup
  \item Ordering is by similarity score
\end{itemize}

\vspace{2mm}
\textbf{At this stage:}
\begin{itemize}
  \item \y{no LLM involved}
  \item \y{no generation}
  \item \y{no reasoning}
\end{itemize}

\end{column}

\end{columns}

\end{tightmath}
\end{frame}
