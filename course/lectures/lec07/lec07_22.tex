%!TEX root = lec07.tex
% ================================================================================
% Lecture 7 — Slide 22
% ================================================================================
\begin{frame}[t,fragile]

\mytitle{Composing the Retrieval Context}

\begin{columns}[T,totalwidth=\textwidth]

% --- Left column ---------------------------------------------------------------
\begin{column}[T]{0.48\textwidth}

\textbf{Context construction}

Retrieved chunks are:
\begin{itemize}
  \item ordered by similarity
  \item concatenated into one context block
\end{itemize}

\end{column}

% --- Right column --------------------------------------------------------------
\begin{column}[T]{0.48\textwidth}

Typical structure:
\begin{itemize}
  \item file reference
  \item optional score
  \item text snippet
\end{itemize}

\vspace{2mm}
Context size is \y{explicitly limited}.

\end{column}

\end{columns}

\vspace{0mm}
\begin{codeonly}{Context assembly (ICON example)}
context = ""
for r in retrieved_chunks:
  context += (
    f"[{r.file}, lines {r.start}-{r.end}]\n"
    + r.text + "\n\n" )
\end{codeonly}

\end{frame}
