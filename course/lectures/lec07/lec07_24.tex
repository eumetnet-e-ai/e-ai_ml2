%!TEX root = lec07.tex
% ================================================================================
% Lecture 7 — Slide XX
% ================================================================================
\begin{frame}[t]
\begin{tightmath}

\mytitle{\rtext{Google Search} as Retrieval-Augmented Generation}

\begin{columns}[T,totalwidth=\textwidth]

% --- Left column ---------------------------------------------------------------
\begin{column}[T]{0.48\textwidth}

\textbf{Core idea}

Retrieval-Augmented Generation (RAG)
combines an LLM with an
\y{external information source}.

\vspace{2mm}
In classical RAG, retrieval uses:
\begin{itemize}
  \item vector databases (FAISS)
  \item local document collections
\end{itemize}

\vspace{2mm}
\textbf{Google-based RAG}

Here, retrieval is delegated to:
\begin{itemize}
  \item Google Search
  \item live web documents
\end{itemize}

\end{column}

% --- Right column --------------------------------------------------------------
\begin{column}[T]{0.48\textwidth}

\textbf{Same logical pipeline}

\vspace{-4mm}
\begin{eqnarray*}
&& \text{Query}
\;\rightarrow\;
\text{Retrieve}
\;\rightarrow\; \\
&& \text{Context}
\;\rightarrow\;
\text{LLM}
\;\rightarrow\;
\text{Answer}
\end{eqnarray*}

\vspace{2mm}
\textbf{Key difference}

\begin{itemize}
  \item Local RAG: curated, controlled, static
  \item Google RAG: open, dynamic, up-to-date
\end{itemize}

\vspace{2mm}
\textbf{Interpretation}

The LLM does \emph{not} search —  
\y{it \emph{summarizes retrieved evidence}.}

\end{column}

\end{columns}

\end{tightmath}
\end{frame}
