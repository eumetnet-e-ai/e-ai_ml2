%!TEX root = lec07.tex
% ================================================================================
% Lecture 7 — Slide 02
% ================================================================================
\begin{frame}[t]

\mytitle{Core Idea of Retrieval-Augmented Generation}

\begin{columns}[T,totalwidth=\textwidth]

% --- Left column ---------------------------------------------------------------
\begin{column}[T]{0.48\textwidth}

\textbf{Standard LLM workflow}

\[
\text{Prompt} \;\rightarrow\; \text{LLM} \;\rightarrow\; \text{Answer}
\]

\begin{itemize}
  \item No external knowledge
  \item No verification
\end{itemize}

\vspace{4mm}
\begin{minipage}{7cm}
\fontsize{9pt}{10pt}\selectfont
Pure LLMs work well for general reasoning, language understanding, summarization, and creative text generation when the required knowledge is common and static.
They perform poorly when accurate, up-to-date, proprietary, or highly technical domain knowledge is required, because they cannot verify facts or access external sources.
In such cases, LLMs tend to hallucinate plausible but incorrect answers.
\end{minipage}




\end{column}

% --- Right column --------------------------------------------------------------
\begin{column}[T]{0.48\textwidth}

\textbf{RAG workflow}

{\color{red}\bf\[
\text{Query}
\;\rightarrow\;
\text{Retrieve}
\;\rightarrow\;
\text{LLM}
\;\rightarrow\;
\text{Answer}
\]}

\begin{itemize}
  \item Search relevant documents
  \item Inject context into prompt
  \item \y{Grounded answers}
\end{itemize}

\end{column}

\end{columns}

\end{frame}
