%!TEX root = lec07.tex
% ================================================================================
% Lecture 7 — Slide 16
% ================================================================================
\begin{frame}[t,fragile]

\mytitle{OpenAI Streaming}

\begin{columns}[T,totalwidth=\textwidth]

% --- Left column ---------------------------------------------------------------
\begin{column}[T]{0.35\textwidth}

\textbf{Streaming API call}

\begin{itemize}
  \item Same prompt structure
  \item \texttt{stream=True}
  \item Tokens arrive as \y{deltas}
\end{itemize}

\vspace{2mm}
Each chunk contains new text:
\[
\delta_k \subset \text{response}
\]

\begin{minipage}{4cm}
\tiny\color{red}
There are several older OpenAI interface versions that language models may still suggest.
Be careful: to avoid deprecated APIs, it is often necessary to explicitly provide a current code template.
\end{minipage}

\end{column}

% --- Right column --------------------------------------------------------------
\begin{column}[T]{0.65\textwidth}

\vspace{-9mm}
\begin{codeonly}{OpenAI streaming in Jupyter}
stream = client.chat.completions.create(
  model="gpt-4o-mini", messages=[...],
  stream=True )

accumulated = ""
handle = display(
  Markdown(""), display_id=True)

for chunk in stream:
    delta = chunk.choices[0].delta
    if delta.content:
        accumulated += delta.content
        handle.update(
          Markdown(accumulated))
\end{codeonly}

\end{column}

\end{columns}

\end{frame}
