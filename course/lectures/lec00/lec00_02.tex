% ================================================================================
% Lecture 00 — Slide 02
% ================================================================================

\begin{frame}[t,fragile]
\begin{tightmath}

\mytitle{AI as Tool of Discovery: From Ideas to Tested Systems}

\begin{columns}[T,totalwidth=\textwidth]

% ------------------------------------------------------------
\begin{column}[T]{0.48\textwidth}
\footnotesize
\vspace{-2mm}

\textbf{Discovery viewpoint}

\vspace{1mm}
AI is not only about producing forecasts --- it increasingly supports
\y{scientific discovery} and \y{systematic development}.

\vspace{2mm}
\textbf{What AI adds to science and engineering}

\begin{itemize}
  \item accelerate \y{coding and refactoring} of research and operational software
  \item help to \y{formulate} \rtext{equations, losses, and constraints} in a consistent way
  \item generate \y{tests}, validation scripts, and diagnostic plots automatically
  \item \y{explore hypotheses}: \rtext{patterns, regimes, causal links} in large datasets
\end{itemize}


\end{column}

% ------------------------------------------------------------
\begin{column}[T]{0.48\textwidth}
\footnotesize

\vspace{2mm}
\textbf{Result}

\vspace{1mm}
We move from isolated scripts to \y{AI-supported workflows}
that iterate fast: idea $\rightarrow$ prototype $\rightarrow$ test $\rightarrow$ deploy.

\vspace{5mm}
\textbf{Key takeaway}

\vspace{1mm}
\begin{quote}
\textbf{AI becomes part of the scientific method:}
hypothesize, implement, test, improve.
\end{quote}

\includegraphics[height=2cm]{../../images/img16/physics_and_ai.png}

\end{column}

\end{columns}

\end{tightmath}
\end{frame}
