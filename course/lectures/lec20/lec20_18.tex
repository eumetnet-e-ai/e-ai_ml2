%!TEX root = lec20.tex
% ================================================================================
% Lecture 20 — Slide 18
% ================================================================================
\begin{frame}[t,fragile]
\mytitle{Observations: Partial Measurements of $x_1$ or $x_2$}

\begin{columns}[T,totalwidth=\textwidth]

% ------------------------------------------------------------
\begin{column}[T]{0.45\textwidth}
\footnotesize
\vspace{-2mm}

\textbf{Time-dependent observation operator}
\vspace{-1mm}
\begin{itemize}
  \item At each step $k$ we observe only \y{one component}
  \item Selector:
  \[
    s_k\in\{1,2\}
  \]
  \item Observation operator:
  \[
    H_k =
    \begin{cases}
      [1,0] & s_k=1 \ (\text{observe }x_1)\\
      [0,1] & s_k=2 \ (\text{observe }x_2)
    \end{cases}
  \]
  \item Observation equation:
  \[
    y_k = H_k x_k + \epsilon_k,
    \qquad
    \epsilon_k\sim\mathcal N(0,R)
  \]
\end{itemize}

\vspace{0mm}
\rtext{\bf Interpretation:}
This mimics \y{heterogeneous sensors} and \y{missing data}.

\end{column}

% ------------------------------------------------------------
\begin{column}[T]{0.58\textwidth}
\footnotesize
\vspace{-2mm}

\begin{center}
\includegraphics[width=0.88\textwidth]{../../images/img20/origen/origen02.png}
\end{center}

\vspace{-1mm}
\begin{center}
{\scriptsize Blue: $x_1$ observed \qquad Orange: $x_2$ observed}
\end{center}

\end{column}

\end{columns}
\end{frame}
