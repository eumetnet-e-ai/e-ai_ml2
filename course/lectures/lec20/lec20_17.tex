%!TEX root = lec20.tex
% ================================================================================
% Lecture 20 — Slide 17
% ================================================================================
\begin{frame}[t,fragile]
\mytitle{ORIGEN on a Minimal Example: Simple Oscillator}

\begin{columns}[T,totalwidth=\textwidth]

% ------------------------------------------------------------
\begin{column}[T]{0.4\textwidth}
\footnotesize
\vspace{-2mm}

\textbf{Minimal testbed (2D circle)}
\begin{itemize}
  \item \y{Truth trajectory} $x_k=(x_{1,k},x_{2,k})$ on a circle
  \[
  x_k=
  \begin{bmatrix}
  \cos\theta_k\\ \sin\theta_k
  \end{bmatrix},
  \qquad
  \theta_k=\frac{2\pi k}{n}
  \]
  \item Observations are \y{scalar} and \y{partial}:
  \[
  y_k = H_k x_k + \epsilon_k,
  \qquad
  H_k\in\{[1,0],[0,1]\}
  \]
\end{itemize}

\vspace{1mm}
\rtext{\bf Aim:}
Explain ORIGEN mechanics without complex dynamics.

\end{column}

% ------------------------------------------------------------
\begin{column}[T]{0.58\textwidth}
\footnotesize
\vspace{-2mm}
\begin{center}
\includegraphics[width=0.8\textwidth]{../../images/img20/origen/origen01.png}
\end{center}

\end{column}

\end{columns}
\end{frame}
