%!TEX root = lec20.tex
% ================================================================================
% Lecture 20 — Slide 21
% ================================================================================
\begin{frame}[t,fragile]
\mytitle{Sequential Rounds: With All Observations We Converge in Two Cycles}

\begin{columns}[T,totalwidth=\textwidth]

% ------------------------------------------------------------
\begin{column}[T]{0.44\textwidth}
\footnotesize
\vspace{-2mm}

\textbf{Idea: assimilate in two rounds}
\vspace{-1mm}
\begin{itemize}
  \item \y{Round 1:} observe $x_1$ only
  
  \vspace{-3mm}
  \[
    y_k^{(1)} = x_{1,k} + \epsilon_k
  \]
  \item \y{Round 2:} observe $x_2$ only, using

  \vspace{-3mm}
  \[
    x_{k}^{b,(2)} := x_{k}^{a,(1)}
  \]
\end{itemize}

\vspace{1mm}
\textbf{If obs error is small}
\vspace{-1mm}
\begin{itemize}
  \item after Round 1: $x_1$ aligns with truth
  \item after Round 2: $x_2$ aligns with truth
  \item \y{both components observed} $\Rightarrow$ fast convergence
\end{itemize}

\vspace{2mm}
\rtext{\bf Message:}
With complete information, ORIGEN-style cycling converges quickly.

\end{column}

% ------------------------------------------------------------
\begin{column}[T]{0.56\textwidth}
\footnotesize
\vspace{-2mm}

\begin{center}
\includegraphics[width=0.92\textwidth]{../../images/img20/origen/origen05.png}
\end{center}

\vspace{-1mm}
\begin{center}
{\scriptsize Truth vs Round 1 ($x_1$ obs) vs Round 2 ($x_2$ obs, bg=Round 1)}
\end{center}

\end{column}

\end{columns}
\end{frame}
