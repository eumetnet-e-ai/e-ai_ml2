%!TEX root = lec20.tex
% ================================================================================
% Lecture 20 — Slide 11
% ================================================================================
\begin{frame}[t, fragile]
\mytitle{Flexible RS Input: Two Dataset Modes (A/B Switch)}

\begin{columns}[T,totalwidth=\textwidth]

% ------------------------------------------------------------
\begin{column}[T]{0.7\textwidth}
\footnotesize
\vspace{4mm}

We implement a dataset switch (important for interpretability):

\vspace{2mm}
\textbf{Mode A (synthetic RS from truth)}
\begin{itemize}
  \item sample RS input points randomly from $xtrue[t]$
  \item add realistic noise $\sigma_{rs}$
  \item \y{excellent generalization} for arbitrary RS geometry
\end{itemize}

\vspace{1mm}
\textbf{Mode B (use only stored RS observations)}
\begin{itemize}
  \item RS input points are exactly \texttt{yrs[t,:,:]} at fixed \texttt{rs\_ix, rs\_iz}
  \item strict statement: \rtext{\bf “input uses only observations at time $t$”}
  \item weaker geometric diversity, but realistic RS network
\end{itemize}

\vspace{2mm}
\rtext{\bf Reality:} We have only fixed radiosondes, but we have \y{airplanes}!

\end{column}

% ------------------------------------------------------------
\begin{column}[T]{0.28\textwidth}
\vspace{4mm}
\begin{lstlisting}[basicstyle=\ttfamily\tiny]
Saved:
  data_dyn_obs/dyn_truth_obs.npz

Content:
  xtrue  shape=(10,50,130)
  yrs    shape=(10,7,18)
  ysat   shape=(10,4,130)

  t_snap shape=(10,)
  rs_ix  shape=(7,)
  rs_iz  shape=(18,)
  sat_w  shape=(4,50)

  x      shape=(130,)
  z      shape=(50,)
\end{lstlisting}
\end{column}

\end{columns}

\end{frame}
