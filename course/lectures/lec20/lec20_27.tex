%!TEX root = lec20.tex
% ================================================================================
% Lecture 20 — Slide 27
% ================================================================================
\begin{frame}[t]
\mytitle{Lorenz-63: ORIGEN Reconstruction \& Learned Forecast Model}

\begin{columns}[T,totalwidth=\textwidth]

% ------------------------------------------------------------
\begin{column}[T]{0.50\textwidth}
\footnotesize
\vspace{-1mm}
\textbf{ORIGEN loop (concept)}
\begin{enumerate}
  \item Assimilate random observed subsets ($x$, $y$, $z$, $xy$, $xz$, $yz$)
  \item Update background $\mathbf{x}^b \leftarrow \mathbf{x}^a$
  \item Update covariance $B \leftarrow P^a$ (Kalman-style)
\end{enumerate}

\vspace{1mm}
\textbf{Model learning}
\begin{itemize}
  \item Train NN to learn one-step map
  \[
    \mathbf{x}^a(k)\ \mapsto\ \mathbf{x}^a(k+1)
  \]
  \item \y{\rtext{\bf Fine-Tuning with Rollout}}
\end{itemize}

Rollout: many subsequent short forecasts

\end{column}

% ------------------------------------------------------------
\begin{column}[T]{0.50\textwidth}
\vspace{-2mm}
\centering
\includegraphics[width=0.9\textwidth]{../../images/img20/L63_6_ML_model_fc_test_2.png}

\vspace{1mm}
\scriptsize
Forecast rollouts (blue) starting from analysis states (red dots), compared to the
reconstructed reference trajectory (black dashed).
\end{column}

\end{columns}
\end{frame}
