%!TEX root = lec20.tex
% ================================================================================
% Lecture 20 — Slide 22
% ================================================================================
\begin{frame}[t,fragile]
\mytitle{Iterative 3D-Var: Reconstruction Improves over Cycles (Demo)}

\begin{columns}[T,totalwidth=\textwidth]

% ------------------------------------------------------------
\begin{column}[T]{0.48\textwidth}
\footnotesize
\vspace{-2mm}

\textbf{What is iterated here?}
\vspace{-1mm}
\begin{itemize}
  \item We reconstruct an \y{entire trajectory}
  
  \vspace{-3mm}
  \[
    \{x_k^a\}_{k=0}^{n-1}
  \]
  from \y{noisy, partial observations}

  \vspace{-3mm}
  \[
    y_k = H_k x_k + \epsilon_k
  \]
  \item Each cycle produces a refined estimate of the full sequence:

  \vspace{-3mm}
  \[
    \{x_k^{a,(c)}\} \;\Rightarrow\; \{x_k^{a,(c+1)}\}
  \]
\end{itemize}

\vspace{-2mm}
\textbf{Crucial point}
\vspace{-1mm}
\begin{itemize}
  \item The ``dynamics'' is \y{implicit} in the reconstructed series
  \item We are not propagating with a separate model;
        we repeatedly improve the \y{sequence itself}
\end{itemize}

\end{column}

% ------------------------------------------------------------
\begin{column}[T]{0.54\textwidth}
\footnotesize
\vspace{-5mm}

\begin{center}
\includegraphics[width=0.96\textwidth]{../../images/img20/origen/origen08_crop.png}
\end{center}

\vspace{-3mm}
\begin{center}
{\scriptsize RMSE per assimilation cycle: iterative 3D-Var improves the reconstructed trajectory}
\end{center}

\vspace{-2mm}
\rtext{\bf ORIGEN:}
from the reconstructed $\{x_k^a\}$ we learn a forecast map

\vspace{-3mm}
\[
x^a(t)\mapsto x^a(t+\Delta t),
\]
in the notebook we focus on trajectory reconstruction.

\end{column}

\end{columns}
\end{frame}
