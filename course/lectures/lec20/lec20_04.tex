%!TEX root = lec20.tex
% ================================================================================
% Lecture 20 — Slide 04
% ================================================================================
\begin{frame}[t,fragile]
\mytitle{RK4 Implementation (Core Loop)}

\footnotesize
\vspace{-1mm}

\textbf{Right-hand side}
\[
\mathrm{RHS}(\phi,t)
=
-u\,\partial_x\phi
-w\,\partial_z\phi
+K\nabla^2\phi
+A(t)S(x,z)
-(\lambda+\lambda_{\text{top}})\phi.
\]

\textbf{One RK4 step (schematic)}
\begin{lstlisting}[basicstyle=\ttfamily\scriptsize]
k1 = rhs(phi, t)
k2 = rhs(phi + 0.5*dt*k1, t + 0.5*dt)
k3 = rhs(phi + 0.5*dt*k2, t + 0.5*dt)
k4 = rhs(phi + dt*k3,     t + dt)
phi_next = phi + (dt/6)*(k1 + 2*k2 + 2*k3 + k4)
\end{lstlisting}

\vspace{1mm}
\textbf{Snapshot logic}
\begin{itemize}
  \item Choose $n_{vis}$ time indices between $0$ and $nsteps$
  \item Save numbered PNGs: \texttt{1\_dyn\_XX.png}
\end{itemize}

\vspace{4mm}
\rtext{\bf This gives clean training snapshots + nice lecture figures.}

\end{frame}
