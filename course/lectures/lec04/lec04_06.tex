%!TEX root = lec04.tex
% ================================================================================
% Lecture 4 — Slide 06
% ================================================================================
\begin{frame}[t,fragile]

\mytitle{Torch Tensors — The Core Data Structure}

\begin{columns}[T,totalwidth=\textwidth]

% --- Left column ---------------------------------------------------------------
\begin{column}[T]{0.48\textwidth}

\textbf{What is a tensor?}

\begin{itemize}
  \item Similar to NumPy arrays
  \item Supports CPU and GPU
  \item Tracks operations for gradients
  \item Basis of all learning
\end{itemize}

\vspace{1mm}
\textbf{Key properties}

\begin{itemize}
  \item Shape and dtype
  \item Device awareness
  \item \texttt{requires\_grad=True}
\end{itemize}

\end{column}

% --- Right column --------------------------------------------------------------
\begin{column}[T]{0.48\textwidth}

\vspace{-2mm}
\begin{codeonly}{Basic tensor example}
import torch

x = torch.tensor([2.,3.], requires_grad=True)

y = x[0]**2 + x[1]**2
y.backward()

print(x.grad)
\end{codeonly}

\vspace{1mm}
\[
\nabla_x (x_1^2 + x_2^2)
\]

\end{column}

\end{columns}

\end{frame}
