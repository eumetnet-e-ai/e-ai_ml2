%!TEX root = lec04.tex
% ================================================================================
% Lecture 4 — Slide 10
% ================================================================================
\begin{frame}[t,fragile]

\mytitle{Defining a Simple Neural Network}

\begin{columns}[T,totalwidth=\textwidth]

% --- Left column ---------------------------------------------------------------
\begin{column}[T]{0.4\textwidth}

\textbf{Neural network idea}

\begin{itemize}
  \item Learn a mapping $f_\theta : x \rightarrow \hat{y}$
  \item \y{Parameters} $\theta$ are trainable
  \item Composition of simple operations
\end{itemize}

\vspace{1mm}
\textbf{Basic building blocks}

\begin{itemize}
  \item \y{Linear transformation}
  \item \y{Nonlinear activation}
  \item Output layer
\end{itemize}

\vspace{-2mm}
\[
\hat{y} = f_\theta(x)
\]

\end{column}

% --- Right column --------------------------------------------------------------
\begin{column}[T]{0.58\textwidth}

\vspace{-5mm}
\begin{codeonly}{Minimal PyTorch model}
import torch.nn as nn

class SimpleNN(nn.Module):
    def __init__(self):
        super().__init__()
        self.fc1 = nn.Linear(1,16)
        self.relu = nn.ReLU()
        self.fc2 = nn.Linear(16,1)

    def forward(self,x):
        x = self.fc1(x)
        x = self.relu(x)
        return self.fc2(x)
\end{codeonly}

\end{column}

\end{columns}

\end{frame}
