%!TEX root = lec04.tex
% ================================================================================
% Lecture 4 — Slide 08
% ================================================================================
\begin{frame}[t,fragile]

\mytitle{Data Handling in PyTorch: Dataset and DataLoader}

\begin{columns}[T,totalwidth=\textwidth]

% --- Left column ---------------------------------------------------------------
\begin{column}[T]{0.48\textwidth}

\textbf{Why data loaders exist}

\begin{itemize}
  \item Datasets often too large for memory
  \item Training uses \y{mini-batches}
  \item Data order matters for optimization
  \item Separation of data and model logic
\end{itemize}

\vspace{1mm}
\[
(x_i, y_i) \;\rightarrow\; (X_B, Y_B)
\]
Samples are grouped into batches.

\end{column}

% --- Right column --------------------------------------------------------------
\begin{column}[T]{0.48\textwidth}

\textbf{What DataLoader provides}

\begin{itemize}
  \item Batching
  \item Optional shuffling
  \item Parallel loading (CPU workers)
  \item Consistent interface for training loops
\end{itemize}

\vspace{1mm}
\[
(X_B, Y_B) \;\rightarrow\; \mathcal{L}(f_\theta(X_B), Y_B)
\]
Each batch produces one loss value.

\end{column}

\end{columns}

\end{frame}
