%!TEX root = lec04.tex
% ================================================================================
% Lecture 4 — Slide 09
% ================================================================================
\begin{frame}[t,fragile]

\mytitle{\y{Batches} Explained: What Comes Out of the DataLoader}

\begin{columns}[T,totalwidth=\textwidth]

% --- Left column ---------------------------------------------------------------
\begin{column}[T]{0.48\textwidth}

\textbf{Features and labels}

\begin{itemize}
  \item \y{Features} $x$: input quantities
  \item \y{Labels} $y$: measured target values
  \item Learning means fitting $x \rightarrow y$
\end{itemize}

\vspace{1mm}
\textbf{Structure of the data}

\begin{itemize}
  \item $N$ = number of samples
  \item $d$ = number of features per sample
  \item One row = one $(x,y)$ pair
\end{itemize}

\vspace{1mm}
\[
X \in \mathbb{R}^{N \times d}
\;\Rightarrow\;
X_B \in \mathbb{R}^{B \times d}
\]

\end{column}

% --- Right column --------------------------------------------------------------
\begin{column}[T]{0.48\textwidth}

\textbf{Labels and targets}

\begin{itemize}
  \item Labels collected in $Y$
  \item One target per input sample
  \item Same batching as for features
\end{itemize}
\[
Y \in \mathbb{R}^{N \times k}
\;\Rightarrow\;
Y_B \in \mathbb{R}^{B \times k}
\]

\vspace{0mm}
\textbf{Why mini-batches help}

\begin{itemize}
  \item Memory-efficient processing
  \item Faster parameter updates
  \item Noise improves generalization
\end{itemize}

\vspace{-1mm}
\[
\nabla_\theta \mathcal{L}(X_B, Y_B)
\;\approx\;
\nabla_\theta \mathcal{L}(X, Y)
\]

\end{column}

\end{columns}

\end{frame}
