%!TEX root = lec18.tex
% ================================================================================
% Lecture 18 — Slide 01
% ================================================================================
\begin{frame}[t]

\mytitle{AI Data Assimilation — Why Does It Matter?}

\begin{columns}[T,totalwidth=\textwidth]

% ------------------------------------------------------------
\begin{column}[T]{0.52\textwidth}
\footnotesize

\textbf{Numerical Weather Prediction (NWP)}

\vspace{1mm}
Weather forecasts are produced by integrating
\y{high-dimensional dynamical models} forward in time.

\vspace{0mm}
However:

\begin{itemize}
  \item The atmosphere is \y{chaotic}
  \item Small initial errors grow rapidly
  \item Forecast skill is \rtext{\bf dominated by initial conditions}
\end{itemize}

\vspace{0mm}
\textbf{Core problem}

\vspace{1mm}
At any analysis time, we have:
\begin{itemize}
  \item an \y{imperfect model forecast} (background)
  \item \y{sparse, noisy observations}
\end{itemize}

\vspace{1mm}
These must be combined optimally.

\end{column}

% ------------------------------------------------------------
\begin{column}[T]{0.46\textwidth}
\footnotesize

\textbf{Data assimilation}

\vspace{1mm}
Data assimilation provides a \y{statistically consistent framework}
to merge:
\[
\text{model information}
\quad + \quad
\text{observations}
\]

\vspace{2mm}
The result is the \y{analysis state}:
\begin{itemize}
  \item best estimate of the atmospheric state
  \item starting point for forecasts
\end{itemize}

\vspace{2mm}
\rtext{\bf Data assimilation is the information bottleneck of NWP.}

\vspace{1mm}
Everything that follows — forecasts, warnings, applications —
depends on it.

\end{column}

\end{columns}

\end{frame}
