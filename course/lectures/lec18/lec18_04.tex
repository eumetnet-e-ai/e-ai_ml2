%!TEX root = lec18.tex
% ================================================================================
% Lecture 18 — Slide 04
% ================================================================================
\begin{frame}[t]

\mytitle{Why Bring AI into Data Assimilation?}

\begin{columns}[T,totalwidth=\textwidth]

% ------------------------------------------------------------
\begin{column}[T]{0.54\textwidth}
\footnotesize

\textbf{Limits of classical DA}

\vspace{1mm}
State-of-the-art data assimilation systems are:

\begin{itemize}
  \item \y{computationally expensive}
  \item difficult to scale to higher resolution
  \item reliant on \y{adjoint models}
\end{itemize}

\vspace{2mm}
Operational challenges include:

\begin{itemize}
  \item complex model development and maintenance
  \item long wall-clock times
  \item limited flexibility for new observation types
\end{itemize}

\vspace{2mm}
\hspace*{1cm}\begin{minipage}{5cm}
\rtext{\bf DA is often the most expensive component of NWP.}
\end{minipage}

\end{column}

% ------------------------------------------------------------
\begin{column}[T]{0.48\textwidth}
\footnotesize

\vspace{-2mm}
\textbf{What AI can offer}

\vspace{1mm}
Modern neural networks provide:

\begin{itemize}
  \item fast inference once trained
  \item automatic differentiation
  \item flexible nonlinear mappings
\end{itemize}

\vspace{2mm}
Potential benefits for DA:

\begin{itemize}
  \item orders-of-magnitude speedup
  \item end-to-end differentiability
  \item easier adaptation to new data streams
\end{itemize}

\vspace{2mm}
\rtext{\bf Goal: replace the solver, not the statistics.}

\end{column}

\end{columns}

\end{frame}
