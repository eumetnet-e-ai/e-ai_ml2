%!TEX root = lec18.tex
% ================================================================================
% Lecture 18 — Slide 19  (tight version)
% ================================================================================
\begin{frame}[t]
\begin{tightmath}

\mytitle{AI Particle Filter (Gaussian Mixture): Core Idea}

\begin{columns}[T,totalwidth=\textwidth]

% ------------------------------------------------------------
\begin{column}[T]{0.54\textwidth}
\footnotesize

\textbf{Filtering problem}

\vspace{0mm}
Sequential posterior:

\vspace{-3mm}
\[
p(x_n \mid y_{1:n})
\]

\vspace{-1mm}
Represent the distribution by \y{particles}

\vspace{-2mm}
\[
X_n = \{x^{(i)}_n\}_{i=1}^N
\]

\vspace{1mm}
\textbf{AI Particle Filter (this tutorial)}

\vspace{0mm}
Forecast ensemble $X^b_n$ is transformed into analysis ensemble:
\[
X^a_n = \mathcal{N}_\theta(X^b_n,\; y_n)
\]

\vspace{1mm}
\textbf{Gaussian mixture view}

\vspace{0mm}
Both prior and posterior are approximated by mixtures:
\[
q^b(x)\approx \frac1N\sum_i \mathcal{N}(x|x^{b,(i)},\Sigma),
\qquad
q^a(x)\approx \frac1N\sum_i \mathcal{N}(x|x^{a,(i)},\Sigma)
\]

\end{column}

% ------------------------------------------------------------
\begin{column}[T]{0.42\textwidth}
\footnotesize

\textbf{Model problem: Lorenz-63}

\vspace{0mm}
\begin{itemize}
  \item state $x=(x_1,x_2,x_3)\in\mathbb{R}^3$
  \item partial observations $y=(x_1,x_2)$ + noise
  \item wrong forecast model (intentional)
\end{itemize}

\vspace{1mm}
\textbf{Key message}

\vspace{0mm}
Instead of resampling / MCMC moves, we learn a
\y{distribution transform} that fits the ensemble to the posterior.

\vspace{4mm}
\rtext{\bf Network output is a posterior particle cloud.}

\end{column}

\end{columns}

\end{tightmath}
\end{frame}
