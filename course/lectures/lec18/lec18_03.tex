%!TEX root = lec18.tex
% ================================================================================
% Lecture 18 — Slide 03
% ================================================================================
\begin{frame}[t]

\mytitle{Two AI Paths for Using Observations}

\begin{columns}[T,totalwidth=\textwidth]

% ------------------------------------------------------------
\begin{column}[T]{0.48\textwidth}
\footnotesize

\textbf{Path 1: AI-based forecasting}

\vspace{1mm}
Observations are used \y{directly inside neural networks}
that produce forecasts.

\vspace{1mm}
Typical characteristics:

\begin{itemize}
  \item observations as additional inputs
  \item sometimes \y{no explicit model state}
  \item learning focuses on \rtext{\bf prediction skill}
\end{itemize}

\vspace{2mm}
This approach bypasses the classical analysis concept.

\vspace{1mm}
\rtext{\bf Observations $\rightarrow$ forecast directly}

\end{column}

% ------------------------------------------------------------
\begin{column}[T]{0.48\textwidth}
\footnotesize

\vspace{-2mm}
\textbf{Path 2: AI-based data assimilation}

\vspace{1mm}
Observations are used to compute an \y{analysis state},
not a forecast.

\vspace{1mm}
Key properties:

\begin{itemize}
  \item preserves the \y{analysis–forecast separation}
  \item consistent with Bayesian DA theory
  \item AI replaces the \y{analysis algorithm}, not the model
\end{itemize}

\vspace{2mm}
\rtext{\bf Observations $\rightarrow$ analysis $\rightarrow$ forecast}

\vspace{1mm}
This is the conceptual space of \y{\bf AI-Var}.

\end{column}

\end{columns}

\end{frame}
