%!TEX root = lec18.tex
% ================================================================================
% Lecture 18 — Slide 17  (reduced)
% ================================================================================
\begin{frame}[t, fragile]
\begin{tightmath}

\mytitle{2D AI-Var (Tutorial Code): Flattened Control + Obs Encoding + 3D-Var Loss}

\begin{columns}[T,totalwidth=\textwidth]

% ------------------------------------------------------------
\begin{column}[T]{0.52\textwidth}
\footnotesize

\textbf{Inputs: background + obs on grid + mask}

\vspace{0mm}
{\tiny
\begin{verbatim}
# flatten 2D -> 1D control
nz, nx = xb.shape
n = nz * nx
xb_t = torch.tensor(xb.reshape(-1), dtype=dtype, device=device)

# obs indices (iz-major, ix-minor)
obs_indices = iz_idx * nx + ix_idx
y_vec = y_field.reshape(-1)[obs_indices]
y_t   = torch.tensor(y_vec, dtype=dtype, device=device)

# obs on grid + mask (length n)
y_grid = np.zeros(n);  mask = np.zeros(n)
y_grid[obs_indices] = y_vec
mask[obs_indices]   = 1.0

inp_t = torch.cat([xb_t,
       torch.tensor(y_grid, dtype=dtype, device=device),
       torch.tensor(mask,   dtype=dtype, device=device)], dim=0)
\end{verbatim}
}

\vspace{-1mm}
\rtext{\bf Input is $[x_b,\; y_{\rm grid},\; {\rm mask}]$ in control space.}

\end{column}

% ------------------------------------------------------------
\begin{column}[T]{0.46\textwidth}
\footnotesize

\textbf{Increment MLP + variational loss}

\vspace{0mm}
{\tiny
\begin{verbatim}
class IncrementMLP(nn.Module):
    def __init__(self, n):
        super().__init__()
        self.net = nn.Sequential(
            nn.Linear(3*n, 256), nn.Tanh(),
            nn.Linear(256, 256), nn.Tanh(),
            nn.Linear(256, n),
        )

def quadform(A, v): return torch.dot(v, A @ v)

def J_3dvar(dx):
    x = xb_t + dx
    innov = y_t - (H_t @ x)
    Jb = 0.5 * quadform(B_inv_t, dx)
    Jo = 0.5 * quadform(R_inv_t, innov)
    return Jb + Jo
\end{verbatim}
}

\vspace{-1mm}
\y{Exactly the classical 3D-Var objective,} \\
\y{now differentiable.}

\end{column}

\end{columns}

\end{tightmath}
\end{frame}
