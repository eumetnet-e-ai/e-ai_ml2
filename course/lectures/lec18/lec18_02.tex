%!TEX root = lec18.tex
% ================================================================================
% Lecture 18 — Slide 02
% ================================================================================
\begin{frame}[t]

\mytitle{Classical Data Assimilation: The Analysis Cycle}

\begin{columns}[T,totalwidth=\textwidth]

% ------------------------------------------------------------
\begin{column}[T]{0.54\textwidth}
\footnotesize

\textbf{The classical DA cycle}

\vspace{1mm}
Operational NWP systems run a \y{repeating assimilation cycle}:

\begin{enumerate}
  \item Start from a \y{background state} $x_b$
  \item Assimilate observations $\;y$
  \item Compute an \y{analysis} $x_a$
  \item Run the numerical model forward
  \item Use the forecast as next background
\end{enumerate}

\vspace{2mm}
This cycle is repeated every few hours as new
observations become available.

\vspace{2mm}
\rtext{\bf The analysis step is the only place where observations enter.}

\end{column}

% ------------------------------------------------------------
\begin{column}[T]{0.48\textwidth}
\footnotesize

\vspace{-2mm}
\textbf{Established DA methods}

\vspace{1mm}
Several algorithmic families are used in practice:

\begin{itemize}
  \item \y{\bf 3D-Var / 4D-Var}
  \begin{itemize}
    \item variational optimization
    \item adjoint-based
  \end{itemize}

  \vspace{0mm}
  \item \y{\bf Ensemble Kalman Filters (EnKF)}
  \begin{itemize}
    \item flow-dependent uncertainty
    \item ensemble statistics
  \end{itemize}

  \vspace{0mm}
  \item \y{\bf Particle Filters}
  \begin{itemize}
    \item fully Bayesian
    \item now also high-dimensional!
  \end{itemize}
\end{itemize}

\vspace{0mm}
All methods aim at the \y{same goal}:
a statistically optimal analysis.

\end{column}

\end{columns}

\end{frame}
