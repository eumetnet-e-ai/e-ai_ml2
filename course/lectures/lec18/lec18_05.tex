%!TEX root = lec18.tex
% ================================================================================
% Lecture 18 — Slide 05
% ================================================================================
\begin{frame}[t]

\mytitle{From Variational Data Assimilation to AI-Var}

\begin{columns}[T,totalwidth=\textwidth]

% ------------------------------------------------------------
\begin{column}[T]{0.48\textwidth}
\footnotesize

\textbf{Variational DA: main Idea}

\vspace{3mm}
The analysis is obtained by minimizing:
\[
J(x)
=
\frac{1}{2}(x-x_b)^T B^{-1}(x-x_b)
+
\frac{1}{2}(y-H(x))^T R^{-1}(y-H(x))
\]

\vspace{2mm}
\textbf{Interpretation}

\begin{itemize}
  \item first term: \y{background constraint}
  \item second term: \y{observation constraint}
\end{itemize}

\vspace{2mm}
\rtext{\bf Classical DA = iterative \\numerical minimization.}

\end{column}

% ------------------------------------------------------------
\begin{column}[T]{0.48\textwidth}

\footnotesize
\centering
AI-VAR is the \y{AI verion} of \rtext{\bf 3D-Var, 4D-Var or En-VAR} depending on 
how exactly the minimizer is formulated. 

\vspace{3mm}
\hspace*{-1.3cm}
\begin{minipage}{9cm}
\includegraphics[width=\textwidth]{../../images/img18/aivar2.png}
\end{minipage}

\vspace{1mm}
\footnotesize
AI-Var sits inside the \y{variational DA family},
alongside 3D-Var, 4D-Var, and EnVar.

\end{column}

\end{columns}

\end{frame}
