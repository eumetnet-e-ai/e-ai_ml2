%!TEX root = lec18.tex
% ================================================================================
% Lecture 18 — Slide 11
% ================================================================================
\begin{frame}[t,fragile]

\mytitle{AI-Var in 1D: Training Loop with Background \& Observations}

\begin{columns}[T,totalwidth=\textwidth]

% ------------------------------------------------------------
\begin{column}[T]{0.52\textwidth}
\footnotesize

\textbf{Neural input construction}

\vspace{1mm}
The network receives \y{both} background and observations:

\begin{verbatim}
# observations mapped to grid
y_grid = torch.zeros(n)
mask   = torch.zeros(n)

y_grid[obs_idx] = y
mask[obs_idx]   = 1.0

# NN input
inp = torch.cat([xb_t, y_grid, mask])
\end{verbatim}

\vspace{-4mm}
\[
\delta x_\theta
=
\mathcal{N}_\theta(x_b,\; y)
\]

\vspace{0mm}
\hspace*{2cm}\rtext{\bf The solver is replaced by learning.}

\end{column}

% ------------------------------------------------------------
\begin{column}[T]{0.48\textwidth}
\footnotesize

\textbf{Training loop}

\vspace{-2mm}
\begin{verbatim}
for ep in range(n_epochs):
    optimizer.zero_grad()

    delta_x = model(inp)

    loss, Jb, Jo = J_3dvar(delta_x)

    loss.backward()
    optimizer.step()
\end{verbatim}

\vspace{0mm}
\textbf{Key points}

\vspace{1mm}
\begin{itemize}
  \item loss = classical 3D-Var functional
  \item no analysis targets
  \item gradients pass through $B^{-1}$ and $H$
\end{itemize}

\end{column}

\end{columns}

\end{frame}
