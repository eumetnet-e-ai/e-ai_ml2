%!TEX root = lec18.tex
% ================================================================================
% Lecture 18 — Slide 10
% ================================================================================
\begin{frame}[t,fragile]

\mytitle{AI-Var in 1D: From Mathematics to Code}

\begin{columns}[T,totalwidth=\textwidth]

% ------------------------------------------------------------
\begin{column}[T]{0.52\textwidth}
\footnotesize

\textbf{Neural increment model}

\vspace{1mm}
The network predicts the \y{analysis increment}:
\[
\delta x_\theta = \mathcal{N}_\theta(x_b, y)
\]

\vspace{1mm}
\textbf{PyTorch implementation (kept simple)}

\tiny
\begin{verbatim}
class IncrementMLP(nn.Module):
    def __init__(self, n):
        super().__init__()
        self.net = nn.Sequential(
            nn.Linear(3*n, 256),
            nn.Tanh(),
            nn.Linear(256, 256),
            nn.Tanh(),
            nn.Linear(256, n))

    def forward(self, inp):
        return self.net(inp)
\end{verbatim}

\end{column}

% ------------------------------------------------------------
\begin{column}[T]{0.44\textwidth}
\footnotesize

\vspace{-4mm}
{\bf Input vector:}
\[
\texttt{inp} = [x_b,\; y_{\text{grid}},\; \text{mask}]
\]


\textbf{Variational loss in code}

\vspace{1mm}
The 3D-Var cost is used \y{directly}:

\begin{verbatim}
def J_3dvar(delta_x):
    x = xb + delta_x
    innov = y - H @ x

    Jb = 0.5 * (delta_x @ B_inv @ delta_x)
    Jo = 0.5 * (innov @ R_inv @ innov)
    return Jb + Jo
\end{verbatim}

\vspace{2mm}
\rtext{\bf No analysis data.  
No solver.  
Only the variational objective.}

\end{column}

\end{columns}

\end{frame}
