%!TEX root = lec18.tex
% ================================================================================
% Lecture 18 — Slide 06
% ================================================================================
\begin{frame}[t]

\mytitle{Core Idea of AI-Var}

\begin{columns}[T,totalwidth=\textwidth]

% ------------------------------------------------------------
\begin{column}[T]{0.44\textwidth}
\footnotesize

\textbf{Classical variational DA}

\vspace{2mm}
\begin{itemize}
  \item iterative minimization of $J(x)$
  \item requires adjoints and solvers
  \item expensive and sequential
\end{itemize}

\vspace{3mm}
\rtext{\bf Replace the minimization algorithm.}

\vspace{2mm}
\textbf{AI-Var}

\vspace{1mm}
\begin{itemize}
  \item neural network approximates the minimizer
  \item outputs analysis $x_a$ directly
  \item trained using the \y{same cost function}
\end{itemize}

\end{column}

% ------------------------------------------------------------
\begin{column}[T]{0.54\textwidth}

\centering
\vspace{-8mm}
\hspace*{-1cm}
\begin{minipage}{9cm}
\includegraphics[width=\textwidth]{../../images/img18/aivar1.png}
\end{minipage}

\vspace{1mm}
\footnotesize
Classical DA workflow (left) versus AI-based inference of the analysis (right).

\vspace{5mm}
\footnotesize
\rtext{\bf Paradigm shift}

\vspace{1mm}
\begin{itemize}
  \item \color{darkgreen} DA becomes \y{inference}, not optimization
  \item milliseconds instead of iterations
  \item same statistics, different machinery
  \item flexible inflow of further information
\end{itemize}

\end{column}

\end{columns}

\end{frame}
