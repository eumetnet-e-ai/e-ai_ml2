%!TEX root = lec18.tex
% ================================================================================
% Lecture 18 — Slide 25
% ================================================================================
\begin{frame}[t]

\mytitle{Summary: Two AI Paths for Data Assimilation}

\begin{columns}[T,totalwidth=\textwidth]

% ------------------------------------------------------------
\begin{column}[T]{0.48\textwidth}
\footnotesize

\textbf{AI-Var (Keller \& Potthast)}

\y{\bf What it learns:}

\vspace{-3mm}
\[
(x_b,\;y)\;\mapsto\; x_a
\]

\vspace{0mm}
\begin{itemize}
  \item neural network approximates the minimizer
  \item trained by variational cost $J(x)$
  \item output = \y{one deterministic analysis field}
\end{itemize}

\vspace{2mm}
\y{\bf Strengths}
\begin{itemize}
  \item stable, structured increments
  \item scalable in dimension (1D $\rightarrow$ 2D demo)
  \item direct link to operational 3D/4D-Var
\end{itemize}

\end{column}

% ------------------------------------------------------------
\begin{column}[T]{0.48\textwidth}
\footnotesize

\vspace{-3mm}
\textbf{AI Particle Filter (Gaussian mixture)}

\y{\bf What it learns:}

\vspace{-2mm}
\[
(X^b,\;y)\;\mapsto\; X^a
\]

\vspace{0mm}
\begin{itemize}
  \item neural update transforms particle cloud
  \item trained to fit posterior mass distribution
  \item output = \y{analysis distribution (ensemble)}
\end{itemize}

\vspace{2mm}
\y{\bf Strengths}
\begin{itemize}
  \item non-Gaussian posteriors (multi-modal)
  \item distribution-aware filtering
  \item background-term ablation shows skill gain
\end{itemize}

\end{column}

\end{columns}

\vspace{4mm}
\rtext{\bf Take-home: AI-Var learns the analysis; AI-PF learns a posterior distribution.}

\end{frame}
