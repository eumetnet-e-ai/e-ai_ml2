%!TEX root = lec01.tex
% ================================================================================
% Slide
% ================================================================================
\begin{frame}[t,fragile]

% --- Title ---------------------------------------------------------------------
\mytitle{NumPy Arrays: 1D, 2D, and 3D Data - \y{Standard Programming Skills}}

% --- Content -------------------------------------------------------------------
\begin{columns}[T,totalwidth=\textwidth]

% --- Left column ---------------------------------------------------------------
\begin{column}{0.34\textwidth}

\textbf{Core idea}

\begin{itemize}
  \item One data structure for science
  \item Same logic for 1D, 2D, 3D data
  \item Used for time series, maps, fields
  \item Basis for ML tensors
  \item \y{Learn the basics}
  \y{yourself, its easy.}
\end{itemize}

\end{column}

% --- Right column --------------------------------------------------------------
\begin{column}{0.64\textwidth}

\begin{codeonly}{NumPy array dimensions}
import numpy as np

x = np.array([1, 2, 4])          # 1D
A = np.array([[1, 2], [3, 4]])   # 2D
B = np.zeros((10, 50, 100))      # 3D

print(x.shape, A.shape, B.shape)
# prints (3,) (2, 2) (10, 50, 100)

print(x)
# prints [1 2 4]
\end{codeonly}

\vspace{1mm}
{\footnotesize Dimensions encode structure, not meaning}

% --- End Columns ---------------------------------------------------------------
\end{column}
\end{columns}

\end{frame}
