% ================================================================================
% Lecture 17 — Slide 18
% ================================================================================
\begin{frame}[t]
\mytitle{Graph Transformer attention: which neighbors matter?}

\vspace{-2mm}
\begin{columns}[T,totalwidth=\textwidth]

% ------------------------------------------------------------
\begin{column}[T]{0.48\textwidth}
\vspace{-1mm}
\footnotesize

\textbf{Graph Transformer vs.\ plain message passing}
\begin{itemize}
\item classic GNN: neighbors contribute via fixed averaging / sum
\item GraphTransformer: each edge gets a learned \y{attention weight}
\end{itemize}

\vspace{2mm}
\textbf{Interpretation}
\begin{itemize}
\item for one receiver node $i$, all incoming edges $(j \rightarrow i)$
\item attention weights $\alpha_{i,j}$ sum to 1 (per head)
\item thick edges $\Rightarrow$ \y{high influence} of neighbor $j$
\end{itemize}

\vspace{2mm}
\color{red}\bf
\textbf{Take-away:}
the model learns \y{where to look} on the graph depending on flow regime and structure.
\end{column}

% ------------------------------------------------------------
\begin{column}[T]{0.52\textwidth}
\vspace{-2mm}
\raggedleft

\includegraphics[width=0.86\linewidth]{../../images/img17/aicon_attention_graph.png}

\vspace{-1mm}
\includegraphics[width=0.86\linewidth]{../../images/img17/aicon_attention_graph2.png}

\vspace{-2mm}
\scriptsize
\textbf{Example:}
(top) ICON multi-mesh graph (coarse + fine connections);
(bottom) attention setup
\end{column}

\end{columns}
\end{frame}
