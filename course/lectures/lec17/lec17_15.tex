% ================================================================================
% Lecture 17 — Slide 15
% ================================================================================
\begin{frame}[t]
\mytitle{ICON multimesh: the hidden graph behind AICON}

\vspace{-2mm}
\begin{columns}[T,totalwidth=\textwidth]

% ------------------------------------------------------------
\begin{column}[T]{0.5\textwidth}
\vspace{-1mm}
\footnotesize

\textbf{Key idea}
\begin{itemize}
\item AICON does not run on a regular lat--lon grid
\item it uses ICON's \y{hierarchical triangular mesh}
\item the GNN operates on a \y{hidden multi-mesh}
\end{itemize}

\vspace{2mm}
\textbf{Multi-mesh principle}
\begin{itemize}
\item union of coarse-to-fine subgraphs \\
\quad R$n$B0 $\cup$ R$n$B1 $\cup$ ... $\cup$ R$n$Bk
\item mixes \y{short-range} + \y{long-range} edges
\item similar spirit as GraphCast multi-mesh
\end{itemize}

\vspace{2mm}
\color{red}\bf
\textbf{Take-away:}
the multi-mesh gives the GNN both \y{local physics} and \y{global context}.
\end{column}

% ------------------------------------------------------------
\begin{column}[T]{0.5\textwidth}
\vspace{-2mm}
\raggedleft
% Use the level-wise multimesh plot from the notebook:
\includegraphics[width=0.95\linewidth]{../../images/img17/grid_visual.png}

\vspace{4mm}
\scriptsize
Hidden mesh edges shown level-wise: coarse levels enable long-range interaction.
\end{column}

\end{columns}
\end{frame}
