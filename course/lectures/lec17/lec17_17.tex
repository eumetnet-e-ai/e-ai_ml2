% ================================================================================
% Lecture 17 — Slide 17
% ================================================================================
\begin{frame}[t]
\mytitle{Graph structure: local neighborhoods on the ICON mesh}

\vspace{-2mm}
\begin{columns}[T,totalwidth=\textwidth]

% ------------------------------------------------------------
\begin{column}[T]{0.49\textwidth}
\vspace{-1mm}
\footnotesize

\textbf{Why this matters}
\begin{itemize}
\item AICON runs on the ICON grid as \y{graph neural network (GNN)}
\item each ICON cell $\Rightarrow$ one \y{graph node}
\item edges connect \y{mesh neighbors}
\end{itemize}

\vspace{2mm}
\textbf{Message passing view}
\begin{itemize}
\item prediction at one node uses information from:
\begin{itemize}
\item \textbf{1st neighbors} (directly connected)
\item \textbf{2nd neighbors} (neighbors-of-neighbors)
\end{itemize}
\item this defines the local receptive field of the GNN
\end{itemize}


\end{column}

% ------------------------------------------------------------
\begin{column}[T]{0.50\textwidth}
\vspace{-2mm}

\vspace{2mm}
\color{red}\bf
\textbf{Take-away:}
GNNs learn transport + interaction patterns through \y{local connectivity}.

\vspace{0mm}
\raggedleft
% Replace with your exported image:
\includegraphics[width=0.8\linewidth]{../../images/img17/graph_neighbours_crop.png}

\vspace{-2mm}
\scriptsize
Example: subgraph around one node (orange) with 1st and 2nd neighbors.
\end{column}

\end{columns}
\end{frame}
