% =================================================================================
% E-AI Tutorial Tex Macros for Slides
% Filename: lec_macros.tex
% 
% DWD, Roland Potthast 2025/2026
% Licence: CC-BY4.0
% =================================================================================

% --------------------------------------------------------------------------------
% Packages
% --------------------------------------------------------------------------------
\usepackage[T1]{fontenc}
\usepackage{lmodern}
\usepackage{graphicx}
\usepackage{tikz}
\usepackage{ragged2e}
\usepackage{xcolor}
\usepackage{tcolorbox}
\tcbuselibrary{listings,skins,breakable}
\usepackage{listings}
\tcbuselibrary{breakable}
\usepackage{graphicx}
\usepackage{fancybox}
\usepackage{xcolor}
\usepackage{xparse}
\usepackage{tikz}
\usepackage{fontawesome5}
\usepackage{amsmath,amssymb}
\usepackage{hyperref}
\usepackage{adjustbox}
% in lec_macros.tex (or lec17.tex preamble)
\usetikzlibrary{calc}

% Shortcuts for number sets
\newcommand{\R}{\mathbb{R}}

\usetikzlibrary{positioning,arrows.meta}


% --------------------------------------------------------------------------------
% Beamer setup
% --------------------------------------------------------------------------------
\setbeamertemplate{navigation symbols}{}
\setbeamercolor{background canvas}{bg=white}
\usepackage{pifont}
\usepackage{xcolor}

\usepackage{pifont}
\newcommand{\mdcheck}{\textcolor{green!70!black}{\ding{51}}}

% --------------------------------------------------------------------------------
% Colours
% --------------------------------------------------------------------------------
\definecolor{DWDblue}{RGB}{0,68,136}
\setbeamercolor{DWDrule}{fg=DWDblue}
\definecolor{codegreen}{rgb}{0.0, 0.5, 0.0} % Dark green for comments
\definecolor{codeblue}{rgb}{0.0, 0.2, 0.6} % Muted dark blue for keywords
\definecolor{codered}{rgb}{0.6, 0.1, 0.1}  % Muted red for strings
\definecolor{codegray}{rgb}{0.4, 0.4, 0.4} % Gray for numbers and operators
\definecolor{backcolor}{rgb}{0.97, 0.97, 0.97} % Very light gray background
\definecolor{lightblue}{RGB}{220, 230, 241} % Light blue for alternating rows
\definecolor{headerblue}{RGB}{180, 200, 230} % Slightly darker blue for headers
\definecolor{appendixblue}{RGB}{200, 220, 240} % Blue for appendix header
\definecolor{darkgreen}{RGB}{0,110,80}

\newcommand{\y}[1]{%
\colorbox{yellow!50}{#1}%
}\newcommand{\g}[1]{%
\colorbox{darkgreen!30}{#1}%
}
\newcommand{\red}{\color{red}}

% --------------------------------------------------------------------------------
% 
% --------------------------------------------------------------------------------
\newcommand{\rtext}[1]{{\color{red}#1}}

\newenvironment{tightmath}{
  \setlength{\abovedisplayskip}{3pt}
  \setlength{\belowdisplayskip}{3pt}
  \setlength{\abovedisplayshortskip}{2pt}
  \setlength{\belowdisplayshortskip}{2pt}
}{}

% --------------------------------------------------------------------------------
% Fonts
% --------------------------------------------------------------------------------
\setbeamerfont{title}{size=\Large,series=\bfseries}
\setbeamerfont{normal text}{size=\normalsize}

% --------------------------------------------------------------------------------
% Slide metadata
% --------------------------------------------------------------------------------
\newcommand{\LectureAuthor}{Roland Potthast}
\newcommand{\LectureYear}{2025--2026}

% --------------------------------------------------
% Header with logos
% --------------------------------------------------
\newcommand{\toplogos}{
  \noindent
  \vspace*{-5mm}
  \begin{minipage}{\textwidth}
    \raisebox{-0.5\height}{\includegraphics[height=1cm]{../../images/img00/ecmwf.png}}
    \hfill
    \raisebox{-0.5\height}{\includegraphics[height=0.9cm]{../../images/img00/university-of-reading-logo-vector.png}}
    \hfill
    \raisebox{-0.5\height}{\includegraphics[height=0.7cm]{../../images/img00/DWD-Logo_2013.svg.png}}
  \end{minipage}

  \vspace{-1mm}
\noindent
\begin{tikzpicture}
  \draw[DWDblue, line width=0.6pt] (0,0) -- (\textwidth,0);
\end{tikzpicture}
  \vspace{-3mm}
}

% --------------------------------------------------
% Title macro
% --------------------------------------------------
\newcommand{\mytitle}[1]{%
  {\usebeamerfont{title}\normalsize\color{DWDblue} #1}%
  \vspace{2mm}
}

% --------------------------------------------------
% Two-column layout macro
% --------------------------------------------------
\newcommand{\mycolumns}[2]{%
  \begin{columns}[T,totalwidth=\textwidth]
    \begin{column}{0.48\textwidth}
      \justifying
      #1
    \end{column}
    \begin{column}{0.48\textwidth}
      #2
    \end{column}
  \end{columns}
}

\makeatletter
% --------------------------------------------------------------------------------
% Footer: line + author | lecture | year | slide number
% --------------------------------------------------------------------------------
\setbeamertemplate{footline}{
  \leavevmode

% --- horizontal line above footer --------------------------------------------
\hbox to \paperwidth{%
  \hskip\beamer@leftmargin
  \begin{tikzpicture}
    \draw[DWDblue, line width=0.6pt]
      (0,0) --
      (\dimexpr\paperwidth-\beamer@leftmargin-\beamer@rightmargin\relax,0);
  \end{tikzpicture}
  \hskip\beamer@rightmargin
}

\vskip 0.6ex

  % --- footer content -----------------------------------------------------------
  \hbox to \paperwidth{%
    \hskip\beamer@leftmargin
    \begin{beamercolorbox}[
      wd=\dimexpr\paperwidth-\beamer@leftmargin-\beamer@rightmargin\relax,
      ht=2.2ex,
      dp=1.0ex,
      leftskip=0pt,
      rightskip=0pt
    ]{author in head/foot}
      \tiny\color{DWDblue}
      \LectureAuthor \hfill \LectureNumber \hfill \LectureYear \hfill Slide \insertframenumber
    \end{beamercolorbox}
    \hskip\beamer@rightmargin
  }
  \vskip 0.8ex
}
\makeatother

% ================================================================================
% Environment for Slides
% Arguments:
%   #1 : title
%   #2 : left column content
%   #3 : right column content
% ================================================================================
\newenvironment{myslide}[3]
{
  \begin{frame}[fragile]
    \toplogos
    \mytitle{#1}
    \begin{columns}[T,totalwidth=\textwidth]
      \begin{column}{0.48\textwidth}
        #2
      \end{column}
      \begin{column}{0.48\textwidth}
        #3
      \end{column}
    \end{columns}
  \end{frame}
}
{}

% Define the Python syntax highlighting style
\lstdefinestyle{pythonstyle}{
    language=Python,
    basicstyle=\ttfamily\small,    % Monospace font
    keywordstyle=\color{codeblue}\bfseries, % Keywords in blue
    stringstyle=\color{codered},  % Strings in red
    commentstyle=\color{codegreen}\itshape, % Comments in green italics
    numberstyle=\tiny\color{codegray}, % Line numbers in gray
    backgroundcolor=\color{gray!5}, % Light gray background
    breaklines=true,
    columns=fixed,
    showspaces=false,
    showstringspaces=false,
    keepspaces=true,
    numbers=left, % Show line numbers
    numbersep=5pt, % Adjust separation between numbers and text
    xleftmargin=7pt, % **Fix: Keeps numbers inside the box**
    framexleftmargin=15pt, % **Fix: Adds padding inside the box for numbers**
}


% ================================================================================
% Listing Environment
% ================================================================================
\newtcblisting{codeonly}[2][]{%
  listing only,
  listing options={
		style=pythonstyle, 
		basicstyle=\ttfamily\fontsize{9pt}{12pt}\selectfont,
		tabsize=2,          % Tabs = 2 spaces
    showtabs=false      % Don't show visible tab markers
		},
  breakable,
  colback=gray!5,
  colframe=gray!30,
  boxrule=0.2pt,
  arc=1.5mm,
  left=2mm, right=2mm, top=-2mm, bottom=-1mm,
  before skip=2pt, after skip=2pt,
  coltitle=gray!100,
  title={#2},
  fonttitle=\bfseries\footnotesize,
  #1
}

\makeatletter
\setbeamertemplate{headline}{
  \leavevmode
  \vspace*{2mm}
  \hbox to \paperwidth{%
    \hskip\beamer@leftmargin
    \begin{minipage}{\dimexpr\paperwidth-\beamer@leftmargin-\beamer@rightmargin\relax}
      \raisebox{-0.5\height}{%
        \includegraphics[height=1cm]{../../images/img00/ecmwf.png}%
      }\hfill
      \raisebox{-0.5\height}{%
        \includegraphics[height=0.9cm]{../../images/img00/university-of-reading-logo-vector.png}%
      }\hfill
      \raisebox{-0.5\height}{%
        \includegraphics[height=0.7cm]{../../images/img00/DWD-Logo_2013.svg.png}%
      }
    \end{minipage}
    \hskip\beamer@rightmargin
  }

  \vspace{-2.5mm}

  \hbox to \paperwidth{%
    \hskip\beamer@leftmargin
    \begin{tikzpicture}
      \draw[DWDblue, line width=0.6pt]
        (0,0) --
        (\dimexpr\paperwidth-\beamer@leftmargin-\beamer@rightmargin\relax,0);
    \end{tikzpicture}
    \hskip\beamer@rightmargin
  }

  \vspace{0mm}
}
\makeatother

\lstset{
  basicstyle=\ttfamily\footnotesize,
  columns=fullflexible,
  keepspaces=true,
  showstringspaces=false,
  upquote=true
}

% --------------------------------------------------------------------------------
% Image box with soft shadow (no fixed size)
% --------------------------------------------------------------------------------
\newcommand{\slideimage}[2][]{%
  \tcbox[
    colback=white,
    boxrule=0pt,
    arc=2mm,
    shadow={1mm}{-1mm}{0mm}{black!30},
    enhanced,
    left=0pt,
    right=0pt,
    top=0pt,
    bottom=0pt,
  ]{%
    \includegraphics[#1]{#2}%
  }%
}

% ------------------------------------------------------------
% Red rounded box overlay on an image
% Usage:
% \imgwithbox{<includegraphics>}{<x1>}{<y1>}{<x2>}{<y2>}
%
% Coordinates are in the same unit as you provide, typically cm.
% Lower-left corner (x1,y1), upper-right corner (x2,y2).
% ------------------------------------------------------------
\newcommand{\imgwithbox}[5]{%
\begin{tikzpicture}
  \node[anchor=south west, inner sep=0] (img) {#1};
  \begin{scope}[x={(img.south east)}, y={(img.north west)}]
    % coordinates are now normalized 0..1 of image width/height
    \draw[red, very thick, rounded corners=4pt]
      (#2,#3) rectangle (#4,#5);
  \end{scope}
\end{tikzpicture}%
}


% ============================================================
% Agenda highlight box from lecture number (1..20)
% 5 columns × 4 rows (20 lectures)
%
% User defines ONE rectangle region on the agenda image:
%   x in [\AgendaXLeft, \AgendaXRight]
%   y in [\AgendaYBottom, \AgendaYTop]
%
% Mapping:
%   row = mod(L-1,4) + 1    (top to bottom)
%   col = floor((L-1)/4)+1  (left to right)
%
% Box size inside each cell:
%   dx, dy ∈ (0..1]  fraction of cell width/height
%   (can be tuned per lecture)
%
% Outputs:
%   \AgBoxXone, \AgBoxYone, \AgBoxXtwo, \AgBoxYtwo
% ============================================================

% ------------------------------------------------------------
% REGION (you adjust these by trial & error)
% ------------------------------------------------------------
\def\AgendaXLeft{0.13}
\def\AgendaXRight{0.98}

\def\AgendaYTop{0.87}
\def\AgendaYBottom{0.03}

% ------------------------------------------------------------
% DEFAULT box size inside each cell (fractions of cell size)
% ------------------------------------------------------------
\def\AgendaBoxDX{1.05}   % 90% of cell width
\def\AgendaBoxDY{1.00}   % 80% of cell height

% ------------------------------------------------------------
% OUTPUT variables (used by your drawing overlay)
% ------------------------------------------------------------
\def\AgBoxXone{0.10}
\def\AgBoxYone{0.10}
\def\AgBoxXtwo{0.20}
\def\AgBoxYtwo{0.20}

% ------------------------------------------------------------
% MAIN FUNCTION
% Usage:
%   \setagendaboxforlecture{L}
%   \setagendaboxforlecture[dx][dy]{L}
%
% Example:
%   \setagendaboxforlecture{17}
%   \setagendaboxforlecture[0.75][0.60]{17}
% ------------------------------------------------------------
\NewDocumentCommand{\setagendaboxforlecture}{ O{\AgendaBoxDX} O{\AgendaBoxDY} m }{%
  \pgfmathtruncatemacro{\L}{#3}%

  % row/col mapping
  \pgfmathtruncatemacro{\row}{mod(\L-1,4)+1}% 1..4
  \pgfmathtruncatemacro{\col}{floor((\L-1)/4)+1}% 1..5

  % cell size
  \pgfmathsetmacro{\W}{(\AgendaXRight-\AgendaXLeft)/5.0}%
  \pgfmathsetmacro{\H}{(\AgendaYTop-\AgendaYBottom)/4.0}%

  % cell bounds
  \pgfmathsetmacro{\xAcell}{\AgendaXLeft + (\col-1)*\W}%
  \pgfmathsetmacro{\xBcell}{\AgendaXLeft + (\col)*\W}%

  % row=1 is top
  \pgfmathsetmacro{\yTopCell}{\AgendaYTop - (\row-1)*\H}%
  \pgfmathsetmacro{\yBotCell}{\AgendaYTop - (\row)*\H}%

  % box fractions dx,dy
  \pgfmathsetmacro{\dx}{#1}%
  \pgfmathsetmacro{\dy}{#2}%

  % mid point of the cell
  \pgfmathsetmacro{\xMid}{0.5*(\xAcell+\xBcell)}%
  \pgfmathsetmacro{\yMid}{0.5*(\yTopCell+\yBotCell)}%

  % half box size in absolute figure coordinates
  \pgfmathsetmacro{\xHalf}{0.5*\dx*\W}%
  \pgfmathsetmacro{\yHalf}{0.5*\dy*\H}%

  % export corners
  \xdef\AgBoxXone{\xMid-\xHalf}%
  \xdef\AgBoxXtwo{\xMid+\xHalf}%
  \xdef\AgBoxYone{\yMid-\yHalf}%
  \xdef\AgBoxYtwo{\yMid+\yHalf}%
}

