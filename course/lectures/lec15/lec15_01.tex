%!TEX root = lec15.tex
% ================================================================================
% Lecture 15 — Slide 01
% ================================================================================
\begin{frame}[t,fragile]

\mytitle{Why Anemoi?}

\begin{columns}[T,totalwidth=\textwidth]

% ------------------------------------------------------------
\begin{column}[T]{0.55\textwidth}
\footnotesize

\textbf{Weather forecasting is a large-scale problem}

\begin{itemize}
  \item modern forecasts rely on \y{very large state vectors}
  \item global and regional fields with millions of grid points
  \item high temporal resolution and long time series
\end{itemize}

\vspace{2mm}
\textbf{Training data is massive}

\begin{itemize}
  \item global reanalyses such as \y{ERA5}, \y{ICON-DREAM}
  \item convection-permitting simulations (e.g.\ Arome, ICON-Force)
  \item multi-variable, multi-level, multi-year datasets
\end{itemize}

\vspace{2mm}
These datasets quickly reach \rtext{terabyte scale}.

\end{column}

% ------------------------------------------------------------
\begin{column}[T]{0.43\textwidth}
\footnotesize

\vspace{-8mm}
\textbf{What this implies for ML}

\begin{itemize}
  \item training must be \y{parallel and distributed}
  \item data access must be chunked and efficient
  \item geometry and grid structure must be respected
\end{itemize}

\vspace{2mm}
\textbf{Role of Anemoi}

\begin{itemize}
  \item parallelizes \y{fields and samples}
  \item integrates data, graphs, models, and training
  \item supports the \y{full ML lifecycle} for weather models
\end{itemize}

\vspace{2mm}
\rtext{\bf Anemoi is not a model — it is a framework.}

\end{column}

\end{columns}

\end{frame}
