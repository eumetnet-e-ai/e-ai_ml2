%!TEX root = lec15.tex
% ================================================================================
% Lecture 15 — Slide 05
% ================================================================================
\begin{frame}[t,fragile]

\mytitle{Hydra: Managing Variants of the Same Experiment}

\begin{columns}[T,totalwidth=\textwidth]

% ------------------------------------------------------------
\begin{column}[T]{0.50\textwidth}
\footnotesize

\textbf{Motivation}

\begin{itemize}
  \item same training code
  \item same task (e.g.\ learning $\sin(x)$)
  \item different model architectures
\end{itemize}

\vspace{0mm}
Hydra allows:
\begin{itemize}
  \item selecting model variants via YAML
  \item switching architectures without code changes
  \item keeping experiments comparable
\end{itemize}

\vspace{0mm}
\textbf{Base configuration}

\begin{itemize}
  \item defines which components are active
  \item serves as the experiment entry point
\end{itemize}

\vspace{2mm}
\footnotesize
\rtext{\bf The experiment is composed, not rewritten.}

\end{column}

% ------------------------------------------------------------
\begin{column}[T]{0.50\textwidth}
\footnotesize

\vspace{-4mm}
\begin{codeonly}{config.yaml}
defaults:
  - model: mlp
  - training: simple
  - _self_
\end{codeonly}

\vspace{0mm}
\begin{codeonly}{model/mlp.yaml}
layer_sizes: [1, 64, 1]
activation: relu
\end{codeonly}

\vspace{0mm}
\begin{codeonly}{model/deep.yaml}
layer_sizes: [1, 128, 64, 32, 1]
activation: tanh
\end{codeonly}

\end{column}

\end{columns}


\end{frame}



