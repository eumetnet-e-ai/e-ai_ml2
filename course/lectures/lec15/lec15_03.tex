%!TEX root = lec15.tex
% ================================================================================
% Lecture 15 — Slide 03
% ================================================================================
\begin{frame}[t,fragile]

\mytitle{YAML: The Configuration Backbone}

\begin{columns}[T,totalwidth=\textwidth]

% ------------------------------------------------------------
\begin{column}[T]{0.52\textwidth}
\footnotesize

\textbf{Why configuration matters at scale}

\begin{itemize}
  \item experiments involve many interacting components
  \item parameters change more often than code
  \item reproducibility depends on \y{explicit configuration}
\end{itemize}

\vspace{2mm}
In Anemoi, \y{YAML files define}:
\begin{itemize}
  \item datasets and preprocessing
  \item model architectures and hyperparameters
  \item training strategies and resources
\end{itemize}

\end{column}

% ------------------------------------------------------------
\begin{column}[T]{0.46\textwidth}
\footnotesize

\textbf{Why YAML?}

\begin{itemize}
  \item human-readable and versionable
  \item hierarchical and structured
  \item easy to override and compose
\end{itemize}

\vspace{2mm}
\textbf{Design choice}

\begin{itemize}
  \item no Python code for experiment logic
  \item no hidden defaults in scripts
  \item configuration becomes a \rtext{first-class artifact}
\end{itemize}

\vspace{2mm}
\rtext{\bf In Anemoi, changing the experiment means changing YAML.}

\end{column}

\end{columns}

\end{frame}
