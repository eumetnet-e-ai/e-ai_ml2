%!TEX root = lec15.tex
% ================================================================================
% Lecture 15 — Slide 05
% ================================================================================
\begin{frame}[t]

\mytitle{OmegaConf: From YAML to Runtime Objects}

\begin{columns}[T,totalwidth=\textwidth]

% ------------------------------------------------------------
\begin{column}[T]{0.52\textwidth}
\footnotesize

\textbf{The problem}

\begin{itemize}
  \item YAML files are static text
  \item training code needs structured objects
  \item configuration must be inspectable at runtime
\end{itemize}

\vspace{2mm}
\textbf{OmegaConf solves this}

\begin{itemize}
  \item represents configuration as Python objects
  \item preserves hierarchy and structure
  \item supports interpolation and defaults
\end{itemize}

\vspace{2mm}
The result is a \y{\texttt{DictConfig}} object.

\end{column}

% ------------------------------------------------------------
\begin{column}[T]{0.44\textwidth}
\footnotesize

\textbf{Properties of \texttt{DictConfig}}

\begin{itemize}
  \item hierarchical (mirrors YAML structure)
  \item accessed via attributes or keys
  \item supports type checking
\end{itemize}

\vspace{2mm}
\textbf{Why this matters}

\begin{itemize}
  \item training code stays generic
  \item behaviour is fully configuration-driven
  \item experiments become reproducible by construction
\end{itemize}

\vspace{2mm}
\rtext{\bf Configuration becomes part of the program state.}

\end{column}

\end{columns}

\end{frame}
