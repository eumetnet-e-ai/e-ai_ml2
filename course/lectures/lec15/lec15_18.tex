%!TEX root = lec15.tex
% ================================================================================
% Lecture 15 — Slide 18
% ================================================================================
\begin{frame}[t,fragile]

\mytitle{Validating and Inspecting Zarr Datasets}

\begin{columns}[T,totalwidth=\textwidth]

% ------------------------------------------------------------
\begin{column}[T]{0.56\textwidth}
\footnotesize

\textbf{Why validation matters}

\begin{itemize}
  \item training assumes consistent metadata
  \item missing values or shapes cause silent errors
  \item errors should be caught \y{before} training
\end{itemize}

\vspace{2mm}
Anemoi provides built-in tools to:
\begin{itemize}
  \item verify temporal coverage
  \item check spatial resolution and shapes
  \item ensure statistics are present
\end{itemize}

\vspace{2mm}
Validation is part of the dataset lifecycle,
not an afterthought.

\end{column}

% ------------------------------------------------------------
\begin{column}[T]{0.40\textwidth}
\footnotesize

\begin{codeonly}{Inspecting a dataset}
anemoi-datasets inspect era5.zarr
\end{codeonly}

\vspace{2mm}
Typical output includes:
\begin{itemize}
  \item time range and frequency
  \item variables and dimensions
  \item min / max / mean / std
  \item total size and chunking
\end{itemize}

\vspace{2mm}
\rtext{\bf Only validated datasets enter training.}

\end{column}

\end{columns}

\vspace{2mm}
\footnotesize
\href{https://anemoi.readthedocs.io/projects/datasets/en/latest/}
     {\texttt{Docs: anemoi-datasets}}

\end{frame}
