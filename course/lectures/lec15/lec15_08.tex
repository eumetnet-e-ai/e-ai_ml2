%!TEX root = lec15.tex
% ================================================================================
% Lecture 15 — Slide 06
% ================================================================================
\begin{frame}[t]

\mytitle{Hydra in Practice: Same Code, Different Outcomes}

\begin{columns}[T,totalwidth=\textwidth]

% ------------------------------------------------------------
\begin{column}[T]{0.46\textwidth}
\footnotesize

\textbf{What changes between these runs}

\begin{itemize}
  \item only the \y{Hydra configuration}
  \item different model definitions (\texttt{mlp} vs \texttt{deep})
  \item same training loop, same data
\end{itemize}

\vspace{2mm}
Hydra controls:
\begin{itemize}
  \item which YAML files are composed
  \item which model architecture is instantiated
  \item which hyperparameters are active
\end{itemize}

\vspace{0mm}
\textbf{Key observation}

\begin{itemize}
  \item shallow model converges smoothly
  \item deeper model shows slower, less stable training
\end{itemize}

\end{column}

% ------------------------------------------------------------
\begin{column}[T]{0.54\textwidth}
\centering

\vspace{-2mm}
\includegraphics[width=0.48\textwidth]{../../images/img15/hydra1_mlp.png}
\includegraphics[width=0.48\textwidth]{../../images/img15/hydra2_mlp.png}

\vspace{2mm}

\includegraphics[width=0.48\textwidth]{../../images/img15/hydra1_deep.png}
\includegraphics[width=0.48\textwidth]{../../images/img15/hydra2_deep.png}

\vspace{2mm}
\footnotesize
\rtext{\bf Hydra makes architectural choices explicit, comparable, and reproducible.}

\end{column}

\end{columns}



\end{frame}
