%!TEX root = lec15.tex
% ================================================================================
% Lecture 15 — Slide 10
% ================================================================================
\begin{frame}[t]

\mytitle{Icosahedral and Geodesic Graphs}

\begin{columns}[T,totalwidth=\textwidth]

% ------------------------------------------------------------
\begin{column}[T]{0.46\textwidth}
\footnotesize

\textbf{ICON as a guiding principle}

\begin{itemize}
  \item refined icosahedral grids
  \item nearly uniform cell areas
  \item no singularities at the poles
\end{itemize}

\vspace{2mm}
Anemoi adopts the same idea:
\begin{itemize}
  \item nodes placed on a refined icosahedron
  \item spherical geometry is explicit
  \item resolution controlled by refinement level
\end{itemize}

\vspace{2mm}
\textbf{Key trade-off}

\begin{itemize}
  \item higher resolution $\Rightarrow$ more nodes
  \item increased cost, but better spatial fidelity
\end{itemize}

\end{column}

% ------------------------------------------------------------
\begin{column}[T]{0.50\textwidth}

\centering
\vspace{-10mm}

\includegraphics[width=0.95\textwidth]{../../images/img15/anemoi_graph_2.png}

\vspace{-1mm}
\footnotesize
Icosahedral graph on the sphere (orthographic projection).
Edges follow great-circle distances.

\end{column}

\end{columns}

\end{frame}
