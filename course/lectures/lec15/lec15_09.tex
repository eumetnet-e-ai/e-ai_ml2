%!TEX root = lec15.tex
% ================================================================================
% Lecture 15 — Slide 09
% ================================================================================
\begin{frame}[t]

\mytitle{Why Graphs in Weather Machine Learning?}

\begin{columns}[T,totalwidth=\textwidth]

% ------------------------------------------------------------
\begin{column}[T]{0.52\textwidth}
\footnotesize

\textbf{Weather data is not Cartesian}

\begin{itemize}
  \item global models use spherical geometry
  \item grids are often \y{non-uniform and unstructured}
  \item classical CNN assumptions break down
\end{itemize}

\vspace{2mm}
Examples:
\begin{itemize}
  \item ICON triangular grid
  \item reduced Gaussian grids
  \item observation networks
\end{itemize}

\end{column}

% ------------------------------------------------------------
\begin{column}[T]{0.44\textwidth}
\footnotesize

\textbf{Why graphs are a natural abstraction}

\begin{itemize}
  \item nodes represent spatial locations
  \item edges represent physical neighbourhoods
  \item locality is explicit, not implicit
\end{itemize}

\vspace{2mm}
\textbf{Key idea}

\begin{itemize}
  \item move from \y{array indices} to \y{connectivity}
  \item geometry becomes part of the model
\end{itemize}

\vspace{2mm}
\rtext{\bf Graphs decouple geometry from resolution.}

\end{column}

\end{columns}

\end{frame}
