%!TEX root = lec15.tex
% ================================================================================
% Lecture 15 — Slide 12
% ================================================================================
\begin{frame}[t]

\mytitle{Anemoi: A Modular ML Framework for Weather}

\begin{columns}[T,totalwidth=\textwidth]

% ------------------------------------------------------------
\begin{column}[T]{0.52\textwidth}
\footnotesize

\textbf{Design philosophy}

\begin{itemize}
  \item separate concerns cleanly
  \item make each component replaceable
  \item scale from experiments to operations
\end{itemize}

\vspace{2mm}
Anemoi is not a single package, but a \y{coordinated ecosystem}.

\vspace{2mm}
Each package addresses one layer:
\begin{itemize}
  \item data handling
  \item graph construction
  \item model definition
  \item training orchestration
\end{itemize}

\end{column}

% ------------------------------------------------------------
\begin{column}[T]{0.44\textwidth}
\footnotesize

\textbf{Core Anemoi packages}

\begin{itemize}
  \item \texttt{anemoi-datasets}
  \item \texttt{anemoi-graphs} {\color{blue}in anemoi-core repo}
  \item \texttt{anemoi-models} {\color{blue}in anemoi-core repo}
  \item \texttt{anemoi-training} {\color{blue}in anemoi-core repo}
\end{itemize}

\vspace{2mm}
All packages:
\begin{itemize}
  \item use YAML + Hydra + OmegaConf
  \item integrate via clearly defined interfaces
  \item are developed in a shared monorepo
\end{itemize}

\vspace{2mm}
\rtext{\bf Graphs connect geometry; packages connect workflows.}

\end{column}

\end{columns}

\end{frame}
