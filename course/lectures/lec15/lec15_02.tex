%!TEX root = lec15.tex
% ================================================================================
% Lecture 15 — Slide 02
% ================================================================================
\begin{frame}[t,fragile]

\mytitle{Design Philosophy of Anemoi}

\begin{columns}[T,totalwidth=\textwidth]

% ------------------------------------------------------------
\begin{column}[T]{0.52\textwidth}
\footnotesize

\textbf{Core principles}

\begin{itemize}
  \item \y{configuration over hard-coded logic}
  \item clear separation of responsibilities
  \item scalable by construction
\end{itemize}

\vspace{2mm}
Anemoi is designed such that:
\begin{itemize}
  \item data handling,
  \item graph construction,
  \item model definition,
  \item training orchestration
\end{itemize}
are \rtext{independent but composable} components.

\end{column}

% ------------------------------------------------------------
\begin{column}[T]{0.46\textwidth}
\footnotesize

\vspace{-6mm}
\textbf{Declarative workflows}

\begin{itemize}
  \item experiments are defined in \y{YAML}
  \item components are selected, not programmed
  \item changes are traceable and reproducible
\end{itemize}

\vspace{2mm}
\textbf{Operational mindset}

\begin{itemize}
  \item same setup works on laptop, HPC, and cloud
  \item parallelism is explicit, not accidental
  \item full provenance of data and models
\end{itemize}

\vspace{2mm}
\rtext{\bf Anemoi treats ML experiments as engineering systems.}

\end{column}

\end{columns}

\end{frame}
