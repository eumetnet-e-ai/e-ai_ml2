%!TEX root = lec15.tex
% ================================================================================
% Lecture 15 — Slide 04
% ================================================================================
\begin{frame}[t,fragile]

\mytitle{YAML in Practice: Declaring an Experiment}

\begin{columns}[T,totalwidth=\textwidth]

% ------------------------------------------------------------
\begin{column}[T]{0.48\textwidth}
\footnotesize

\textbf{A single YAML file describes an experiment}

\begin{itemize}
  \item what data is used
  \item which model is trained
  \item how training is executed
\end{itemize}

\vspace{2mm}
No experiment logic is hidden in Python code.
The configuration is the experiment.


\vspace{5mm}
\footnotesize
\rtext{\bf This YAML file fully specifies one training run.}

\end{column}

% ------------------------------------------------------------
\begin{column}[T]{0.50\textwidth}
\footnotesize

\vspace{-5mm}
\begin{codeonly}{Example: training configuration (simplified)}
dataset:
  name: era5
  variables: [t2m, u10, v10]
  resolution: 0.25

model:
  name: graph_transformer
  hidden_dim: 256
  num_layers: 6

training:
  batch_size: 4
  max_epochs: 50
  accelerator: gpu
\end{codeonly}

\end{column}

\end{columns}


\end{frame}
