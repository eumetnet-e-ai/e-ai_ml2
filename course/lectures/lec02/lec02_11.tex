%!TEX root = lec02.tex
% ================================================================================
% Slide
% ================================================================================
\begin{frame}[t,fragile]

% --- Title ---------------------------------------------------------------------
\mytitle{APIs as a Scaling Principle}

% --- Content -------------------------------------------------------------------
\begin{columns}[T,totalwidth=\textwidth]

% --- Left column ---------------------------------------------------------------
\begin{column}[T]{0.48\textwidth}

\textbf{Core Idea}

\begin{itemize}
  \item APIs defines a contract on \y{how} functionality is accessed
  \item Clear \y{separation} of interface and implementation
  \item Same principle from notebooks to services
\end{itemize}

\vspace{1mm}
\textbf{Why This Matters}

\begin{itemize}
  \item Logic becomes reusable and testable
  \item Notebooks stay thin and readable
  \item Systems can grow without rewrites
\end{itemize}

\end{column}

% --- Right column --------------------------------------------------------------
\begin{column}[T]{0.48\textwidth}

\vspace{-1cm}
\textbf{Example 1: Local API (project code)}

\begin{codeonly}{Function API usage}
from model import forecast
y = forecast(x, params)
\end{codeonly}

\vspace{1mm}
\textbf{Example 2: Web API (service)}

\begin{codeonly}{HTTP request}
response = requests.get(
  "/weather",
  params={"city": "Berlin", "date": "2025-01-20"})
data = response.json()
\end{codeonly}

\vspace{1mm}
\textbf{Key Insight}

\begin{itemize}
  \item APIs shift complexity behind a \y{stable boundary}
\end{itemize}

% --- End Columns, Column and Frame ---------------------------------------------
\end{column}
\end{columns}
\end{frame}
