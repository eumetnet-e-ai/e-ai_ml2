%!TEX root = lec02.tex
% ================================================================================
% Slide 
% ================================================================================
\begin{frame}[t]

\mytitle{Remote Jupyter — Firewalls \& Troubleshooting}

\begin{columns}[T,totalwidth=\textwidth]

% --- Left column ---------------------------------------------------------------
\begin{column}[T]{0.48\textwidth}

\textbf{Typical Problems}

\begin{itemize}
  \item Local port already in use
  \item Remote port 8888 blocked by firewall
  \item Unstable SSH connection
  \item Multi-hop access (Jump / Bastion host)
\end{itemize}

\vspace{1mm}
\textbf{Important Note}

\begin{itemize}
  \item A blocked \texttt{8888} is \y{not an error}
  \item SSH tunnels do not require open inbound ports
\end{itemize}

\end{column}

% --- Right column --------------------------------------------------------------
\begin{column}[T]{0.48\textwidth}

\textbf{Best Practices}

\begin{itemize}
  \item Remote:
  \begin{itemize}
    \item \texttt{--ip=127.0.0.1}
    \item no public binding
  \end{itemize}
  \item Local:
  \begin{itemize}
    \item choose a free port (9001, 9002, \ldots)
  \end{itemize}
  \item Infrastructure:
  \begin{itemize}
    \item Bastion / jump host via \texttt{-J}
  \end{itemize}
\end{itemize}

\vspace{1mm}
\textbf{Recommendation}

\begin{itemize}
  \item \y{Never} expose Jupyter notebook ports publicly
\end{itemize}

\end{column}

\end{columns}
\end{frame}
