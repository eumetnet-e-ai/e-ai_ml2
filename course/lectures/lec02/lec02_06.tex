%!TEX root = lec02.tex
% ================================================================================
% Slide
% ================================================================================
\begin{frame}[t,fragile]

% --- Title ---------------------------------------------------------------------
\mytitle{Markdown and Narrative Computing}

% --- Content -------------------------------------------------------------------
\begin{columns}[T,totalwidth=\textwidth]

% --- Left column ---------------------------------------------------------------
\begin{column}[T]{0.48\textwidth}

\textbf{Why Markdown Matters}

\begin{itemize}
  \item Makes notebooks \y{readable}
  \item \y{Explains} intent, not just results
  \item Turns experiments into documents
\end{itemize}

\vspace{1mm}
\textbf{Narrative Computing}

\begin{itemize}
  \item Code, text and results in one place
  \item \y{Reasoning} becomes explicit
  \item Supports review and reuse
\end{itemize}

\end{column}

% --- Right column --------------------------------------------------------------
\begin{column}[T]{0.48\textwidth}

\textbf{Minimum You Should Use}

\begin{itemize}
  \item Headings \,(\texttt{\#}, \texttt{\#\#})
  \item Bullet lists \,(\texttt{-})
  \item Inline formulas \,(\texttt{\$x\^{}2\$})
  \item Short explanations 
\end{itemize}

\vspace{1mm}
\textbf{Markdown Cells}

\begin{itemize}
  \item Change cell type: \texttt{Esc} \,$\rightarrow$\, \texttt{M}
\end{itemize}

\vspace{1mm}
\textbf{Key Message}

\begin{itemize}
  \item A notebook should read like a \y{lab notebook}
\end{itemize}

% --- End Columns, Column and Frame ---------------------------------------------
\end{column}
\end{columns}
\end{frame}
