%!TEX root = lec02.tex
% ================================================================================
% Slide
% ================================================================================
\begin{frame}[t,fragile]

\mytitle{Native Code Integration: Fortran and C++ in Python \& ML}

\begin{columns}[T,totalwidth=\textwidth]

% --- Left column ---------------------------------------------------------------
\begin{column}[T]{0.52\textwidth}

\textbf{Why Native Code Matters}

\begin{itemize}
  \item Decades of validated \y{Fortran} in NWP
  \item High-performance kernels in \y{C++}
  \item Tight control over memory and execution
  \item Reuse of trusted implementations
\end{itemize}

\vspace{1mm}
\textbf{Typical Use Cases}

\begin{itemize}
  \item Physical parameterizations
  \item Linear operators, solvers, kernels
  \item Observation operators
  \item Legacy model components
\end{itemize}

\end{column}

% --- Right column --------------------------------------------------------------
\begin{column}[T]{0.48\textwidth}

\textbf{Python as the Orchestration Layer}

\begin{itemize}
  \item Python controls the \y{workflow}
  \item Native code provides \y{compute kernels}
  \item Clean separation via \y{APIs}
\end{itemize}

\vspace{1mm}
\textbf{Integration Options (Overview)}

\begin{itemize}
  \item \texttt{ctypes} – explicit C-compatible interfaces
  \item \texttt{f2py} – automatic Fortran bindings
  \item \texttt{pybind11} – modern C++ bindings
  \item Shared libraries: \texttt{.so} / \texttt{.dylib}
\end{itemize}

\end{column}

% --- End Columns --------------------------------------------------------------
\end{columns}
\end{frame}
