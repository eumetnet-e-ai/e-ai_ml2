%!TEX root = lec02.tex
% ================================================================================
% Slide 
% ================================================================================
\begin{frame}[t,fragile]

\mytitle{Remote Jupyter — Port Forwarding Recipe}

\begin{columns}[T,totalwidth=\textwidth]

% --- Left column ---------------------------------------------------------------
\begin{column}[T]{0.46\textwidth}

\textbf{Step 1: Start Jupyter Remotely}

\vspace{1mm}
\begin{codeonly}{Remote (Linux)}
jupyter notebook \
  --no-browser \
  --port=8888
\end{codeonly}

\vspace{1mm}

\begin{itemize}
  \item Access token appears in the terminal
  \item Port is \y{local only} on the remote server
\end{itemize}

{\tiny\begin{lstlisting}
To access the server, open this file in a browser:
  file:///hpc/uhome/rpotthas/.local/share/jupyter/runtime/jpserver-1662241-open.html
Or copy and paste one of these URLs:
  http://localhost:8888/tree?token=369b469c83e8eaa369ee02ea443d994865ca93e4185bb385
  http://127.0.0.1:8888/tree?token=369b469c83e8eaa369ee02ea443d994865ca93e4185bb385
\end{lstlisting}}

\end{column}

% --- Right column --------------------------------------------------------------
\begin{column}[T]{0.52\textwidth}

\textbf{Step 2: Create SSH Tunnel}

\vspace{1mm}
\begin{codeonly}{Local (Bash / PowerShell)}
ssh -N -L 9001:localhost:8888 \
    user@remote-host
\end{codeonly}

\vspace{1mm}
\textbf{Step 3: Open in Browser}

\begin{itemize}
  \item \texttt{http://localhost:9001}
  \item Use token from the remote log
  \item \y{Adjust port} if needed
\end{itemize}

\end{column}

\end{columns}
\end{frame}
