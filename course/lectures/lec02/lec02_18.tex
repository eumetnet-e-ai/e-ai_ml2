%!TEX root = lec02.tex
% ================================================================================
% Slide
% ================================================================================
\begin{frame}[t,fragile]

\mytitle{Fortran with C Bindings: A Stable Interface}

\begin{columns}[T,totalwidth=\textwidth]

% --- Left column ---------------------------------------------------------------
\begin{column}[T]{0.37\textwidth}

\vspace{-2mm}
\textbf{Why C Bindings?}

\begin{itemize}
  \item Fortran and Python do \y{not} talk directly
  \item The common denominator is the \y{C ABI}
  \item Stable, explicit, language-independent
\end{itemize}

\vspace{1mm}
\textbf{Key Concept}

\begin{itemize}
  \item Fortran exposes functions as \y{C-compatible symbols}
  \item No name mangling
  \item Well-defined data types
\end{itemize}

\end{column}

% --- Right column --------------------------------------------------------------
\begin{column}[T]{0.62\textwidth}

\vspace{-3mm}
\begin{codeonly}{Minimal Fortran Example}
function f_sin_cos(x) result(f) bind(C)
  use iso_c_binding
  real(c_double), intent(in) :: x
  real(c_double) :: f
  f = sin(x) * cos(x)
end function
\end{codeonly}

\vspace{1mm}
\textbf{What This Ensures}

\begin{itemize}
  \item Symbol name is predictable
  \item Argument layout follows C rules
  \item Callable from Python, C, C++
\end{itemize}

\end{column}

% --- End Columns --------------------------------------------------------------
\end{columns}
\end{frame}
