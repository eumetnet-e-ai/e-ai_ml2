%!TEX root = lec02.tex
% ================================================================================
% Slide
% ================================================================================
\begin{frame}[t,fragile]

% --- Title ---------------------------------------------------------------------
\mytitle{Lecture 2: Jupyter Notebooks, APIs and Servers}

% --- Content -------------------------------------------------------------------
\begin{columns}[T,totalwidth=\textwidth]

% --- Left column ---------------------------------------------------------------
\begin{column}[T]{0.48\textwidth}

\textbf{Goal of this Lecture}

\begin{itemize}
  \item Work \y{productively} with Jupyter Notebooks
  \item Understand Notebooks as part of a \y{scientific workflow}
  \item Prepare the ground for \y{reproducible} ML experiments
\end{itemize}

\vspace{1mm}
\textbf{Focus}

\begin{itemize}
  \item Not UI details, but \y{how things fit together}
  \item From exploration to engineering
\end{itemize}

\end{column}

% --- Right column --------------------------------------------------------------
\begin{column}[T]{0.48\textwidth}

\textbf{Topics Overview}

\begin{itemize}
  \item Jupyter Notebooks: Kernel, Server, Browser
  \item Environments and package management
  \item Markdown, magic commands, shell integration
  \item Data visualization as quality control
  \item \y{APIs} as a structuring principle
  \item Local, library and web APIs
  \item Native code integration (Fortran / C++)
\end{itemize}

% --- End Columns, Column and Frame ---------------------------------------------
\end{column}
\end{columns}
\end{frame}
