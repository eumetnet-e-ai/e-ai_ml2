%!TEX root = lec02.tex
% ================================================================================
% Slide
% ================================================================================
\begin{frame}[t,fragile]

\mytitle{Testing the Task REST API}

\begin{columns}[T,totalwidth=\textwidth]

% --- Left column ---------------------------------------------------------------
\begin{column}[T]{0.38\textwidth}

\textbf{What We Test}

\begin{itemize}
  \item Send JSON to the server (\texttt{POST /tasks})
  \item Server creates a new task
  \item Server assigns \y{ID} and initial \y{state}
\end{itemize}

\vspace{1mm}
\textbf{Expected Result}

\begin{itemize}
  \item HTTP status \texttt{201 Created}
  \item JSON response with:
  \begin{itemize}
    \item task id
    \item state = \texttt{created}
    \item echoed input data
  \end{itemize}
\end{itemize}

\end{column}

% --- Right column --------------------------------------------------------------
\begin{column}[T]{0.6\textwidth}

\vspace{-1.1cm}
\begin{codeonly}{Test via \texttt{curl}}
curl -X POST http://127.0.0.1:5000/tasks  -H "Content-Type: application/json" -d '{"type":"demo","params":{"x":1}}' 
\end{codeonly}

\begin{codeonly}{Minimal Python Task Creation}
import requests
r = requests.post(
  "http://127.0.0.1:5000/tasks",
  json={"type":"demo","params":{"x":1}})
print(r.status_code); print(r.json())
\end{codeonly}

\vspace{-1mm}
\begin{lstlisting}
{'id': 1, 'state': 'created',
 'data': {'type': 'demo',
          'params': {'x': 1}}}
\end{lstlisting}

% --- End Columns ---------------------------------------------------------------
\end{column}
\end{columns}
\end{frame}
