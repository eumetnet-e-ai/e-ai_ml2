%!TEX root = lec13.tex
% ================================================================================
% Lecture 13 — Slide 09
% ================================================================================
\begin{frame}[t]

\mytitle{Who Defines What “Good” Means?}

\begin{columns}[T,totalwidth=\textwidth]

% ------------------------------------------------------------
\begin{column}[T]{0.55\textwidth}
\footnotesize

In MLOps, \y{quality is not a technical concept}.

\vspace{2mm}
It is defined by:
\begin{itemize}
  \item physical constraints,
  \item domain knowledge,
  \item operational requirements,
  \item user expectations.
\end{itemize}

\vspace{2mm}
These aspects cannot be learned from data alone.

\vspace{2mm}
\textbf{Therefore:}

\begin{itemize}
  \item \y{Domain experts define what “good” means},
  \item ML systems optimize \emph{towards} this definition,
  \item Operations validate it against reality.
\end{itemize}


\end{column}

% ------------------------------------------------------------
\begin{column}[T]{0.41\textwidth}

\vspace{2mm}
{\color{darkgreen}
\textbf{\textit{Without a domain expertise, metrics are meaningless.}}
}

\vspace{6mm}
\footnotesize
\textbf{Domain Expert}

\begin{itemize}
  \item \y{Defines quality and success criteria}
  \item Specifies relevant metrics and constraints
  \item Interprets model behaviour in context
  \item \rtext{Does not implement models or pipelines}
\end{itemize}

\end{column}

\end{columns}

\end{frame}
