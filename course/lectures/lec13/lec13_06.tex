%!TEX root = lec13.tex
% ================================================================================
% Lecture 13 — Slide 06
% ================================================================================
\begin{frame}[t,fragile]

\mytitle{How Automation Is Achieved in MLOps}

\begin{columns}[T,totalwidth=\textwidth]

% ------------------------------------------------------------
\begin{column}[T]{0.52\textwidth}
\footnotesize

\textbf{Automation principles}

\begin{itemize}
  \item Each step is \y{explicitly defined} and scripted
  \item Execution is \y{trigger-based}, not manual
  \item Results are \y{logged and versioned} automatically
\end{itemize}

\vspace{2mm}
Automation replaces informal procedures by
\rtext{repeatable execution rules}.

\end{column}

% ------------------------------------------------------------
\begin{column}[T]{0.46\textwidth}
\footnotesize

\vspace{-5mm}
\textbf{Technical mechanisms}

\begin{itemize}
  \item \y{CI/CD pipelines} trigger builds and checks
  \item \y{Version control} tracks code, configs, and metadata
  \item \y{Registries} store models and runtime artifacts
  \item \y{Containers} freeze execution environments
\end{itemize}

\vspace{2mm}
\textbf{Key effect}

\begin{itemize}
  \item Humans decide \y{what should happen}
  \item Systems enforce \rtext{how it happens}
\end{itemize}

\end{column}

\end{columns}

\vspace{2mm}
\footnotesize
\rtext{\bf Automation makes MLOps scalable, auditable, and safe.}

\end{frame}
