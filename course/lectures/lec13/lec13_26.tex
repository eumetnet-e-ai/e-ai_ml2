%!TEX root = lec13.tex
% ================================================================================
% Lecture 13 — Slide 27
% ================================================================================
\begin{frame}[t]

\mytitle{Why Containers are Essential in ML Frameworks}

\begin{columns}[T,totalwidth=\textwidth]

% ------------------------------------------------------------
\begin{column}[T]{0.48\textwidth}
\footnotesize

\textbf{The core problem in ML}

\begin{itemize}
  \item ML code depends on:
  \begin{itemize}
    \item specific library versions
    \item CUDA / CPU features
    \item system-level dependencies
  \end{itemize}
  \item These dependencies \rtext{change over time}
  \item Results become hard to reproduce
\end{itemize}

\vspace{2mm}
\rtext{\bf ML models are not standalone artifacts.}

\end{column}

% ------------------------------------------------------------
\begin{column}[T]{0.48\textwidth}
\footnotesize

\textbf{What containers provide}

\begin{itemize}
  \item Encapsulation of:
  \begin{itemize}
    \item code
    \item libraries
    \item runtime environment
  \end{itemize}
  \item Identical execution on:
  \begin{itemize}
    \item laptops
    \item CI systems
    \item HPC clusters
  \end{itemize}
\end{itemize}

\vspace{2mm}
\y{\bf Containers turn models into executable artifacts.}

\end{column}

\end{columns}

\vspace{2mm}
\centering
\footnotesize
Reproducibility, portability, and controlled execution are \rtext{non-negotiable} in ML.

\end{frame}
