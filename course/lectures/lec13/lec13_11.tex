%!TEX root = lec13.tex
% ================================================================================
% Lecture 13 — Slide 11
% ================================================================================
\begin{frame}[t,fragile]

\mytitle{Starting Point: A Baseline ML Application}

\begin{columns}[T,totalwidth=\textwidth]

% ------------------------------------------------------------
\begin{column}[T]{0.48\textwidth}
\footnotesize

\vspace{-2mm}
We start with a \y{minimal working ML example}.

\vspace{2mm}
The goal at this stage is simple:
\begin{itemize}
  \item load a pretrained model,
  \item apply it to input data,
  \item inspect the output.
\end{itemize}

\vspace{2mm}
This corresponds to running the notebook:
\begin{itemize}
  \item \texttt{00\_face-detection-onnx.py}
\end{itemize}

\vspace{2mm}
At this point:
\begin{itemize}
  \item the method works,
  \item results can be inspected visually.
\end{itemize}

\vspace{2mm}
\rtext{Nothing here is operational yet.}

\end{column}

% ------------------------------------------------------------
\begin{column}[T]{0.5\textwidth}

\vspace{-4mm}
\includegraphics[width=\textwidth]{../../images/img13/i01.png}

\vspace{1mm}
\footnotesize
The notebook instantiates a \y{pretrained} \y{face-detection model}
and applies it directly to an input image.

\vspace{0mm}
\color{darkgreen}
\tiny
\begin{lstlisting}
def mark_faces(image_filename):
    """Mark all faces recognized in the image"""
    image = PIL.Image.open(image_filename)

    faces = detect_faces(image)

    render_data = detections_to_render_data(
        faces, bounds_color=Colors.GREEN, line_width=3 )
    render_to_image(render_data, image)
    display(image)
\end{lstlisting}

\end{column}

\end{columns}

\end{frame}
