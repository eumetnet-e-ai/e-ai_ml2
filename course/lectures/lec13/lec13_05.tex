%!TEX root = lec13.tex
% ================================================================================
% Lecture 13 — Slide 05
% ================================================================================
\begin{frame}[t,fragile]

\mytitle{The Classical MLOps Cycle}

\begin{columns}[T,totalwidth=\textwidth]

% ------------------------------------------------------------
\begin{column}[T]{0.52\textwidth}
\footnotesize

\textbf{From experiment to operation}

\begin{itemize}
  \item \y{Data ingestion} and preprocessing
  \item \y{Model training} and validation
  \item \y{Deployment} into an operational environment
\end{itemize}

\vspace{2mm}
This cycle extends the classical DevOps loop by explicitly
including \y{data and models}.

\vspace{6mm}
\color{darkgreen}\textbf{
Modern MLOps has many elements of typical \g{NWP development
cycles}. NWP also integrates data, e.g.\ orography, canopy layers,
and real data through observations.}

\end{column}

% ------------------------------------------------------------
\begin{column}[T]{0.46\textwidth}
\footnotesize

\textbf{Operational feedback loop}

\begin{itemize}
  \item \y{Monitoring} of system and model performance
  \item Detection of \rtext{drift and degradation}
  \item Triggering \y{retraining or rollback}
\end{itemize}

\vspace{2mm}
\textbf{Key property}

\begin{itemize}
  \item The cycle is \rtext{continuous}, not linear
  \item Operations actively influence development
\end{itemize}

\end{column}

\end{columns}

\vspace{2mm}
\footnotesize
\hfill{\bf MLOps is a closed loop, not a one-time deployment.}

\end{frame}
