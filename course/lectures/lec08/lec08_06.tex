%!TEX root = lec08.tex
% ================================================================================
% Lecture 08 — Slide 06
% ================================================================================
\begin{frame}[t,fragile]

\mytitle{Example A: Core Implementation Idea}

\begin{columns}[T,totalwidth=\textwidth]

% --- Left column ---------------------------------------------------------------
\begin{column}[T]{0.35\textwidth}

\textbf{Step 1: Symbolic preprocessing}

\begin{itemize}
  \item Reduce the wind field to a few key descriptors
  \item Physics-informed, human-designed
  \item Low-dimensional and interpretable
\end{itemize}

\vspace{2mm}
\rtext{This step encodes \emph{what matters} for the language model.}

\end{column}

% --- Right column --------------------------------------------------------------
\begin{column}[T]{0.66\textwidth}

\textbf{Step 2: Prompt-based text generation}

\begin{codeonly}{Symbolic prompt and text generation}
summary = f"Mean speed: {vmean:.1f} m/s, "
summary += f"Direction: {direction}"

inputs = tokenizer(
    summary,
    return_tensors="pt"
)

outputs = model.generate(**inputs)
\end{codeonly}

\vspace{1mm}
In this example the \y{LLM sees only text, not physics}.

\end{column}

\end{columns}

\end{frame}
