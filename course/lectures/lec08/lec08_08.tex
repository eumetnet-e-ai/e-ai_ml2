%!TEX root = lec08.tex
% ================================================================================
% Lecture 08 — Slide 08
% ================================================================================
\begin{frame}[t,fragile]

\mytitle{Example B: Visual Multimodality}

\begin{columns}[T,totalwidth=\textwidth]

% --- Left column ---------------------------------------------------------------
\begin{column}[T]{0.48\textwidth}

\textbf{Key difference to Example A}

\begin{itemize}
  \item No symbolic preprocessing
  \item No explicit speed or direction given
  \item The model receives an \textbf{image only}
  \item Converted to RGB pixel values
\end{itemize}

\vspace{2mm}
All physical structure must be inferred visually.

{\tiny
\begin{lstlisting}
# Draw the canvas
fig.canvas.draw()

buf = np.asarray(fig.canvas.buffer_rgba())
image = buf[..., :3]   # drop alpha channel

# Normalize to [0, 1]
return image.astype(np.float32) / 255.0
\end{lstlisting}
}

\end{column}

% --- Right column --------------------------------------------------------------
\begin{column}[T]{0.48\textwidth}

\textbf{Multimodal processing pipeline}

\begin{itemize}
  \item Wind field rendered as image
  \item Vision encoder extracts features
  \item Features are mapped to language space
  \item Language model generates text
\end{itemize}

\vspace{2mm}
This is \emph{true} multimodal learning.

\vspace{1mm}
{\tiny
\begin{lstlisting}
# -----------------------------------------
# Render the wind field as an image
# This image is the ONLY input seen by
# the multimodal model.
# -----------------------------------------
img = render_wind_image(LON, LAT, U, V)
\end{lstlisting}
}

\end{column}

\end{columns}

\end{frame}
