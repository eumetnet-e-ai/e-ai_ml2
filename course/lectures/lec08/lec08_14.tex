%!TEX root = lec08.tex
% ================================================================================
% Lecture 08 — Slide 14
% ================================================================================
\begin{frame}[t]

\mytitle{Radar Reflectivity as Multimodal Input}

\begin{columns}[T,totalwidth=\textwidth]

% --- Left column ---------------------------------------------------------------
\begin{column}[T]{0.48\textwidth}

\textbf{Weather radar data}

\begin{itemize}
  \item Active remote sensing (microwave)
  \item Measures backscattered power
  \item Proxy for precipitation intensity
\end{itemize}

\vspace{2mm}
Radar reflectivity is expressed in
\[
\text{dBZ} = 10 \log_{10}(Z)
\]

\vspace{1mm}
with $Z$ the radar reflectivity factor.

\vspace{2mm}
Radar images are routinely interpreted visually by forecasters.

\end{column}

% --- Right column --------------------------------------------------------------
\begin{column}[T]{0.48\textwidth}

\vspace{-1cm}
\includegraphics[width=7cm]{../../images/img08/radar_map_germany_crop.png}

\vspace{-1mm}
{\footnotesize
DWD radar composite with state borders.
Colors indicate reflectivity intensity (dBZ).
}

\end{column}

\end{columns}

\end{frame}
