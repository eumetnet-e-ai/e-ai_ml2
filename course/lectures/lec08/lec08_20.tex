%!TEX root = lec08.tex
% ================================================================================
% Lecture 08 — Slide 20
% ================================================================================
\begin{frame}[t]

\mytitle{Cloud Top Height as Multimodal Input}

\begin{columns}[T,totalwidth=\textwidth]

% --- Left column ---------------------------------------------------------------
\begin{column}[T]{0.49\textwidth}

\vspace{-1mm}
\textbf{What is Cloud Top Height (CTH)?}

\begin{itemize}
  \item Satellite-derived cloud product
  \item Estimates height of the upper cloud boundary
  \item Derived from thermal infrared observations
\end{itemize}

\vspace{0mm}
CTH is typically expressed in meters above sea level
and provides information on:
\begin{itemize}
  \item vertical cloud structure
  \item convective depth
  \item storm intensity
\end{itemize}


\end{column}

% --- Right column --------------------------------------------------------------
\begin{column}[T]{0.48\textwidth}

\includegraphics[width=\linewidth]{../../images/img08/cth_map.png}

\vspace{1mm}
{\footnotesize
Satellite-derived Cloud Top Height (CTH) over Central Europe.
Higher values correspond to deeper cloud systems.
}

\end{column}

\end{columns}

\vspace{1mm}
High CTH values often indicate \y{deep convection}.

\end{frame}
