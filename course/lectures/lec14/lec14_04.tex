%!TEX root = lec14.tex
% ================================================================================
% Lecture 14 — Slide 04
% ================================================================================
\begin{frame}[t,fragile]

\mytitle{Starting Locally: A Concrete Git Hook Example}

\begin{columns}[T,totalwidth=\textwidth]

% ------------------------------------------------------------
\begin{column}[T]{0.52\textwidth}
\footnotesize

{\bf Black} is an \y{automatic code formatter} for Python.

{\sl 
It rewrites Python source code into a single, consistent style, without asking questions.
}

\vspace{2mm}
\textbf{Minimal pre-commit hook}

\begin{codeonly}{.git/hooks/pre-commit}
#!/bin/sh
pytest || exit 1
black .
\end{codeonly}

\vspace{1mm}
This hook is executed automatically when running:
\begin{itemize}
  \item \texttt{git commit}
\end{itemize}

If tests fail, the commit is \rtext{blocked}.

\end{column}

% ------------------------------------------------------------
\begin{column}[T]{0.44\textwidth}
\footnotesize

\textbf{What this enforces locally}

\begin{itemize}
  \item Code must be \y{syntactically correct}
  \item Tests must \y{pass}
  \item Code formatting is \y{consistent}
\end{itemize}

\vspace{2mm}
Key properties:
\begin{itemize}
  \item Runs \rtext{before} code leaves the laptop
  \item No CI server involved
  \item Immediate feedback to the developer
\end{itemize}

\end{column}

\end{columns}

\vspace{2mm}
\footnotesize
\rtext{\bf Git hooks enforce local discipline — not global policy.}

\end{frame}
