%!TEX root = lec14.tex
% ================================================================================
% Lecture 14 — Slide 06
% ================================================================================
\begin{frame}[t,fragile]

\mytitle{A Git Hook in Action: What Actually Happens}

\begin{columns}[T,totalwidth=\textwidth]

% ------------------------------------------------------------
\begin{column}[T]{0.48\textwidth}
\footnotesize

\vspace{-2mm}
\textbf{Before committing}

\begin{itemize}
  \item Python test file is \rtext{poorly formatted}
  \item Code is syntactically valid
  \item Tests pass when run manually
\end{itemize}

\vspace{0mm}
After running \texttt{black}:
\begin{itemize}
  \item Formatting is rewritten automatically
  \item No code logic is changed
\end{itemize}

\textbf{During \texttt{git commit}}

\begin{itemize}
  \item \texttt{pytest} is executed automatically
  \item \texttt{black} is executed automatically
  \item Both run via the \y{pre-commit hook}
\end{itemize}

\vspace{0mm}
\footnotesize
\rtext{\bf The hook turns a manual checklist into an automatic guarantee.}

\end{column}

% ------------------------------------------------------------
\begin{column}[T]{0.42\textwidth}
\footnotesize


\vspace{-3mm}
{\bf Outcome:}
\begin{itemize}
  \item Tests pass
  \item Formatting is consistent
  \item Commit is \y{accepted}
\end{itemize}

\hspace{-3mm}\color{black}\rule{0.5pt}{4cm}

\vspace{-4.3cm}
\begin{minipage}{7cm}
\tiny\color{red}
\begin{lstlisting}
def add( a ,b ):
  return a+b
def test_answer( ):
  print("Testing add(1, 3) == 4")
  assert add(1,3)==4
\end{lstlisting}

\vspace{-2mm}
\color{darkgreen}
\begin{lstlisting}
def add(a, b):
    return a + b
def test_answer():
    print("Testing add(1, 3) == 4")
    assert add(1, 3) == 4
\end{lstlisting}
\end{minipage}

\end{column}

\end{columns}

\end{frame}
