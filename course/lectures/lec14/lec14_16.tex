%!TEX root = lec14.tex
% ================================================================================
% Lecture 14 — Slide 16
% ================================================================================
\begin{frame}[t,fragile]

\mytitle{Synthetic Forecast Generation (moving patterns)}

\begin{columns}[T,totalwidth=\textwidth]

% ------------------------------------------------------------
\begin{column}[T]{0.58\textwidth}
\footnotesize
\vspace{-2mm}

\textbf{Idea}

\vspace{0mm}
We generate physically plausible \y{spatio-temporal structure}:
\begin{itemize}
  \item coherent ``weather patterns'' move eastward with time
  \item correlated variables:
  \begin{itemize}
    \item pressure wave $\rightarrow$ wind via spatial gradients
    \item temperature advected + noise
    \item precipitation linked to fronts (thresholded patterns)
  \end{itemize}
\end{itemize}

\vspace{-1mm}
\begin{minipage}{6cm}
\begin{lstlisting}[basicstyle=\ttfamily\tiny]
# grid (Europe)
lon2d, lat2d = np.meshgrid(lon, lat)
# moving phase (eastward shift with lead time)
phase = kx * lon2d + ky * lat2d - omega * lead_hours[t]

# temperature: advected pattern
t2m[t] = 10.0 + 6.0*np.cos(phase + 0.7) + 0.8*rng.normal(size=lon2d.shape)
\end{lstlisting}
\end{minipage}

\end{column}

% ------------------------------------------------------------
\begin{column}[T]{0.38\textwidth}
\footnotesize
\vspace{-2mm}

\textbf{Result}

\vspace{1mm}
Fields are realistic enough to demonstrate:
\begin{itemize}
  \item map snapshots
  \item point extraction
  \item patch-based ML sampling
\end{itemize}

\vspace{2mm}
\textbf{Example snapshot}

\vspace{1mm}
\includegraphics[width=5cm]{../../images/img14/zarr_ex_field_120.png}
Lead Time: 120h
\end{column}

\end{columns}

\end{frame}
