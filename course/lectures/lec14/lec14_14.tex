%!TEX root = lec14.tex
% ================================================================================
% Lecture 14 — Slide 14
% ================================================================================
\begin{frame}[t,fragile]

\mytitle{Zarr: Limitations and When to Use Alternatives}

\begin{columns}[T,totalwidth=\textwidth]

% ------------------------------------------------------------
\begin{column}[T]{0.50\textwidth}
\footnotesize

\textbf{Limitations}

\begin{itemize}
  \item Limited native integration with ML frameworks
  \item Metadata overhead for many small arrays
  \item Not well suited for tabular data
  \item Care needed for concurrent writes
\end{itemize}

\vspace{2mm}
Zarr adds complexity that is not always justified.

\end{column}

% ------------------------------------------------------------
\begin{column}[T]{0.46\textwidth}
\footnotesize

\textbf{When to choose alternatives}

\begin{itemize}
  \item Tabular ML data: \y{Parquet}, Arrow
  \item TensorFlow pipelines: \y{TFRecord}
  \item PyTorch streaming: \y{WebDataset}
  \item Small datasets: NetCDF, HDF5, NPY
\end{itemize}

\vspace{2mm}
\textbf{Guiding principle}

\begin{itemize}
  \item Choose formats based on data structure
  \item Optimize for access pattern, not fashion
\end{itemize}

\end{column}

\end{columns}

\vspace{2mm}
\footnotesize
\rtext{\bf Zarr is powerful when the problem matches the format.}

\end{frame}
