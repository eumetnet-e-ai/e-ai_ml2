%!TEX root = lec14.tex
% ================================================================================
% Lecture 14 — Slide 13
% ================================================================================
\begin{frame}[t,fragile]

\mytitle{Zarr: Why It Is Attractive for AI/ML}

\begin{columns}[T,totalwidth=\textwidth]

% ------------------------------------------------------------
\begin{column}[T]{0.50\textwidth}
\footnotesize

\textbf{Design principles}

\begin{itemize}
  \item Chunked, array-based storage
  \item Designed for cloud and HPC
  \item Partial reads without full downloads
\end{itemize}

\vspace{2mm}
\textbf{Performance features}

\begin{itemize}
  \item Parallel reads of independent chunks
  \item Flexible chunk layout
  \item Compression per chunk
\end{itemize}

\end{column}

% ------------------------------------------------------------
\begin{column}[T]{0.46\textwidth}
\footnotesize

\textbf{Why this helps ML training}

\begin{itemize}
  \item Efficient random sampling
  \item Scales to distributed workers
  \item Reduces I/O bottlenecks
\end{itemize}

\vspace{2mm}
\textbf{Typical use cases}

\begin{itemize}
  \item Images and video
  \item Satellite and geospatial data
  \item Scientific simulation output
\end{itemize}

\end{column}

\end{columns}

\vspace{2mm}
\footnotesize
\rtext{\bf Zarr optimizes data access patterns — not the ML logic itself.}

\end{frame}
