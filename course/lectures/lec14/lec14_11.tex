%!TEX root = lec14.tex
% ================================================================================
% Lecture 14 — Slide 11
% ================================================================================
\begin{frame}[t,fragile]

\mytitle{CI/CD Tools: Automation and Execution}

\begin{columns}[T,totalwidth=\textwidth]

% ------------------------------------------------------------
\begin{column}[T]{0.48\textwidth}
\footnotesize

\y{\textbf{CI orchestration}}

\begin{itemize}
  \item \textbf{GitHub Actions} \\
  {\tiny Event-driven CI system integrated into GitHub, executing workflows
  defined in YAML on managed runners.}

  \item \textbf{GitLab CI} \\
  {\tiny Pipeline-based CI system configured via \texttt{.gitlab-ci.yml},
  supporting self-hosted and specialized runners (e.g.\ HPC, GPU).}

  \item \textbf{Jenkins} \\
  {\tiny Standalone automation server using scripted pipelines,
  common in legacy and enterprise environments.}
\end{itemize}

\end{column}

% ------------------------------------------------------------
\begin{column}[T]{0.48\textwidth}
\footnotesize

\y{\textbf{Execution environments}}

\begin{itemize}
  \item \textbf{Python virtual environments} \\
  {\tiny Isolate Python dependencies to ensure reproducible runtime behavior.}

  \item \textbf{Containers (Docker, Apptainer)} \\
  {\tiny Package applications and dependencies into portable,
  reproducible execution units across systems.}
\end{itemize}

\vspace{2mm}
\y{\textbf{Artifacts}}

\begin{itemize}
  \item \textbf{CI artifacts and registries} \\
  {\tiny Store outputs such as logs, test reports, trained models,
  and container images produced during pipelines.}
\end{itemize}

\end{column}

\end{columns}

\vspace{4mm}
\centering
\footnotesize
\rtext{\bf CI platforms enforce rules; environments make them reproducible.}

\end{frame}
