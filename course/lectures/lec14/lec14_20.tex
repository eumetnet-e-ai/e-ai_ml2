%!TEX root = lec14.tex
% ================================================================================
% Lecture 14 — Slide 20
% ================================================================================
\begin{frame}[t,fragile]

\mytitle{Town Forecast Dynamics (build time series table)}

\begin{columns}[T,totalwidth=\textwidth]

% ------------------------------------------------------------
\begin{column}[T]{0.54\textwidth}

\footnotesize
\vspace{-1mm}
\textbf{Prepare time series data}

\vspace{0mm}
We convert the extracted point dataset into a clean table:
\begin{itemize}
  \item convert times to \texttt{datetime}
  \item convert lead time to hours
  \item compute derived variables (e.g.\ wind speed)
\end{itemize}

\vspace{-2mm}
\begin{minipage}{7.2cm}
\begin{lstlisting}[basicstyle=\ttfamily\tiny]
# ------------------------------------------------------------
# Create a clean timeseries table
# ------------------------------------------------------------
lead_h = (pt["lead_time"].values / np.timedelta64(1, "h")).astype(int)

df = pd.DataFrame({
    "valid_time": pd.to_datetime(pt["valid_time"].values),
    "lead_h": lead_h,
    "t2m_C": pt["t2m"].values,
    "wind_ms": np.sqrt(pt["u10"].values**2 + pt["v10"].values**2),
    "mslp_hPa": pt["mslp"].values,
    "tp_mm": pt["tp"].values,
})
\end{lstlisting}
\end{minipage}

\end{column}

% ------------------------------------------------------------
\begin{column}[T]{0.44\textwidth}

\footnotesize
\vspace{-1mm}

\includegraphics[width=6.2cm]{../../images/img14/zarr_t2m_dynamics.png}

\vspace{2mm}
\includegraphics[width=6.2cm]{../../images/img14/zarr_wind_dynamics.png}

\end{column}

\end{columns}

\end{frame}
