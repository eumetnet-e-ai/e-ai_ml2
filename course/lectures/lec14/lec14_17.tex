%!TEX root = lec14.tex
% ================================================================================
% Lecture 14 — Slide 17
% ================================================================================
\begin{frame}[t,fragile]

\mytitle{Writing the Zarr Archive (chunking for point extraction)}

\begin{columns}[T,totalwidth=\textwidth]

% ------------------------------------------------------------
\begin{column}[T]{0.5\textwidth}

\footnotesize
\vspace{-1mm}
\textbf{Zarr write: key decisions}

\vspace{0mm}
We write the dataset as a \y{chunked archive}:
\begin{itemize}
  \item \textbf{time chunk = 1:} fast town time series extraction
  \item \textbf{lat/lon chunks = 60:} moderate patch size for ML sampling
\end{itemize}

\textbf{Chunk layout}

\vspace{1mm}
\begin{itemize}
  \item chunks stored as independent files
  \item town extraction reads \y{few chunks only}
  \item patches match ML mini-batches
\end{itemize}

\vspace{2mm}
We generate chunks that contain exactly \y{one lead time}
and a \y{60$\times$60} spatial patch.


\end{column}

% ------------------------------------------------------------
\begin{column}[T]{0.48\textwidth}

\footnotesize
\vspace{-1mm}
Example: \\
\texttt{(valid\_time, lat, lon) = (1,60,60)}

\begin{minipage}{7cm}
\begin{lstlisting}[basicstyle=\ttfamily\tiny]
# ------------------------------------------------------------
# Zarr write
# ------------------------------------------------------------
OUTDIR = "data"
os.makedirs(OUTDIR, exist_ok=True)

ZARR_PATH = os.path.join(OUTDIR, "demo_eu_forecast.zarr")

# best practice: chunk so point extraction is fast
# - valid_time chunk = 1 (fast timeseries reading)
# - lat/lon chunks ~ moderate
chunked = ds.chunk({"valid_time": 1, "lat": 60, "lon": 60})

# overwrite if exists
if os.path.exists(ZARR_PATH):
    import shutil
    shutil.rmtree(ZARR_PATH)

chunked.to_zarr(ZARR_PATH, mode="w", consolidated=True)
print("Wrote:", ZARR_PATH)
\end{lstlisting}
\end{minipage}

\end{column}

\end{columns}

\end{frame}
