%!TEX root = lec14.tex
% ================================================================================
% Lecture 14 — Slide 07
% ================================================================================
\begin{frame}[t,fragile]

\mytitle{Strengths and Limits of Git Hooks}

\begin{columns}[T,totalwidth=\textwidth]

% ------------------------------------------------------------
\begin{column}[T]{0.50\textwidth}
\footnotesize

\textbf{Strengths}

\begin{itemize}
  \item \y{Extremely fast feedback}. Helps you!!
  \item No external infrastructure required
  \item Works offline
  \item Integrated directly into Git commands
\end{itemize}

\vspace{2mm}
Git hooks are ideal for:
\begin{itemize}
  \item formatting checks
  \item unit tests
  \item catching trivial mistakes early
\end{itemize}

\end{column}

% ------------------------------------------------------------
\begin{column}[T]{0.46\textwidth}
\footnotesize

\textbf{Limitations}

\begin{itemize}
  \item Run only on the developer machine
  \item Not enforced across a team
  \item Can be bypassed or disabled
  \item No neutral execution environment
\end{itemize}

\vspace{2mm}
As a consequence:
\begin{itemize}
  \item Hooks improve discipline
  \item But they do \rtext{not guarantee quality}
\end{itemize}

\end{column}

\end{columns}

\vspace{4mm}
\footnotesize
\rtext{\bf Git hooks accelerate development — CI platforms enforce standards.}

\end{frame}
