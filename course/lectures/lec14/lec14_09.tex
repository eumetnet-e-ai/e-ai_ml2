%!TEX root = lec14.tex
% ================================================================================
% Lecture 14 — Slide 09
% ================================================================================
\begin{frame}[t,fragile]

\mytitle{Code and Test: Defining and Enforcing Behavior}

\begin{columns}[T,totalwidth=\textwidth]

% ------------------------------------------------------------
\begin{column}[T]{0.48\textwidth}
\footnotesize

\textbf{Formatted Code}

\begin{codeonly}{test\_example.py}
def add(a, b):
    return a + b
\end{codeonly}

\vspace{2mm}
This code:
\begin{itemize}
  \item is syntactically correct
  \item is well formatted
  \item may still be \rtext{logically wrong}
\end{itemize}

\vspace{2mm}
Formatting alone cannot guarantee correctness.

\vspace{2mm}
\footnotesize
\y{\rtext{\bf Correct behavior is defined by function tests,}}
\y{\rtext{\bf not by appearance.}}

\end{column}

% ------------------------------------------------------------
\begin{column}[T]{0.48\textwidth}
\footnotesize

\textbf{Test that enforces behavior}

\begin{codeonly}{test\_example.py}
def test_add():
    assert add(2, 3) == 5
\end{codeonly}

\vspace{0mm}
This test:
\begin{itemize}
  \item executes the function
  \item checks the expected result
  \item fails automatically if behavior changes
\end{itemize}

\vspace{0mm}
The test can run:
\begin{itemize}
  \item locally
  \item in Git hooks
  \item in CI pipelines
\end{itemize}

\end{column}

\end{columns}


\end{frame}
