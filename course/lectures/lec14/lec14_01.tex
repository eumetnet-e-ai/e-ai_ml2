%!TEX root = lec14.tex
% ================================================================================
% Lecture 14 — Slide 01
% ================================================================================
\begin{frame}[t,fragile]

\mytitle{Why CI/CD Is Not Optional for AI/ML}

\begin{columns}[T,totalwidth=\textwidth]

% ------------------------------------------------------------
\begin{column}[T]{0.52\textwidth}
\footnotesize

\textbf{What continuously changes in AI/ML}

\begin{itemize}
  \item Model parameters through retraining
  \item Training and validation data
  \item Feature engineering and preprocessing
  \item Hyperparameters and runtime configuration
\end{itemize}

\vspace{2mm}
As a consequence:
\begin{itemize}
  \item System behavior is \rtext{not fixed}
  \item Outputs depend on \y{code, data, and environment}
  \item Small changes can have \rtext{large effects}
\end{itemize}

\end{column}

% ------------------------------------------------------------
\begin{column}[T]{0.44\textwidth}
\footnotesize

\textbf{Why manual workflows break down}

\begin{itemize}
  \item Experiments cannot be reproduced reliably
  \item Results depend on undocumented environments or training data
  \item Errors surface \rtext{late or not at all}
  \item Deployment decisions become guesswork
\end{itemize}

\vspace{2mm}
Without automation:
\begin{itemize}
  \item Models cannot be trusted operationally
  \item Debugging becomes \rtext{forensic work}
\end{itemize}

\end{column}

\end{columns}

\vspace{2mm}
\footnotesize
\rtext{\bf CI/CD is the mechanism that makes AI/ML systems controllable and trustworthy.}

\end{frame}
