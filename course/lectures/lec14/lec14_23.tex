%!TEX root = lec14.tex
% ================================================================================
% Lecture 14 — Slide 23
% ================================================================================
\begin{frame}[t,fragile]

\mytitle{Training Loop and Outcome (loss curve)}

\begin{columns}[T,totalwidth=\textwidth]

% ------------------------------------------------------------
\begin{column}[T]{0.50\textwidth}

\footnotesize
\vspace{-1mm}
\textbf{Training loop}

\vspace{0mm}
We train on randomly sampled spatial patches:
\begin{itemize}
  \item each step draws a new batch: location + time
  \item objective: MSE between predicted and true \texttt{t2m(t+1)}
  \item optimizer: Adam
\end{itemize}

\vspace{1mm}
\textbf{Expected behavior}

\vspace{0mm}
\begin{itemize}
  \item loss decreases quickly in the first iterations
  \item later it stabilizes (noise floor of the synthetic data)
  \item model learns local transport and correlations
\end{itemize}

\end{column}

% ------------------------------------------------------------
\begin{column}[T]{0.48\textwidth}

\footnotesize
\vspace{-8mm}
Example:

\vspace{0mm}
\begin{minipage}{7cm}
\begin{lstlisting}[basicstyle=\ttfamily\tiny]
rng = np.random.default_rng(123)
n_steps = 800
batch_size = 8

loss_hist = []

model.train()
for step in range(1, n_steps + 1):
    Xt, yt = sample_patch_batch(batch_size=batch_size, rng=rng)
    pred = model(Xt)
    loss = loss_fn(pred, yt)

    opt.zero_grad()
    loss.backward()
    opt.step()

    loss_hist.append(float(loss.item()))
\end{lstlisting}
\end{minipage}

\vspace{-2mm}
\includegraphics[width=6.2cm]{../../images/img14/zarr_training_loss.png}

\end{column}

\end{columns}

\end{frame}
