%!TEX root = lec14.tex
% ================================================================================
% Lecture 14 — Slide 05
% ================================================================================
\begin{frame}[t,fragile]

\mytitle{When Are Git Hooks Executed?}

\begin{columns}[T,totalwidth=\textwidth]

% ------------------------------------------------------------
\begin{column}[T]{0.52\textwidth}
\footnotesize

\textbf{Git commands trigger hooks}

Git automatically executes hooks at
\y{well-defined points} in its workflow.

\vspace{2mm}
Common examples:
\begin{itemize}
  \item \texttt{pre-commit} — before a commit is created
  \item \texttt{commit-msg} — to validate commit messages
  \item \texttt{pre-push} — before pushing to a remote
\end{itemize}

\vspace{2mm}
Hooks run \rtext{before} Git completes the command.

\end{column}

% ------------------------------------------------------------
\begin{column}[T]{0.44\textwidth}
\footnotesize

\textbf{Effect on the workflow}

\begin{itemize}
  \item Hook succeeds $\rightarrow$ Git continues
  \item Hook fails $\rightarrow$ Git aborts the command
\end{itemize}

\vspace{2mm}
This means:
\begin{itemize}
  \item Invalid code never enters the repository
  \item Errors are caught \y{immediately}
  \item No manual checks are required
\end{itemize}

\vspace{2mm}
Hooks are \rtext{deterministic}:
same input $\rightarrow$ same outcome.

\end{column}

\end{columns}

\vspace{2mm}
\footnotesize
\rtext{\bf Hooks turn Git commands into enforced quality gates.}

\end{frame}
