%!TEX root = lec14.tex
% ================================================================================
% Lecture 14 — Slide 08
% ================================================================================
\begin{frame}[t,fragile]

\mytitle{Formatting vs Behavior: Two Different Checks}

\begin{columns}[T,totalwidth=\textwidth]

% ------------------------------------------------------------
\begin{column}[T]{0.50\textwidth}
\footnotesize

\vspace{-2mm}
\textbf{Code formatting (Black)}

\begin{itemize}
  \item Enforces a consistent style
  \item Removes whitespace and layout differences
  \item Does \rtext{not} change program logic
\end{itemize}

\vspace{2mm}
Example:
\begin{codeonly}{black reformats code}
def add( a ,b ):
  return a+b
\end{codeonly}

becomes:
\begin{codeonly}{}
def add(a, b):
    return a + b
\end{codeonly}

\end{column}

% ------------------------------------------------------------
\begin{column}[T]{0.46\textwidth}
\footnotesize

\textbf{Behavior testing (pytest)}

\begin{itemize}
  \item Checks whether code does the \y{right thing}
  \item Executes functions and validates results
  \item Fails if behavior changes unexpectedly
\end{itemize}

\vspace{2mm}
Formatting makes code readable.  
Testing makes code \rtext{\bf correct}.

\end{column}

\end{columns}

\vspace{2mm}
\footnotesize
\rtext{\bf Style and correctness are independent concerns.}

\end{frame}
