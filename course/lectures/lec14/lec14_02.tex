%!TEX root = lec14.tex
% ================================================================================
% Lecture 14 — Slide 02
% ================================================================================
\begin{frame}[t,fragile]

\mytitle{AI/ML Is Not Special — But Updating Becomes Complex}

\begin{columns}[T,totalwidth=\textwidth]

% ------------------------------------------------------------
\begin{column}[T]{0.52\textwidth}
\footnotesize

\textbf{From a software engineering perspective}

\begin{itemize}
  \item AI/ML systems are still \y{software systems}
  \item They use standard languages, libraries and toolchains
  \item The same DevOps principles apply
\end{itemize}

\vspace{2mm}
There is \rtext{no special engineering magic} in AI/ML:
\begin{itemize}
  \item version control
  \item testing
  \item packaging
  \item deployment
\end{itemize}

\vspace{2mm}
\footnotesize
\rtext{\bf CI/CD is needed to manage updates, not to handle “AI magic”.}

\end{column}

% ------------------------------------------------------------
\begin{column}[T]{0.44\textwidth}
\footnotesize

\color{darkgreen}
\textbf{Where the real complexity comes from}

\begin{itemize}
  \item \color{darkgreen}
	Frequent retraining with \y{new data}
  \item Multiple preprocessing and feature pipelines
  \item Changing \y{model architectures}
  \item Different \y{loss functions} and objectives
  \item Many runtime and training configurations
\end{itemize}

\vspace{2mm}
As a result:
\begin{itemize}
  \item Updates are \rtext{continuous}
  \item Reproducibility becomes \rtext{non-trivial}
  \item Manual tracking no longer works
\end{itemize}

\end{column}

\end{columns}


\end{frame}
